\chapter{Analysis and Optimization}\label{ch:analysis_and_optimization}

	In this chapter we look at the static analyses performed on the PLR syntax 
	tree before compilation, these include some classical dataflow analyses 
	as well as some analyses that are more specific to the process algebra 
	domain. The results of these analyses can be used to optimize the 
	compilation process, the optimizations are presented and explained, and 
	their implications for debugging support are discussed.
	
\section{Analysing process algebra}
	
	\subsection{Traditional data flow analysis}
	Many of the most useful static analyses that compilers perform are based on 
	\textit{data flow analysis}. For these types of analyses it is necessary to 
	build up a \textit{control flow graph} of the program being analysed. The 
	control flow graph shows which program points lead directly to which other 
	program points, in some cases the reverse control flow graph is needed, 
	which shows for a program point $p$ what its immediate predecessors are. The 
	following snippet of typical imperative code is an example:
	\begin{Exa}
	\label{ex:control_flow}
	\begin{verbatim}
	
                     1: y := 2
                     2: x := 1
                     3: while x < 6 do
                     4: 	  x := x + 1   
                        od
                     5: print x
	\end{verbatim}
	\end{Exa}
	
	In this example the program points are labelled from 1-5. Program points are 
	those points in the program where something happens, expression are 
	evaluated, variables are assigned, etc. The control flow for this example 
	would be (1,2), (2,3), (3,4), (3,5), (4,3) and (4,5). The analysis is then 
	typically performed by having each program point have an input set and an 
	output set, the input set represents the state of the program as the point 
	is reached and is based on the state of its predecessors, the output set 
	represents the state of the program after the program point has been 
	evaluated, and is based on the points input set with some modifications 
	based on what happened at the program point. 
	
	To give a concrete example suppose we have an analysis which is determining 
	for the code snippet in Example~\ref{ex:control_flow} which variables have 
	been assigned at each point in the program. For the program point labelled 1 
	(x := 1), its input set would be $\{y\}$ as $y$ is the only variable that 
	has been assigned at that point. Since program point 1 assigns to the 
	variable $x$ then its output set would be the union of its input set and 
	$\{x\}$, or $\{x,y\}$. The functions used to modify the input set to create 
	the output set are commonly called \textit{Kill} and \textit{Gen}, the 
	\textit{Kill} function removes items from the input set and the \textit{Gen} 
	function adds new items to the output set. 
	
	To get the final result of the analysis these calculations must be repeated 
	for each point in the program until the input and output sets of each become 
	stable. To accomplish this there are many iterative worklist algorithms that 
	can be used, they keep track of which program points change and what other 
	program points must then be re-calculated. There is a lot more to data flow 
	analysis than explained here, for instance whether output sets of 
	predecessors are combined using the union or intersection operator, and what 
	the initial content of the input sets are, but we will not go into more 
	detail on how data flow analysis generally works here. A good resource on 
	the subject is \cite{program_analysis}.
	
	\subsection{Data flow analysis in process algebra}
	
	Two properties of process algebra make it different from imperative 
	languages when it comes to data flow analysis.
	
	\begin{enumerate}
		\item There are no looping constructs. The reason that data flow analysis 
		on imperative languages needs iterative worklist algorithms is because of 
		looping. When looping is not a part of the language then the whole program 
		can be analyzed from top to bottom (or bottom to top) in one pass, each 
		program point only needs to be calculated once.
		
		\item There are branches, but they are never re-joined later in the 
		program. This implies that at every point in the program is is possible to 
		know exactly what path was taken to get to that point. Note that this only 
		holds for forward analysis. Backwards analysis can be seen as multiple 
		branches joining, and so in a backward analysis it is not possible to know 
		at program point $p$ from which branch a particular item in $p$'s input 
		set originates.
		
	\end{enumerate}

	These two properties might not hold for all process algebras in existence, 
	but they certainly hold for both CCS and KLAIM, as well as some other 
	prominent algebras such as CSP, and they do simplify the implementation of 
	these analyses for process algebra. Sections \ref{live_variables} and 
	\ref{reaching_definitions} discuss one backward and one forward data flow 
	analysis and how they were implemented in an extensible way for the PLR.
	
\section{Analyses}
\subsection{Live Variables}\label{live_variables}

  \textit{Live Variables Analysis} is a classic dataflow analysis. Its purpose is to identify at each program point which variables are \textit{live}, that is which variables will be used later on in the program in the paths that follow the program point in question.

\subsection{Reaching Definitions}\label{reaching_definitions}

\subsection{Constant Expressions}



\subsection{Nil Process Elimination}

\subsection{Unmatched Channels}

\subsection{Unused Processes}


	