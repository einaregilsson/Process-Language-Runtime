\chapter{Analysis and Optimization}\label{ch:analysis_and_optimization}

	In this chapter we look at the static analyses performed on the PLR syntax 
	tree before compilation, these include some classical dataflow analyses 
	as well as some analyses that are more specific to the process algebra 
	domain. The results of these analyses can be used to optimize the 
	compilation process, the optimizations are presented and explained, and 
	their implications for debugging support are discussed.
	
\section{Analysing process algebra}
	
	\subsection{Traditional data flow analysis}
	Many of the most useful static analyses that compilers perform are based on 
	\textit{data flow analysis}. For these types of analyses it is necessary to 
	build up a \textit{control flow graph} of the program being analysed. The 
	control flow graph shows which program points lead directly to which other 
	program points, in some cases the reverse control flow graph is needed, 
	which shows for a program point $p$ what its immediate predecessors are. The 
	following snippet of typical imperative code is an example:
	\begin{Exa}
	\label{ex:control_flow}
	\begin{verbatim}
	
                     1: y := 2
                     2: x := 1
                     3: while x < 6 do
                     4: 	  x := x + 1   
                        od
                     5: print x
	\end{verbatim}
	\end{Exa}
	
	In this example the program points are labelled from 1-5. Program points are 
	those points in the program where something happens, expression are 
	evaluated, variables are assigned, etc. The control flow for this example 
	would be (1,2), (2,3), (3,4), (3,5), (4,3) and (4,5). The analysis is then 
	typically performed by having each program point have an input set and an 
	output set, the input set represents the state of the program as the point 
	is reached and is based on the state of its predecessors, the output set 
	represents the state of the program after the program point has been 
	evaluated, and is based on the points input set with some modifications 
	based on what happened at the program point. 
	
	To give a concrete example suppose we have an analysis which is determining 
	for the code snippet in Example~\ref{ex:control_flow} which variables have 
	been assigned at each point in the program. For the program point labelled 1 
	(x := 1), its input set would be $\{y\}$ as $y$ is the only variable that 
	has been assigned at that point. Since program point 1 assigns to the 
	variable $x$ then its output set would be the union of its input set and 
	$\{x\}$, or $\{x,y\}$. The functions used to modify the input set to create 
	the output set are commonly called \textit{Kill} and \textit{Gen}, the 
	\textit{Kill} function removes items from the input set and the \textit{Gen} 
	function adds new items to the output set. This can be shown as (for program 
	point $p$):
	
	\begin{center}
	$p_{output} = p_{input}\ \cup\ Gen(p)\ \setminus\ Kill(p) $
	\end{center}
	
	To get the final result of the analysis these calculations must be repeated 
	for each point in the program until the input and output sets of each become 
	stable. The result is an approximation, either overapproximation (something 
	\textit{may} happen on the path to the program point) or an underapproximation
	(something \textit{must} have happened on the path to the program point).  To 
	calculate the input and output sets of each program points an iterative 
	worklist algorithm is used. There are many variations of these algorithms, 
	they keep track of which program points change and which other program points 
	must then be re-calculated. There is a lot more to data flow analysis than 
	explained here, for instance whether output sets of predecessors are combined 
	using the union or intersection operator, and what the initial content of the 
	input sets are, but we will not go into more detail on how data flow analysis 
	generally works here, this is meant only as a brief introduction before 
	explaining the analyses performed in the PLR. A more formal introduction to 
	the subject of data flow analysis can be found in \cite{program_analysis}.
	
	\subsection{Data flow analysis in process algebra}
	
	Two properties of process algebra make it different from imperative 	
	languages when it comes to data flow analysis.
	
	\begin{enumerate}
		\item There are no looping constructs. The reason that data flow analysis 
		on imperative languages needs iterative worklist algorithms is because of 
		looping. When looping is not a part of the language then the whole program 
		can be analyzed from top to bottom (or bottom to top) in one pass, each 
		program point only needs to be calculated once.
		
		\item There are branches, but they are never re-joined later in the 
		program. This implies that at every point in the program is is possible to 
		know exactly what path was taken to get to that point. Note that this only 
		holds for forward analysis. Backwards analysis can be seen as multiple 
		branches joining, and so in a backward analysis it is not possible to know 
		at program point $p$ from which branch a particular item in $p$'s input 
		set originates.
		
	\end{enumerate}

	These two properties might not hold for all process algebras in existence, 
	but they certainly hold for both CCS and KLAIM, as well as some other 
	prominent algebras such as CSP, and they do simplify the implementation of 
	these analyses for process algebra. Sections \ref{live_variables} and 
	\ref{reaching_definitions} discuss one backward and one forward data flow 
	analysis and how they were implemented in an extensible way for the PLR.
	
\section{Analyses}

\subsection{Live Variables}\label{live_variables}

  \textit{Live Variables Analysis} is a classic data flow analysis. Its purpose 
  is to identify at each program point which variables are \textit{live}, that 
  is which variables will be read later on in the program in the paths that 
  follow the program point in question. This is a backward analysis and is 
  traditionally used to identify assignments to variables that have no effect,
  for instance if the variable $x$ is assigned at program point $p$, but along 
  all paths that follow $p$ the variable is either never read, or assigned to 
  again before being read then the assignment at $p$ has no effect and can be 
  eliminated. The analysis is an over-approximation, we cannot safely eliminate 
  the assignment to $x$ if there \textit{may} be a path after $p$ where $x$ is
  read. This implies that the input set of $p$ will be the \textit{union} of the
  output sets of its pre-decessors (here the pre-decessors actually mean the 
  program points that follow $p$ as it is a backward analysis).
  
  In the analysis of the PLR syntax tree, a process is considered a program 
  point. Additionally, syntax nodes that represent a process constant being 
  defined are considered program points, as they may contain defining occurences
  of variables. For instance, in $ CS(x,y) \defeq a . 0$ we would consider 
  $CS(x,y)$ a program point, since it defines the variables $x$ and $y$, and it
  can be beneficial at that point to know whether the initial value of those 
  variables was ever used in the process. We have two helper functions, 
  \textit{assigned(x)} takes in an action and returns a set of the variables 
  assigned in that action. An action in this case can be the receiving of 
  values through a channel or sending values through a channel. The other 
  helper function, \textit{read(x)} takes in an expression and returns all 
  variables evaluated in that expression, or takes in an action and returns
  all variables evaluated in that action. Figure~\ref{fig:read_assigned} 
  shows some examples of the use of \textit{read} and \textit{assigned}, 
  Figure~\ref{fig:killgen_livevariables} shows how the \textit{Kill} and
  \textit{Gen} functions are defined for each of the process types in the
  PLR syntax tree.
  
	\begin{figure}[h!]
	\label{fig:read_assigned}
	\caption{Examples of the \textit{read} and \textit{assigned} functions}
	\begin{ARRAY}{r l l}
  assigned(\mathrm{channel}(x,y)) & = & \{x,y\} \vspace{5pt}\\
  read(\overline{(\mathrm{channel}(z+1,a-b)}) & = & \{z,a,b\} \vspace{5pt}\\
  read(x+1/z) & = & \{x,z\} \\
	\end{ARRAY}
	\end{figure}
  
  \begin{figure}[h!]
  \label{fig:killgen_livevariables}
  \caption{\textit{Kill} and \textit{Gen} functions for Live Variables}
  \begin{ARRAY}{r l l}
  		
  		Kill(a.P) & = & assigned(a) \\
  		Gen(a.P) & =& read(a) \vspace{7pt}\\
      
      Kill(P \mid Q) & = & \emptyset \\
      Gen(P \mid Q) & = & \emptyset \vspace{7pt}\\
   
      Kill(P + Q) & = & \emptyset \\
      Gen(P + Q) & = & \emptyset \vspace{7pt}\\
  
      Kill(0) & = & \emptyset \\
      Gen(0) & = & \emptyset \vspace{7pt}\\
      
      Kill(\mathrm{if}\ bexp\ \mathrm{then}\ P\ \mathrm{else}\ Q) & = & \emptyset\\
      Gen(\mathrm{if}\ bexp\ \mathrm{then}\ P\ \mathrm{else}\ Q) & = & read(bexp)\vspace{7pt}\\

      (\mathrm{process\ definitions}) & &\\
      Kill(K(x,y,z) \defeq) & = & \{x,y,z\} \\
      Gen(K(x,y,z) \defeq) & = & \emptyset \vspace{7pt}\\

      (\mathrm{process\ invocations}) & &\\
      Kill(K(exp_1,...,exp_n)) & = & \emptyset \\
      Gen(K(exp_1,...,exp_n)) & = & read(exp_1) \cup ... \cup\ read(exp_n) \\

  
  \end{ARRAY} 
  \end{figure}
  
  The actual implementation of the \textit{assigned()} and \textit{read()} 
  functions is done with two properties that are present on all syntax nodes
  that derive from either \textsf{Process} or \texts{Action}. These properties
  are \textsf{AssignedVariables} and \textsf{ReadVariables} and return the
  assigned and read variables of a process or action. Having these as properties
  of the syntax nodes instead of as part of the implementation of the analysis 
  allows for greater extensibility. Languages implemented using the PLR that 
  have their own custom syntax nodes simply need to override these properties
  and can then use the analysis without further modifications. This is  
  the case in the KLAIM implementation discussed in Chapter~\ref{ch:klaim}, 
  it has its own custom actions and a custom process type and they simply 
  implement these properties.
  
  Constructing the flow graph is trivial, since there are no loops or joining
  branches. In fact, a special flow graph is not constructed, instead the PLR
  syntax tree is used directly. To make it simpler, each \textsf{Process} node
  is required to implement a \textsf{FlowsTo} property which is a list of all
  the processes it can flow to. For an \texttt{if-else} process these are the 
  if and else branches, for parallel composition these are the composed 
  processes, etc. In addition each node in the syntax tree has a \textsf{Parent} 
  property which references its parent and can be used as a reverse flow graph.
  Again, having these properties directly on the syntax tree is useful so
  that additional process types can be added without having to modify the
  analysis code. The input and output sets are stored in a a \textsf{Tag} property
  which is also on each syntax node, this is a generic object reference which
  analyses may use temporarily to store data associated with each node.
  

\subsection{Reaching Definitions}\label{reaching_definitions}

\subsection{Constant Expressions}

\subsection{Nil Process Elimination}

\subsection{Unmatched Channels}

\subsection{Unused Processes}


	