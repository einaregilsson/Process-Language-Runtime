\chapter{Abstract}

Process languages, also known as Process Algebras or Process Calculi, are languages that are built up of distinct processes communicating with each other. Different process languages have different features, but most of the prominent ones have a common subset of features. That subset includes action prefixing, parallel composition of subprocesses and non-deterministic choice between paths.

The .NET platform is a popular development platform. One of its strengths is
that it supports multiple languages running on the same underlying virtual machine. The languages compile down to a common bytecode format which means that the languages can interoperate and different parts of the same application can be built in different languages.

This thesis explores how well Process Languages can be integrated into the .NET environment and how they can interoperate with code written in other languages.  The design and implementation of an extensible compiler backend and a runtime library for process languages are presented, as well as two case studies of languages implemented using the common compiler and runtime. Finally a quick overview is given on how to integrate a process language into a state-of-the-art integrated development environment.

