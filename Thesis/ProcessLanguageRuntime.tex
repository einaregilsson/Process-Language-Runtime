\chapter{Process Language Runtime}

	This chapter presents the design and implementation of the Process Language 
	Runtime, an extensible compiler backend and a runtime library for running 	
	process languages on the .NET platform. 

\section{Inspiration}

	The inspiration for the Process Language Runtime comes from the Dynamic 
	Language Runtime \TODO{NEED REF} (DLR), a framework from Microsoft for 	
	developing dynamic languages on top of the Common Language Runtime. Since 
	the .NET Common Intermediate Language is statically typed it is ill-suited
	for dynamic languages such as Python, Ruby or JavaScript. To clarify, 
	statically typed languages are languages where the type of a variable is 
	known at compile time and the type of a variable never changes, while in a 
	dynamically typed language a single variable can contain objects of 	
	different types at different times during program execution. The idea of 
	having a common abstract syntax tree for different languages comes from the 
	DLR. However, the DLR was not directly used for this project since its 
	abstract syntax tree is mainly concerned with traditional constructs for 
	imperative programming languages and constructs to support dynamic typing, 
	while the purpose of this project is to provide constructs common to process 
	languages, and dynamic typing is not an issue we are concerned with.

\section{Overview}

	The PLR is a single .NET assembly, named PLR.dll. This assembly contains the 
	abstract syntax tree and associated helper objects as well as all the 
	classes used at runtime. The PLR does not contain any lexer or parser and 
	generally has no notion of any concrete syntax. Creating a lexer and parser 
	is the responsibility of individual language implementations, the PLR takes 
	over once an input file has been parsed and used to construct a PLR abstract 
	syntax tree. 
	
	It is important to note that the PLR does not, nor is it meant 
	to, support all constructs of all process languages. Creating a superset of 
	all existing process languages has never been the goal, instead the goal is 
	to provide a common subset of the most common constructs found in these 
	languages and to make it easy to extend with specific new constructs needed 
	for specific languages. To enable this, the classes and interfaces in the 
	PLR have been engineered to make them easy to subclass and implement.
	
	As the PLR contains classes needed at runtime it must be distributed with 
	any compiled process language application. However, an option is present in 
	the compilation stage that allows the PLR assembly to be embedded in the 
	final compiled executable program, and can optionally embed any additional 
	runtime libraries that specific languages require. This allows a process 
	language program to be distributed as a single file without any external 
	dependencies other than the .NET framework itself.

	The PLR itself is written in the C\# programming language using Visual 
	Studio 2008 as the development environment. It has a dependency on the NUnit 
	unit test framework, however this dependency is only needed when running 
	internal unit tests and so does not need to be distributed with the PLR 
	assembly. The source code for the PLR is licensed under the General Public 
	License (GPL) v3.0.
	
\section{Abstract Syntax Tree}

	The PLR abstract syntax tree is the component that generates CIL 
	bytecode to run a process language application. Once an abstract syntax tree 
	has been constructed, a call to a \method{Compile} method on the tree's root 
	node with the appropriate parameters will create an executable .NET 
	assembly. 
	
	The architecture of the abstract syntax tree is based on Object Oriented 
	principles, namely that an object contains data and methods to operate on 
	that data. As such, each node in the abstract syntax tree knows how to 
	compile itself, 

\section{Runtime Library}
