\chapter{Integrated Development Environment}\label{ch:ide}

	One of the biggest differences between academic programming languages and 
	industrial programming languages is the level of tool support. Academic 
	languages traditionally are edited in text editors and compiled with command 
	line tools, while industrial languages usually have integrated development 
	environments (IDE's). These environments offer a wide range of features such 
	as syntax highlighting, instant visual warnings about syntax errors, 
	background compilation, automatic listing of available methods and variables 
	(IntelliSense), built in debuggers and refactoring. One of the goals of this 
	project was to explore how well the CCS language could be integrated into 
	one of these environments and how it could benefit from the features they 
	have to offer. This chapter presents the results of this exploration. 
	
\section{Choice of Integrated Development Environment}
	
	When choosing which IDE would be most suitable for CCS two environments 
	stood out, Microsoft Visual Studio 2008 and SharpDevelop. Eclipse 
	was briefly considered as well since it provides good plugin support, but 
	was dismissed since it is built on Java and since this project is focused on 
	integrating with the .NET framework it seemed natural to go with a .NET 
	development environment. Below the two candidate environments are described 
	and reasons given for choosing one of them.
	
	\subsection{Microsoft Visual Studio 2008}
	Microsoft's Visual Studio is the most popular environment for .NET 
	development. Visual Studio versions are released alongside new versions of 
	the .NET framework itself and each new version takes full advantage of and 
	supports all the new features in the .NET framework. The latest version as 
	of this writing is Visual Studio 2008 which supports the .NET framework 3.5. 
	Visual Studio has an extensive extensibility API based on COM technology, an 
	older technology for interaction between programs written in different 
	programming languages. Languages are integrated by creating \textit{language 
	services}, the Visual Studio program itself is simply a host for these 
	services. The languages that come with the .NET Framework, C\# and Visual 
	Basic.NET have their own language services that do not have any special 
	access to Visual Studio, this implies that a language service for a new 
	language can be made to offer all the same features as those supported by 
	the built in languages. The downside of Visual Studio is that its 
	extensibility API is fairly complicated and hard to work with. Another 
	drawback is that even though there exist a fair number of samples for 
	language services, the professional level services for languages like C\# 
	and Visual Basic.NET are not available as open source so it is not possible 
	to look at them for inspiration.
	
	\subsection{SharpDevelop}
	The second candidate for a CCS development environment was SharpDevelop, an 
	open source IDE written entirely in C\#. SharpDevelop is very similar to 
	Visual Studio in look and feel, and offers many of the same features. Its 
	extensibility API is entirely in .NET and is in many ways cleaner and 
	clearer than Visual Studio's API. It also has the benefit of being open 
	source software, so it is easy to get a clearer picture of its architecture, 
	and view the source for other language services, even the built in ones for 
	C\# and Visual Basic.NET. The architecture of SharpDevelop itself has even 
	been the subject of a book, \cite{sharpdevelop}. The drawbacks are that its 
	debugger is inferior to Visual Studio's, the application itself is slower, 
	and it is not as well known as Visual Studio. It also does not offer all the 
	same features as Visual Studio, a notable feature that is missing is the 
	ability to highlight syntax errors as the user types in code.
	
	\subsection{Chosen environment}
	After researching both environments, Visual Studio 2008 was chosen as the one
	to implement CCS's language service in. This was based primarily on the fact
	that Visual Studio is the IDE of choice for most .NET developers, it is 
	fast, offers great debugging support and real time syntax checking. While 
	the standard versions of Visual Studio are not free, it is possible to 
	download only the Visual Studio shell and distribute it for free. The shell 
	is the Visual Studio program itself without any language services. This 
	makes it possible to offer the CCS development environment free of charge to 
	anyone who wishes to use it. 
	
\section{Building a language service}
	
	\subsection{Goal}
	The goal of this integration was to be able to use Visual Studio to manage 
	all aspects of working with the CCS language. To achieve that the following 
	features needed to be implemented:
	
	\begin{enumerate}
		\item \textbf{CCS Projects}. The ability to create new projects 
		specifically for CCS applications.

		\item \textbf{Syntax highlighting}. To have different tokens of the 
		language colored differently so that it is easier to see and understand 
		the structure of the code.
		
		\item \textbf{Real time syntax checking}. Display visual warnings about 
		incorrect syntax in the code as it is being written. 
	
		\item \textbf{IntelliSense}. Allow the developer to press a keyboard 
		shortcut and get a list of all available channel names, keywords, variable 
		names and process names in the current application, and insert them at the 
		current location in code. 
		
		\item \textbf{Match braces}. When working with large expressions it 
		can be hard to see which braces (parentheses, curly braces, angle 
		brackets) match, and missing parentheses are a common syntax error. Visual 
		Studio can highlight the matching braces automatically, if the language
		service provides it with the necessary information about which braces 
		match each other.
		
		\item \textbf{Build support}. To be able to compile the code being written 
		from within Visual Studio, using its \textit{Build} menu items and 
		commands. A part of that is being able to use Visual Studio's built in 
		mechanism to search for and add references to other .NET assemblies that 
		the application uses, and pass those references to the compiler at compile 
		time.
		
		\item \textbf{Debugger support}. Launching the application after it had 
		been built and attaching the Visual Studio debugger to the running 
		executable. Also the ability to set breakpoints and step through the code 
		as it is executing.
		
	\end{enumerate}
	
	Visual Studio offers the front-end for all these features, that is the 
	graphical user interface to display them and the infrastructure that calls 
	into the language service's code to get the data necessary for the features 
	to work. However, Visual Studio of course has no knowledge of specific 
	languages and so it is the responsibility of each language service to 
	provide the back-end, the code that understands the language, its tokens, 
	its syntax, which items to display in IntelliSense and so on. The 
	implementation of these features is described in the following sections.
	
	\subsection{Visual Studio API}	
	The COM extensibility API for Visual Studio is fairly complicated and 
	unfriendly to use. It is also poorly documented. Fortunately Microsoft has 
	recently released a framework called the Managed Package Framework, or MPF 
	for short. This comes as part of the Visual Studio SDK (Software Development 
	Kit) and is a collection of .NET classes that wrap a lot of the underlying 
	COM interface, making it easier to work with in .NET languages. 
	The MPF classes implement much of the tedious boilerplate code which is 
	necessary and common to all language services. Parts of the Managed Package 
	Framework are released only as source code and are meant to be included 
	directly in language service projects when they are being built. In the 
	source code repository for this project, these files have been marked 
	specifically with a header stating that they are supplied by Microsoft to 
	avoid confusion about which code is original work and which code is 
	borrowed. 

	\subsection{CCS Projects}
	To allow the user to create a new project of type CCS project four main 
	things needed to be implemented.
	
	\begin{enumerate}
		\item A class named \textsf{CCSProjectPackage}, this class inherits from 
		a \textsf{ProjectPackage} that is provided in the Managed Package 
		Framework. It provides information about the project file ending, the 
		project name and a path to the project templates described in item 4 in 
		this list. It doesn't contain any real code, it only provides the 
		necessary properties for Visual Studio to recognize that a new type of 
		project has been registered. To ensure uniqueness of the package it has a 
		globally unique identifier (GUID).
		
		\item A class named \textsf{CCSProjectFactory}, this class inherits from 
		a \textsf{ProjectFactory} class from the Managed Package Framework. It 
		also contains a globally unique identifier and overrides only one method, 
		\textsf{CreateProject()} which returns a \textsf{ProjectNode} instance.
		
		\item A class named \textsf{CCSProjectNode}. This class represents the 
		project once it has been created, it inherits from \textsf{ProjectNode}, 
		again from Managed Package Framework. In this class it is possible to 
		override a lot of behavior, such as what happens when build dependencies 
		are added to the project, which items can be deleted from the project, how 
		to clean the project and many more. This implementation did not require a 
		lot of overrides, since each project only contains one source file so most 
		project possibilities are simply not enabled.
		
		\item Templates for the project needed to be created. The main template is 
		for the project file. The project file defines which items are included 
		and which MSBuild targets (see Section~\ref{msbuild}) should be used to
		build the project. Another template is for a default CCS source file that 
		is included in every project, and contains some simple sample code to get 
		people started. Finally there is a file named \textit{CCS 
		Project.vstemplate}, this contains some metadata about the templates and 
		is the file used by Visual Studio to determine which items to show when 
		new projects are created.
	
	\end{enumerate}
	
	In addition to these items that needed to be implemented, a lot of source 
	code from the Managed Package Framework is necessary to build the project 
	package successfully. These are standard implementations of a number of 
	interfaces Visual Studio requires, they can be overridden for more complex 
	projects than the CCS projects. It was surprising how much source code is 
	needed to do a relatively simple thing like creating a new project type.
	
	\subsection{Syntax Highlighting}
	Syntax highlighting is when different tokens in a language are given 
	different color to help differentiate them when looking at code. This
	is very helpful when looking at code to quickly sense the structure and
	identify problems, and has been a standard feature of development 
	environments as well as most advanced text editors for many years. In many
	common text editors this is simply implemented as lists of tokens and 
	colors for them. The Managed Package Framework however requires that the 
	language service provides an implementation of an interface named 
	\textsf{IScanner}, shown in Figure~\ref{fig:iscanner}. 

	\begin{figure}
	\begin{csharp}
  public interface IScanner {
    bool ScanTokenAndProvideInfoAboutIt(
    		TokenInfo tokenInfo, ref int state);
    void SetSource(string source, int offset);
  }	
\end{csharp}
	\caption{IScanner interface}
	\label{fig:iscanner}
\end{figure}

	The \textsf{SetSource} method of the interface is called by Visual Studio 
	and provides the \textsf{IScanner} with one line of source code at a time, 
	Visual Studio then repeatedly calls \textsf{ScanTokenAndProvideInfoAboutIt} 
	to get information about each of the tokens in that line. This is done on a 
	line-by-line basis so that only lines that change need to be re-colored, as 
	it is a relatively expensive operation. For this project the 
	\textsf{IScanner} interface was implemented using a lexer class, how that 
	class was generated is described further in 
	Section~\ref{sec:syntax_checking}. The tokens of the language were divided 
	up into eight distinct color classes and colored as follows:
	
	\begin{itemize}
		\item Process constants - Greenblue
		\item Output actions on channels	- Dark gray
		\item Method calls - Magenta
		\item Comments - Dark green
		\item Strings and class names - Maroon
		\item Keywords - Blue
		\item Numbers - Red
		\item All other tokens - Black
	\end{itemize}

	An example of the syntax highlighting can be seen in 
	Figure~\ref{fig:syntaxcheck}.
	
	\subsection{Real time syntax checking}\label{sec:syntax_checking}
	A very useful feature of Visual Studio is its ability to show the user 
	errors in their code in real time, as they are typing. These errors (or 
	warnings) are shown both as text error messages in an error message window, 
	as well as red curvy lines under the places in code where the syntax errors 
	occurs. This makes it extremely easy to look at a page of code and determine 
	whether it is syntactically correct. Figure~\ref{fig:syntaxcheck} shows an 
	example of this feature in action.
	
	\begin{figure}[h!]
		\centering
		\includegraphics[scale=0.6]{syntaxcheck.jpg}
		\caption{Syntax checking for CCS in Visual Studio}
		\label{fig:syntaxcheck}
	\end{figure}

	The component in Visual Studio that is responsible for syntax highlighting 
	and syntax checking is named Babel \cite{babel}. As part of the Managed 
	Package Framework there is a collection of classes to wrap this component, 
	these are called the Managed Babel System. With the Managed Babel System 
	come two programs called MPLex.exe and MPPG.exe, these acronyms stand for 
	\textit{Managed Package Lex} and \textit{Managed Package Parser Generator}. 
	These are .NET implementations of the well known parser generator tools Lex 
	and YACC and derive directly from the \textit{Garden Point Parser Generator} 
	\cite{gppg} developed at the Queensland University of Technology. These 
	tools take as input a \textsf{lexer.lex} file and \textsf{parser.y} file. 
	The lexer file defines the tokens of the language with regular expressions, 
	and the parser file describes the syntax of the language in extended 
	Backus-Naur form, or EBNF. From these input files MPLex.exe and MPPG.exe 
	generate C\# code for a lexer and a parser for the language. A more detailed 
	explanation of Lex and YACC-like tools is outside the scope of this paper 
	but a useful book on the subject is \cite{lexyacc}. 
	
	Once the generated lexer and parser have been built, Visual Studio is 
	responsible for calling them repeatedly in the background while the user is 
	typing code. Visual Studio calls a method named 
	\textsf{ParseSource(ParseRequest req)} which is a method of the 
	\textsf{CCSLanguage} class. That class inherits from 
	\textsf{BabelLanguageService} and overrides its \textsf{ParseSource} 
	method. Inside this method the parser and lexer are instantiated and parse 
	the source code which is a part of the \textsf{ParseRequest} instance passed 
	to the method. The parser then logs every error it encounters, with file 
	name and line numbers, and Visual Studio is responsible for displaying these 
	errors to the user.

	As we saw in Chapter~\ref{ch:ccs_implementation} the parser used by the CCS 
	compiler itself was written using the Coco/R parser generator. It would have 
	been preferable to re-use that parser directly instead of defining a new 
	parser for the same input language. While it would have been possible, the 
	fact is that MPLex.exe and MPPG.exe are optimized for generating parsers 
	that work well with the Managed Babel System, and the Managed Babel 
	infrastructure expects parsers and lexers that conform to a certain 
	interface. For that reason MPLex and MPPG were used to create a new parser 
	and lexer instead of re-using the existing ones. The drawback to this is of 
	course that two implementations of the same language need to be maintained 
	and kept in sync. However, the input language for both these parser 
	generators is based on EBNF syntax and so it is fairly trivial to port from 
	one to the other. In hindsight the best approach would have been to use MPPG 
	and MPLex for the CCS compiler as well as the language service.
	
	\subsection{IntelliSense}
	IntelliSense (or automatic word completion) is one of the most useful 
	features of Visual Studio. When the programmer presses the keyboard 
	combination CTRL+SPACE at some point in the source code, the environment 
	shows a list of items that the programmer might wish to insert at that 
	point, and a description of them. This includes (in CCS's case) channel 
	names, process names, imported .NET method names and CCS keywords. If the 
	programmer invokes IntelliSense once the caret is positioned after a half 
	completed word, such as \textit{cof} and there is only one candidate that 
	starts with \textit{cof}, namely the channel \textit{coffee}, then the word 
	is completed automatically and the list of possibilities is not even shown.
	
	This behaviour is implemented through two main classes, 
	\textsf{AuthoringScope} and \textsf{Resolver}. \textsf{AuthoringScope} is 
	the class that is returned from the \textsf{ParseSource} method we learned 
	about in Section~\ref{sec:syntax_checking}, part of its interface is the 
	method \textsf{GetDeclarations}. This method is called when the user has 
	pressed CTRL+SPACE and it in turns calls a method named 
	\textsf{FindCompletions} on the \textsf{Resolver} class, which returns a 
	list of all the items to display to the user, along with descriptions and 
	icons. 
	
	To create the list of items the resolver uses the generated scanner class we 
	discussed previously and scans all tokens in the source file to find 
	process and channel names. For every channel name it adds an item for the 
	output action and input action on that channel, e.g. both \texttt{coffee} 
	and \texttt{\_coffee\_}. The resolver also adds all the language keywords 
	(if,then,use etc.) to the list so they can be completed automatically, and 
	to provide a reference for the user about what is possible to do. Finally 
	the resolver tries to find all methods that the application could call, and 
	add them along with descriptions that include parameter names and types. To 
	do this the resolver tries to look up all classes that are imported in the 
	source code using the \texttt{use} keyword, as well as look for methods in 
	the class \textsf{PLR.Runtime.BuiltIns} which contains methods that can be 
	called without a \texttt{use} statement. The method names and parameters are 
	found through reflection. Since the PLR only supports static methods and 
	only strings and integers as parameters, the resolver only considers methods 
	that fulfill that criteria. Once this has been done the list of names, 
	keywords and methods is returned so that Visual Studio can display them to 
	the user. Figure~\ref{fig:intellisense} shows an example of this feature in 
	action. Note how different types of items have different icons to 
	distinguish them from each other.

	\begin{figure}[h!]
		\centering
		\includegraphics[scale=0.5]{intellisense.png}
		\caption{IntelliSense for CCS in Visual Studio}
		\label{fig:intellisense}
	\end{figure}
	
	\subsection{Match braces}
	
	This feature was fairly trivial to implement as most of it is provided by 
	the Managed Package Framework. To get it working the generated parser needs 
	to store information about every matching parentheses pair while it is 
	parsing the source. This is simply done by calling a \textsf{Match} method 
	in the parser definition. Figure~\ref{fig:matchbrace} shows how the brace 
	matching for expressions is achieved in the parser definition.

\begin{figure}
\begin{codeblock}	
UnaryMinusTerm
    : '-' UnaryMinusTerm
    | NUMBER
    | '(' Expr ')'  { Match(@1, @3); }
    | LCASEIDENT
    | KWTRUE
    | MethodCall
    | KWFALSE
    ;
\end{codeblock}
\caption{Matching braces for expressions}
\label{fig:matchbrace}
\end{figure}

	The macros @1 and @3 are converted when the parser is generated and mean 
	that the first and third token on the lines match. The only additional thing 
	needed to get the feature working was to check in the \textsf{ParseSource} 
	method if the reason for the parsing was to highlight braces, and if so then 
	use the braces found by the parser earlier and call a \textsf{MatchPair} 
	method on the \textsf{ParseRequest} object that was passed to the method. 
	Figure~\ref{fig:matchbraces} shows how this feature is useful when working 
	with large expressions.
	
	\begin{figure}[h!]
		\centering
		\includegraphics[scale=0.75]{matchbraces.png}
		\caption{Brace matching for CCS in Visual Studio}
		\label{fig:matchbraces}
	\end{figure}
		
	
	\subsection{Build support}\label{msbuild}
	
	Visual Studio has menu items and keyboard shortcuts for a \textit{Build} 
	command. This command is used to build (compile) the source files of the 
	current project into an executable file. The underlying system that does the 
	actual building is named MSBuild, and it is a command line build tool, 
	similar to \textit{make} which is commonly used on Unix and Linux platforms, 
	and \textit{NAnt}, an popular open source build tool for the .NET framework. 
	The input files for MSBuild are XML files that define what the tool should 
	do. The three most important elements in these files are \textit{targets}, 
	\textit{tasks} and \textit{properties}. 
	
	A \textit{target} is a particular action to take, for instance there can be 
	a \textit{build} target which builds an entire project and a \textit{clean} 
	target which deletes all intermediate files. Targets can depend on each 
	other, for example a \textit{rebuild} target can depend on the 
	\textit{clean} and \textit{build} targets, so that whenever \textit{rebuild} 
	is executed the tasks it depends on are automatically executed first. 
		
	\textit{Tasks} are the operations performed by the targets. Each task is 
	usually a single distinct action, for example one task might be to call a 
	compiler, another task might be to copy files. Tasks can take parameters, 
	for instance the names of the files to compile. Custom tasks can be written 
	in .NET to achieve any operation and integrate it into the build process.
	
	\textit{Properties} are essentially variables to use in the build process, 
	and are often passed as parameters to the tasks. A property might for 
	instance be named \textit{Debug} and have either the value \texttt{true} or 
	\texttt{false}. It could then be passed to a task as a parameter.
	
	Each language in Visual Studio has its own \textit{.targets} files which 
	defines all the targets and tasks specific to that language. Additionally 
	there is a common targets file, \textit{Microsoft.Common.targets} which all 
	languages use. These \textit{.targets} files are then referenced in the 
	project files for the languages. To get CCS working with the build system it 
	was necessary to create one custom task, to call the compiler. The task is 
	defined in a class named \textsf{CompileTask} which inherits from MSBuild's 
	\textsf{ToolTask} class. It has an \textsf{Execute} method which calls the 
	CCS compiler and logs all errors that the compiler emits, and four 
	properties that can be set, \textsf{Debug}, \textsf{InputFile}, 
	\textsf{OutputFile} and \textsf{References}. To make that task available in 
	Visual Studio the template CCS project file references a file called 
	\textsf{CCS.targets}. That file in its entirety is shown in 
	Figure~\ref{fig:ccstargets}.
	
	\begin{figure}
	\begin{xml}
<?xml version="1.0" encoding="utf-8" ?>
<Project DefaultTargets="Build" 
  xmlns="http://schemas.microsoft.com/developer/msbuild/2003">
    <UsingTask TaskName="CCS.BuildTasks.CompileTask" 
      AssemblyFile="$(CCS_PATH)CCS.BuildTasks.dll"/>
    <Target Name="CoreCompile">
        <CompileTask 
            Debug="$(DebugSymbols)" 
            OutputFile="@(IntermediateAssembly)" 
            InputFile="@(Compile)" 
            References="@(ReferencePath)"/>
    </Target>
    <Target Name="CreateManifestResourceNames"></Target>
    <Target Name="Build"></Target>
    <Target Name="Compile"></Target>
    <Import Project="$(MSBuildBinPath)\Microsoft.Common.targets" />
</Project>
\end{xml}
\caption{The CCS.targets file}
\label{fig:ccstargets},
\end{figure}
	
	All that is necessary to do in the \textit{CCS.targets} file is to create a 
	target named \textit{CoreCompile}, and in that target call the custom 
	compile task that was created for the CCS compiler. Empty implementations of 
	the targets \textit{CreateManifestResourceNames}, \textit{Build} and 
	\textit{Compile} are also provided, since otherwise these targets try to 
	perform actions that are not necessary for building CCS applications. The 
	\textit{Debug} parameter can be set within Visual Studio and the names of 
	the input and output files are determined by the name given to the project 
	when it is created. References to other .NET assemblies can be added in 
	Visual Studio and they will be passed to the compile task through the 
	\textit{References} parameter. Having the CCS project file reference the 
	CCS.targets file, and that file call the custom compile task is enough to 
	get full build support from within Visual Studio.
	
	\subsection{Debugger support}
	
	Once the build support described in Section~\ref{msbuild} is in place, 
	getting debugger support within Visual Studio is trivial. All that is needed 
	is to select the \textit{Debug} configuration inside Visual Studio, this 
	passes the paramater \textit{Debug=true} to the compile task and on to the 
	compiler. We already saw how the PLR supports emitting debugging symbols in 
	Section~\ref{debug_support}. Pressing the F5 key, or the play button, in 
	Visual Studio will then build the project, launch the executable if the 
	build is successful and attach the debugger. This works due to the fact that 
	the \textit{Microsoft.Common.targets} file defines a \textit{Run} target 
	that calls the \textit{CoreCompile} target (which \textit{CCS.targets} 
	defines) and then launches the debugger, this action is the same for all 
	languages, although the compile step itself may be different.
	
	The only additional thing done to make the debugging experience more user 
	friendly was to implement a method named \textsf{ValidateBreakpointLocation} 
	in the \textsf{CCSLanguage} class. When the user tries to set a breakpoint 
	on a particular line, this method is called and is responsible for looking 
	at the line and determining whether a breakpoint can be set there, and if 
	so, which part of the line should be highlighted. For CCS the valid 
	locations are actions, method calls, process invocations and expressions in 
	\texttt{if} statements. Using the generated \textsf{Scanner} class once 
	again, it is possible to find out which tokens are on a particular line and 
	return their locations if they are valid points for a breakpoint. 
	Figure~\ref{fig:breakpoint} shows how a breakpoint is highlighted. (Note 
	that this is only relevant for the highlighting done at compile time, at run 
	time each line is highlighted according to the sequence points in the actual 
	compiled file.)

	\begin{figure}[h!]
		\centering
		\includegraphics[scale=0.5]{breakpoint.png}
		\caption{Setting breakpoints in Visual Studio}
		\label{fig:breakpoint}
	\end{figure}
	
	\section{Summary}
	
	There is no doubt that the tool support for programming languages is a 
	factor in whether those languages become popular, how productive programmers 
	are when working with them and how enjoyable they are to work with. 
	Programming languages used in industry have for a long time now had great 
	tool support while academic languages often suffer from bad or incomplete 
	tools. As this chapter demonstrates, this need not be the case. Once a 
	programming language (or any tool that takes some text files as input) has 
	been developed, taking the extra time to develop a language service for it 
	can be very beneficial to the end users of that language, and is relatively 
	easy to do. One approach might even be to develop the language service 
	first, thus enabling the author to benefit from it himself while writing the 
	compiler. Visual Studio might not be the perfect environment for all 
	languages, but in any case it is a good idea for the language author to 
	research what environments are out there and evaluate if one of them could 
	be used to offer the end user of the language a better user experience.