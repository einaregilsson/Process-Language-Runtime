\chapter{Integrated Development Environment}

	One of the biggest differences between academic programming languages and 
	industrial programming languages is the level of tool support. Academic 
	languages traditionally are edited in text editors and compiled with command 
	line tools, while industrial languages usually have integrated development 
	environments (IDE's). These environments offer a wide range of features such 
	as syntax highlighting, instant visual warnings about syntax errors, 
	background compilation, automatic listing of available methods and variables 
	(IntelliSense), built in debuggers and refactoring. One of the goals of this 
	project was to explore how well the CCS language could be integrated into 
	one of these environments and how it could benefit from the features they 
	have to offer. This chapter presents the results of this exploration. 
	
\section{Choice of Integrated Development Environment}
	
	When choosing which IDE would be most suitable for CCS two environments 
	stood out. First there is Microsoft's Visual Studio which is the most 
	popular environment for .NET development. Visual Studio versions are 
	released alongside new version of the .NET framework itself and each new 
	version takes full advantage of and supports all the new features in the 
	.NET framework. The latest version as of this writing is Visual Studio 2008 
	which supports the .NET framework 3.5. Visual Studio has an extensibility 
	API based on COM technology, an older technology for interaction between 
	programs written in different programming languages. Languages are 
	integrated by creating \textit{language services}, the Visual Studio program 
	itself is simply a host for these services. The languages that come with 
	Visual Studio, C\# and Visual Basic.NET have their own language services 
	that do not have any special access to Visual Studio, this implies that a 
	language service for a new language can be made to offer all the same 
	features as those supported by the built in languages.
	
	The second candidate for a CCS development environment was SharpDevelop, an 
	open source IDE written entirely in C\#. SharpDevelop is very similar to 
	Visual Studio and offers many of the same features. Its extensibility API is 
	entirely in .NET and is in many ways cleaner and clearer than Visual 
	Studio's API. It also has the benefit of being open source software, so it 
	is easy to get a clearer picture of its architecture, and view the source 
	for other language services. The architecture of SharpDevelop itself has 
	even been the subject of a book, \cite{sharpdevelop}. The drawbacks
	are that its debugger is inferior to Visual Studio's, the application itself
	is slower, and it is not as well known as Visual Studio.
	
	After researching both environments, Visual Studio 2008 was chosen as the one
	to implement CCS's language service in. This was based primarily on the fact
	that Visual Studio is the IDE of choice for most .NET developers, it is fast 
	and offers great debugging support. While the standard versions of Visual 
	Studio are not free, it is possible to download only the Visual Studio shell 
	and distribute it for free, the shell is the Visual Studio program itself 
	without any language services. This makes it possible to offer the CCS 
	development environment free of charge to anyone who wishes to use it.

\section{Building a language service}
	
	\subsection{Language Service basics}	
	The COM extensibility API for Visual Studio is fairly complicated and 
	unfriendly to use. With the Visual Studio SDK (Software Development Kit) 
	Microsoft has included a system called the Managed Babel System 
	\cite{babel}. This is a wrapper around the Visual Studio COM API 
	specifically meant to simplify writing language services in managed 
	languages such as C\#. It implements most of the tedious boilerplate code 
	necessary to create the language service. The first step in creating a 
	language service is registering it with Visual Studio. For this the 
	programmer has to supply a language name, a file extension for the language 
	and a globally unique identifier (GUID). After the language service has been 
	registered with Visual Studio it will call into the language service 
	whenever a file with the registered extensions is opened in the Visual 
	Studio text editor. The additional features of the language service are 
	described in the following sections.
	
	\subsection{Syntax highlighting and visual syntax warnings}
	Syntax highlighting is when different tokens in a language are given 
	different color to help differentiate them when looking at code. This
	is very helpful when looking at code to quickly sense the structure and
	identify problems, and has been a standard feature of development 
	environments as well as most advanced text editors for many years. In many
	common text editors this is simply implemented as lists of tokens and 
	colors for them. Visual Studio however requires a full lexer and parser for
	a language in order to provide syntax highlighting. This also enables it to
	give error messages for syntax errors in the code before compiling. These
	are shown both as text error messages in an error message window, as well
	as red curvy lines under the places in code where the syntax errors occurs.
	This makes it extremely easy to look at a page of code and determine whether
	it is syntactically correct.
	
	The Managed Babel System ships with two programs called MPLex.exe and 
	MPPG.exe. These are .NET implementations of the well known parser generator 
	tools Lex and YACC. These tools take as input a \textsf{lexer.lex} file and
	\textsf{parser.y} file. The lexer file defines the tokens of the language
	with regular expressions, and the parser file describes the syntax of the
	language in extended Backus-Naur form, or EBNF. From these input files 
	MPLex.exe and MPPG.exe generate C\# code for a lexer and a parser for the 
	language. A more detailed explanation of Lex and YACC is outside the scope 
	of this paper but a useful book on the subject is \cite{lexyacc}. 
	The generated lexer and parser are called repeatedly in the background while
	code is being edited and any syntax errors are highlighted and displayed to
	the user. Colors for different tokens are defined in a special 
	\textsf{Configuration} class which is a part of the language service.

	When implementing the CCS compiler a EBNF definition was defined for the
	parser generator Coco/R. Porting that definition to Lex/YACC was not 
	complicated to do, as the underlying principles are the same. The tokens of 
	CCS were divided into six distinct classes, and were colored as follows
	
	\begin{itemize}
		\item Process constants - Blue
		\item Synchronization on channels	- Red
		\item Method calls - Bluegreen
		\item Comments - Green
		\item Strings - Maroon
		\item Operators - Black
	\end{itemize}
	
	\subsection{IntelliSense}
	\TODO{Implement and describe intellisense}

	\subsection{CCS Project and item templates}
	To enable a user to create new CCS projects inside Visual Studio it was 
	necessary to create templates for these files. This was done by creating 
	three files:
	
	\begin{itemize}
		\item \textbf{CCSProject.vstemplate} is a xml file that describes the 
		project type. It contains the name of the project type, unique identifier 
		for it and various parameters such as whether a new project should be 
		created in a new folder. This is the file Visual Studio uses to create an 
		item for CCS in the \textit{New project} dialog.
		
		\item \textbf{project.ccsproj} is a template for new CCS projects. It 
		contains information about debug and release configurations, the name of 
		included CCS sourc files, which assemblies should be referenced by default 
		as well as references to the build tasks to execute to compile CCS 
		projects. Build tasks are further explained in Section~\ref{msbuild}.
		
		\item \textbf{Program.ccs} is a template for CCS source files. It contains
		nothing but a small comment explaining that CCS code should go here. 
	
	\end{itemize}
	
	These three files are put in a zip archive which is stored in a subfolder of
	the Visual Studio main folder. Once this has been done the user can select 
	\textit{CCS Project} as a project type when creating new projects in Visual 
	Studio. For more complex languages an item template could also be created to
	allow users to add additional source files to the program. Since the current
	CCS compiler only supports compiling a single source file this feature was 
	not added, when a user initially creates a new project he gets one source 
	file named \textsf{program.ccs} added automatically, all source code must go
	inside that file.

	\subsection{Build support}\label{msbuild}
	
	
	\section{Summary}
	