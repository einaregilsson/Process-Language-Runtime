\chapter{Final Considerations}

\section{Related work}\label{sec:related_work}
	
	There are quite a few other projects that have implemented process algebras, 
	or languages inspired by process algebras, however most of these projects 
	take a different approach than that taken by the PLR. Below is a short 
	summary of some of the notable ones, especially those that focus on CCS or 
	KLAIM.
	
	\textit{KLAVA} \cite{klava} is an implementation of KLAIM. It is a Java 
	library which represents the KLAIM constructs as Java classes. KLAIM 
	applications can then be written in Java using the KLAVA library. The main 
	difference between KLAVA and the PLR implementation of KLAIM (hereafter 
	referred to as PLR KLAIM) is that in KLAVA it is not possible to write the 
	applications using the actual syntax of KLAIM. KLAVA is a much more feature 
	rich implementation of KLAIM than PLR KLAIM, it supports nodes running on 
	different machines, and additional constructs such as non blocking input 
	operations.
	
	\textit{X-Klaim} \cite{xklaim} (which stands for eXtended KLAIM) is another 
	implementation of KLAIM from the authors of KLAVA. It is at a higher level 
	of abstraction than KLAVA and has its own syntax, which is a superset of
	the original KLAIM syntax. X-Klaim code uses the KLAVA library as its 
	runtime library, the X-Klaim compiler compiles X-Klaim code down to Java 
	code that uses the KLAVA library. X-Klaim is similar to PLR KLAIM in that 
	both have a KLAIM syntax and both use a runtime library, the difference is 
	that PLR KLAIM directly emits bytecodes, whereas X-Klaim emits Java source 
	code, which means that X-Klaim code cannot be debugged using the original
	X-Klaim source files. X-Klaim is feature rich and supports many constructs
	not in the original KLAIM.
	
	\textit{AspectK} is an aspect oriented version of KLAIM. Originally 
	introduced in \cite{aspectk}, a full virtual machine for the language was 
	subsequently developed in \cite{giordano}. The KLAIM subset used in AspectK 
	is the same as that used in PLR KLAIM, AspectK then adds aspects on top of 
	that. The difference (aside from the aspect orientation) is that AspectK has 
	its own virtual machine, with bytecodes for common process algebra tasks 
	whereas PLR KLAIM uses an existing virtual machine. An advantage of having a 
	process algebra focused virtual machine is that generated code can be 
	smaller, since each bytecode instruction can perform more work. The PLR does 
	get a similar reduction in code size by generating bytecodes that call 
	methods defined in the PLR runtime library.
	
	\textit{JACK} \cite{jack} is a process algebra implementation written in 
	Java. It is similar to the PLR in that it aims to be a framework that can be
	used for implementing different types of process algebra, although its main
	focus is Communicating Sequential Processes (CSP). The difference is that 
	JACK, like KLAVA, represents algebra constructs as Java classes, and the  
	systems are written using Java code instead of the native syntax of the 
	process algebra being implemented. That is to say, it is a framework, but 
	not a compiler.
	
	CCS has at least two implementations, in \cite{build_ccs_interpreter} a 
	method is presented for how to build a sound CCS interpreter by following 
	the semantics of the language, and \cite{ccs_interpreter} shows how the 
	functional programming language Haskell can be used to build a CCS 
	interpreter with minimal amount of code. Both of these differ from the PLR 
	in that they are interpreters rather than compilers.
	
	None of the above related work aims to do exactly what the PLR attempts, 
	which is to build compilers for process algebras that operate on the 		
	algebra's standard syntax and integrate tightly with an existing virtual 
	machine. The PLR is also the only one of these that explores how existing 
	infrastructure can be used to add features to process algebras, such as 
	allowing CCS to call .NET methods written in another .NET language. A 
	further look at that topic might prove interesting, specifically how it 
	affects the original semantics of the algebra being implemented, what side 
	effects it might produce and what sort of interesting things could be
	modeled in this way.

\section{Further work}

	The PLR could be improved and built upon in several ways. Here we will look 
	at some of ways in which that could be achieved.
	
	\subsection{Additional process algebra constructs}
	
	Perhaps the most obvious improvement to the PLR would be to add support for 
	some of the constructs that are common in process algebras but are not 
	included in the PLR currently. This would make it even simpler to use the 
	PLR as the basis for implementing other process algebras. To get a sense of 
	which constructs would be most beneficial to add, we look shortly at three 
	of the most prominent process algebras, CSP, $\pi$-calculus and LOTOS, and 
	what would be needed for them to run on the PLR.
	
	\textit{CSP} is very similar to CCS. It does contain a few extra constructs
	
	\textit{$\pi$-calculus} is ...
	
	\textit{LOTOS} is ...
	
	\subsection{More datatypes and expressions}
	The initial version of the PLR supports two types of variables, integers and 
	strings. For expressions it supports constants for integers, booleans and 
	strings, as well as simple arithmetic and relational operators for integers 
	and logical operators for booleans. One way to extend the PLR would be to 
	add more support for using other data types from .NET. An example would be 
	to allow passing of .NET objects through channels and calling instance 
	methods on those objects in the receiving process. This would require some 
	additional syntax nodes for the PLR tree, an expression node for 
	constructing a new object and a node for a method call on an object (as 
	opposed to a static method call which is already supported). Other useful 
	features to add might include support for floating point numbers, string 
	formatting and basic string expressions using the + operator.
	
	\subsection{Additional analysis}
	
	Another potential improvement would be to add additional analyses before 
	compilation. This is where the benefit of having a shared syntax tree for 
	multiple algebras becomes apparent, as many of the analyses could be re-used 
	for multiple process algebras (although probably not all of them). This 
	could include common compiler optimization techniques such as constant 
	propagation and Very Busy Expressions analysis, or analyses more directly 
	related to process algebra, such as finding channels that are never used
	and identifying processes that will always block.
	
	\subsection{Bi-similarity of processes}
	One of the interesting things that could be added, for CCS and maybe others, 
	would be an analysis to compare two processes and see if they are 
	\textit{behaviorally equivalent}, also known as \textit{bi-similar}. 
	B-similarity is a congruence, if processes $P$ and $Q$ are bi-similar then 
	it means that if $P$ is a component in a system then it can be replaced with 
	$Q$ and the system will continue to work in the same way, since $P$ and $Q$ 
	exhibit the same behavior. This can for instance be used to write a 
	specification as a simple process expression and then write an 
	implementation for that specification. If the specification and 
	implementation are bi-similar then the implementation is a correct 
	implementation of the specification. An example could be the specification
  \begin{align*}
			\mathrm{CoffeeMachine} \defeq & coin  \ccsdot \out{coffee} \ccsdot CoffeeMachine \\
	\end{align*} and the implementation
  \begin{align*}
			\mathrm{CoffeeMachineImpl} \defeq & ( \mathrm{CoinReceiver} \mid \mathrm{CoffeePourer} ) \backslash \{pour\} \\
			\mathrm{CoinReceiver} \defeq & coin \ccsdot \out{pour} \ccsdot \mathrm{CoinReceiver} \\
			\mathrm{CoffeePourer} \defeq & pour \ccsdot \out{coffee} \ccsdot \mathrm{CoffeePourer}
	\end{align*}	

	Here we see that according to the specification of \textsf{CoffeeMachine} 
	the observable events are an endless stream of $coin$ and $coffee$. This 
	however tells us nothing about how this machine is implemented. The second 
	process \textsf{CoffeeMachineImpl} is the implementation of this coffee 
	machine, it is composed of two components, a receiver for the coins and a 
	component that pours the coffee. They communicate between themselves on the 
	$pour$ channel. Since that channel is hidden (or restricted) it is not 
	observable from the outside, what is observable from the outside is again an 
	endless stream of $coin$ and $coffeee$. In this trivial case it is obvious 
	that \textsf{CoffeeMachineImpl} is a valid implementation of 
	\textsf{CoffeeMachine}.
	
	Bi-similarity can be analyzed by converting a CCS process expression into a 
	\textit{labelled transition system}, which is a state machine where the 
	transitions between states are the actions performed in the process. Weak 
	bi-simulation, or observational equivalence, is perhaps the most interesting 
	bi-simulation to verify. In general terms it states that if $P$ and $Q$ are 
	weakly bi-similar then they will behave exactly the same when observed from 
	the outside, they will offer the same synchronizations or events. However 
	before and after these public events they can perform any number of internal 
	actions or $\tau$ actions which do not have to match between the two 
	processes since they do not affect their behavior as seen from the outside. 
	
	The PLR syntax tree is a rich data structure and would be well suited for 
	this type of analysis. This might not be the type of feature that belongs in 
	a process algebra compiler, instead it might be incorporated into some kind 
	of analysis tool for CCS (or other process algebras) that could make use of 
	the syntax tree of the compiler and the parser and scanner of the CCS 
	compiler. 
	
	This section has only briefly touched on the possibilities of verifying 
	process behavior using bi-simulation, for a more comprehensive explanation 
	see \cite{reactive}.
	
\section{Conclusions}
