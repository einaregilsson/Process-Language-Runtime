\appendix

\chapter{Software}

	During the course of this project four distinct software packages were 
	developed, a CCS compiler, a CCS integration package for Visual Studio, a 
	KLAIM compiler and the Process Viewer application. Here we look at the 
	practical aspects of this software, where it can be downloaded, how it is 
	licensed and how it can be built, configured and used.

\section{Licensing and availability}

	All the software developed during this project can be downloaded from the 
	url \url{http://einaregilsson.com/plr}. The source code for the entire 
	project is available in a zip file. The binaries for each of the software 
	packages can be downloaded seperately. 
	
	The source code for the project is licensed under the General Public License 
	(GPL) v3.0. In brief, this allows anyone to download and modify the source or
	use it as a basis for something else, as long as the source code for that 
	modified version is also made available under a GPL compatible license. For 
	further information see \url{http://www.gnu.org/licenses/gpl.html}.


\section{Source code}
	
	The source code is organized into two solutions and ten projects. They are 
	as follows:
	
	\begin{itemize}
		\item \textbf{MSC} is the solution for the compilers and runtimes.
		\begin{itemize}
			\item \textbf{PLR} is the project for the Process Language Runtime.
			\item \textbf{CCS} is the project for the CCS compiler.
			\item \textbf{CCS.External} is a project which contains utility 
			functions written in C\# which can be called from CCS if the 
			CCS.External.dll is referenced during compilation.
			\item \textbf{KLAIM} is the project for the KLAIM compiler.
			\item \textbf{KLAIM.Runtime} is the project for the KLAIM runtime 
			library.
			\item \textbf{ProcessViewer} is the project for the Process Viewer
			visualization tool.
		\end{itemize}
		\item \textbf{CCS.Integration} is the solution for Visual Studio 
		Integration.
		\begin{itemize}
			\item \textbf{CCS.BuildTasks} contains MSBuild build tasks for the CCS 
			compiler.
			\item \textbf{CCS.LanguageService} is the main integration project, it 
			contains the Visual Studio language service for CCS.
			\item \textbf{CCS.Projects} containst the Visual Studio package that 
			allows CCS projects to be created in Visual Studio.
			\item \textbf{CCS.Deployment} is a project that builds an installer for 
			the entire integration package.
		\end{itemize}
	\end{itemize}
	
	The solutions and projects can be built using Visual Studio 2008, although 
	that is not required. The C\# compiler and MSBuild build tool are included 
	with the standard .NET framework distribution, they are enough to build the 
	projects. To build the entire MSC solution, first add the location where
	MSBuild is stored to the PATH environment variable, this can be done with 
	\begin{verbatim}SET PATH=%PATH%;c:\Windows\Microsoft.NET\Framework\v3.5\end{verbatim} The solution 
	can then be built with the command\begin{verbatim} 
	msbuild MSC.sln /p:Configuration=Debug /p:Platform="Any CPU"
	\end{verbatim}
	The .NET framework v3.5 is required to build both solutions. To build the
	CCS.Integration solution it is necessary to have the Visual Studio 2008 SDK 
	installed.
	
\section{CCS Compiler}

  The CCS compiler is an executable file named \texttt{ccs.exe}. It has one
  dependency which is the PLR itself, it is in a file named \texttt{PLR.dll}.
  The compiler is a command line tool and is invoked as 
  \begin{verbatim}ccs.exe [options] <filename>\end{verbatim} It accepts one 
  input file (\texttt{<filename>}) and can accept a number of optional command 
  line switches (\texttt{[options]}). By default the generated executable file 
  will have the same name as the input file, except ending with \texttt{.exe} 
  instead of \texttt{.ccs}. The compiled file will have a dependency on the 
  PLR for the runtime system. To get a guide to the available command line 
  options the compiler can be invoked as \texttt{ccs.exe /?} . The output of 
  that command is shown below:
	\begin{footnotesize}
	\begin{verbatim}
CCS Compiler
Copyright (C) 2009 Einar Egilsson

Usage: CCS [options] <filename>

Available options:

    /reference:<files>   The assemblies that this program requires. It is
    /r:<files>           not neccessary to specify the PLR assembly.
                         Other assemblies should be specified in a comma
                         seperated list, e.g. /reference:Foo.dll,Bar.dll.

    /optimize            If specified then the generated assembly will be
    /op                  optimized, dead code eliminated and expressions
                         pre-evaluated where possible. Do not combine this
                         with the /debug switch.

    /embedPLR            Embeds the PLR into the generated file, so it can
    /e                   be distributed as a stand-alone file.
  
    /debug               Emit debugging symbols in the generated file,
    /d                   this allows it to be debugged in Visual Studio, or
                         in the free graphical debugger that comes with the
                         .NET Framework SDK.

    /out:<filename>      Specify the name of the compiled executable. If
    /o:<filename>        this is not specified then the name of the input
                         file is used, with .ccs replaced by .exe.

    /print:<format>      Prints a version of the program source in the
    /p:<format>          specified format. Allowed formats are ccs, html
                         and latex. The generated file will have the same
                         name as the input file, except with the format
                         as extension.	
\end{verbatim}
\end{footnotesize}  
  
\section{CCS Visual Studio Integration Package}

	The integration package for CCS can be downloaded as a MSI installer. To 
	start using the integration package simply follow the instructions in the 
	installer program, it will install the necessary files and register the 
	language service with Visual Studio. 
	
	After installing the integration package the following features will be 
	added to Visual Studio:
	
	\begin{itemize}
		\item When creating a new project there is an option named \textit{CCS 
		Project}. Choosing this option creates a new project with the file ending 
		.ccsproj and includes references to the PLR and a CCS source code file 
		with some example code.
		
		\item The CCS project can be built using Visual Studio's \textit{Build} 
		command or by using the keyboard shortcut Ctrl+Shift+B.
		
		\item Any file that is edited in Visual Studio which has the file ending 
		.ccs will be handled by the language service, which will syntax highlight 
		it and warn about syntax errors.
		
		\item IntelliSense can be invoked in .ccs files by pressing Ctrl+Space.
		
		\item If the \textit{Debug} configuration is chosen then a CCS 
		system can be debugged by pressing the ``play'' button, or by pressing F5.

	\end{itemize}
	
	The integration package can be uninstalled in the standard Windows 
	\textit{Add/Remove programs} dialog.
\newpage
\section{KLAIM Compiler}

  The KLAIM compiler is an executable file named \texttt{kc.exe}. It has two 
  dependencies, the KLAIM runtime (\texttt{KlaimRuntime.dll}) and the PLR 
  itself (\texttt{PLR.dll}). The KLAIM compiler is very similar to the CCS
  compiler, it is a command line tool that is invoked as
  \begin{verbatim}kc.exe [options] <filename>\end{verbatim} It accepts one 
  input file (\texttt{<filename>}) and can accept a number of optional command 
  line switches (\texttt{[options]}). The generated executable file 
  will have the same name as the input file, except ending with \texttt{.exe} 
  instead of \texttt{.klaim}. The command \texttt{kc.exe /?} will print out
  the available options and then exit. The output of that command is shown 
  below:

	\begin{footnotesize}
	\begin{verbatim}
KLAIM Compiler
Copyright (C) 2009 Einar Egilsson

Usage: kc [options] <filename>

Available options:

    /optimize            If specified then the generated assembly will be
    /op                  optimized, dead code eliminated and expressions
                         pre-evaluated where possible. Do not combine this
                         with the /debug switch.
    
    /debug               Emit debugging symbols in the generated file,
    /d                   this allows it to be debugged in Visual Studio, or
                         in the free graphical debugger that comes with the
                         .NET Framework SDK.

    /embedKLAIM          Embeds the KLAIM runtime into the generated file, 
    /ek                  so it does not need to be distributed with the 
                         executable.

    /embedPLR            Embeds the PLR into the generated file, so it does
    /e                   not need to be distributed with the executable.

    /out:<filename>      Specify the name of the compiled executable. If 
    /o:<filename>        this is not specified then the name of the input
                         file is used, with .ccs replaced by .exe.

	\end{verbatim}
	\end{footnotesize}


\section{Process Viewer}

	The Process Viewer is an executable named \texttt{ProcessViewer.exe}. It 
	requires the PLR dll, \texttt{PLR.dll} as well as any assemblies that contain
	parsers for the languages being used. The filenames of the parser assemblies
	are specified in the configuration file, \texttt{ProcessViewer.exe.config}. 
	An example is 
	
	\texttt{<add key="ParserAssemblies" value="CCS.exe;kc.exe"/>}
	
	Here there are two assemblies specified, their names seperated by a 
	semicolon. 
	
	Figure~\ref{fig:process_viewer_tutorial} shows the main screen, with the 
	main controls labelled with numbers. Below each number corresponding to a 
	control is explained.

	\begin{figure}[h!]
		\centering
		\includegraphics[scale=0.4]{process_viewer_appendix.png}
		\caption{The Process Viewer application running}
		\label{fig:process_viewer_tutorial}
	\end{figure}
	
	\begin{enumerate}
		\item The \textit{open} button. Press this to open a new process language 
		executable file, and optionally its original source file.
		\item Starts execution of the loaded process language application.
		\item Starts execution of the loaded process language application in step 
		mode. This means that the user will select each action that is executed.
		\item Pauses execution. This will put a running application into step mode.
		\item Stops execution of running application.
		\item This is a list of all active processes at the current time.
		\item The trace is the list of actions that have been executed.
		\item This is a list of the actions that could potentially be executed 
		next.
		\item Pressing this button executes the currently selected action.
		\item Pressing this button executes a random action.
		\item Additional information about the selected action is displayed here.
		\item The current state of the active processes is shown here, along with
		their parent chains, that is the restricted processes that spawned them
		and are kept alive so that the restrictions still apply.
	\end{enumerate}