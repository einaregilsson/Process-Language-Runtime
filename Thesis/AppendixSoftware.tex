\appendix

\chapter{Software}

	During the course of this project three distinct software packages were 
	developed, a CCS compiler, a CCS integration package for Visual Studio and a 
	KLAIM compiler. Here we look at the practical aspects of this software, 
	where it can be downloaded, how it is licensed and how it can be configured 
	and used.

\section{Licensing and availability}

	All the software developed during this project can be downloaded from the 
	url http://einaregilsson.com/plr. The source code for the entire project is 
	available in a zip file. The binaries for each of the software packages can 
	be downloaded seperately. 
	
	The source code for the project is licensed under the General Public License 
	(GPL) v3.0. In brief, this allows anyone to download and modify the source or
	use it as a basis for something else, as long as the source code for that 
	modified version is also made available under a GPL compatible license. For 
	further information see http://www.gnu.org/licenses/gpl.html.

\section{CCS Compiler}

  The CCS compiler is an executable file named \texttt{ccs.exe}. It has one
  dependency which is the PLR itself, it is in a file named \texttt{PLR.dll}.
  The compiler is a command line tool and is invoked as 
  \begin{verbatim}ccs.exe [options] <filename>\end{verbatim} It accepts one 
  input file (\texttt{<filename>}) and can accept a number of optional command 
  line switches (\texttt{[options]}). By default the generated executable file 
  will have the same name as the input file, except ending with \texttt{.exe} 
  instead of \texttt{.ccs}. The compiled file will have a dependency on the 
  PLR for the runtime system. To get a guide to the available command line 
  options the compiler can be invoked as \texttt{ccs.exe /?} . The output of 
  that command is shown below:
	\begin{footnotesize}
	\begin{verbatim}
CCS Compiler
Copyright (C) 2009 Einar Egilsson

Usage: CCS [options] <filename>

Available options:

    /reference:<files>   The assemblies that this program requires. It is
    /r:<files>           not neccessary to specify the PLR assembly.
                         Other assemblies should be specified in a comma
                         seperated list, e.g. /reference:Foo.dll,Bar.dll.

    /optimize            If specified then the generated assembly will be
    /op                  optimized, dead code eliminated and expressions
                         pre-evaluated where possible. Do not combine this
                         with the /debug switch.

    /embedPLR            Embeds the PLR into the generated file, so it can
    /e                   be distributed as a stand-alone file.
  
    /debug               Emit debugging symbols in the generated file,
    /d                   this allows it to be debugged in Visual Studio, or
                         in the free graphical debugger that comes with the
                         .NET Framework SDK.

    /out:<filename>      Specify the name of the compiled executable. If
    /o:<filename>        this is not specified then the name of the input
                         file is used, with .ccs replaced by .exe.

    /print:<format>      Prints a version of the program source in the
    /p:<format>          specified format. Allowed formats are ccs, html
                         and latex. The generated file will have the same
                         name as the input file, except with the format
                         as extension.	
\end{verbatim}
\end{footnotesize}  
  
\section{CCS Visual Studio Integration Package}


\section{KLAIM Compiler}

  The KLAIM compiler is an executable file named \texttt{kc.exe}. It has two 
  dependencies, the KLAIM runtime (\texttt{KlaimRuntime.dll}) and the PLR 
  itself (\texttt{PLR.dll}).
  dependency which is the PLR itself, it is in a file named \texttt{PLR.dll}.
  The compiler is a command line tool and is invoked as 
  \begin{verbatim}ccs.exe [options] <filename>\end{verbatim} It accepts one 
  input file (\texttt{<filename>}) and can accept a number of optional command 
  line switches (\texttt{[options]}). By default the generated executable file 
  will have the same name as the input file, except ending with \texttt{.exe} 
  instead of \texttt{.ccs}. To get a guide to the available command line 
  options the compiler can be invoked as \texttt{ccs.exe /?} . The output of 
  that command is shown below:
