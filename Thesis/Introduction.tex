\chapter{Introduction}

	In the last decade or so, programming languages have increasingly started to 
	target virtual machines instead of specific physical machine architectures. 
	This approach has a number of benefits. Many virtual machines have 
	implementations on different machine architectures and operating systems, 
	which enables an application developer to write applications in a language 
	that targets the virtual machine and getting the benefit of his application 
	running on multiple architectures and operating systems for free. Another 
	benefit of virtual machines is that code written in different languages can 
	interoperate, allowing developers to write each part of their application in 
	the language best suited for the job. The two most prominent virtual 
	machines used today are the Java Virtual Machine (JVM) and the Common 
	Language Runtime (CLR). The JVM was originally created by Sun 
	Microtechnologies but several implementations are now available by many 
	vendors. The CLR was created by Microsoft for the Windows operating system, 
	but an open source version, Mono, which works on Unix and Linux operating 
	systems is also available. Both the JVM and CLR have a number of different 
	languages that target them, and can therefore be used to build 
	\textit{cross-platform} applications. 

	Process languages are a class of languages that are made up of distinct 
	processes communicating with each other. Some examples of process languages 
	are \textit{Calculus of Communicating Systems} (CCS), \textit{Communicating 
	Sequential Processes} (CSP), $\pi$ \textit{calculus} and \textit{Kernel 
	Language for Agents Interaction and Mobility} (KLAIM). The process languages 
	mentioned here have a number of traits in common, including action 
	prefixing, non-deterministic choice and parallel composition. This leads to 
	the assumption that maybe some of these traits can be abstracted away into a 
	common framework, so that each implementation of a process language does not 
	have to implement them separately.

	Implementing a process language, and having it target a popular virtual 
	machine such as the CLR is an interesting proposition. A number of issues 
	can be explored. How well does the instruction set of the virtual machine 
	fit the execution model of the process language? Is there benefit in having 
	access to the large standard library that comes with the CLR? Is it feasible 
	to write parts of a process language application in another language that 
	targets the virtual machine, for instance numerical functions? A number of 
	advanced tools and integrated development environments exist for CLR and JVM 
	languages, can process languages make use of them? 

\section{Thesis Objectives}

	This thesis presents the design and implementation of the \textit{Process 
	Language Runtime}, hereafter referred to as the PLR. The PLR consists of two 
	main components. First, it contains an extensible abstract syntax tree, 
	which models common idioms of process languages, and can compile itself to 
	.NET bytecode. Secondly, it contains a runtime library that is used by 
	process applications that have been compiled from the PLR syntax tree. 

	Integration with the .NET platform is also explored, e.g. how to allow 
	process language applications to call code developed in other .NET languages 
	and whether it is possible to use existing .NET development tools to aid in 
	writing process language applications. 
	
	A tool to interact with running process applications is also developed, this 
	tool has a graphical user interface and is meant to give more insight into 
	what is happening when a process language application is running and allow 
	users to select which paths are taken during execution.

	Finally the thesis presents two case studies of process language 
	implementations that were made using the PLR. The languages implemented were 
	CCS (Calculus of Communicating Systems) and a subset of KLAIM (Kernel 
	Language for Agents Interaction and Mobility).

	\newpage
	The work carried out consists of the following parts:

	\begin{itemize}
  	\item Design and implementation of the PLR abstract syntax tree and 
  				compiler, including static analysis and compiler optimizations.
  	\item Design and implementation of the runtime library.
  	\item Implementation of the process language CCS.
  	\item Implementation of the process language KLAIM.
  	\item Design and implementation of a tool to interact with running process 
  	language applications.
  	\item Integration of the CCS language into Visual Studio, a 
  				state-of-the-art integrated development environment for .NET 
  				development.
	\end{itemize}


\section{Thesis Outline}

	The thesis consists of nine chapters, of which this introduction is the 
	first. The rest are as follows:

	Chapter 2 gives some background on process languages in general and their 
	common properties, and some technical background on the .NET platform. The 
	concepts presented there are useful for understanding the architecture of 
	the PLR.\\
	Chapter 3 describes the Process Language Runtime, both the syntax tree and 
	runtime library and shows how they are designed and implemented. \\
	Chapter 4 describes the static analysis and optimizations that are performed
	before compilation.\\
	Chapter 5 is a case study of an implementation of the CCS process 
	language.\\
	Chapter 6 is another case study, this time of an implementation of the 
	KLAIM process language.\\
	Chapter 7 is about the graphical tool Process Viewer and how it can be used
	to interact with process language applications.\\
	Chapter 8 gives an overview of how the CCS language was integrated into 
	Visual Studio 2008. \\
	Finally, Chapter 9 contains concluding remarks and an exploration of related 
	work. Ideas for further development of the PLR are also discussed, including 
	how some other common process languages could potentially be implemented 
	using the PLR.\\
 
	In addition, there are two appendices:
	
	Appendix A is about the practical aspects of the software developed during 
	the course of the project, where it can be downloaded, hot it is licensed 
	and how it can be configured and run.\\
	Appendix B shows a small CCS system and all the generated bytecode for it.