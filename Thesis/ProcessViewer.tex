\chapter{Interactive Process Viewer}\label{ch:process_viewer}

	In this chapter we look briefly at \textit{Process Viewer}, a tool to interact
	with process language applications during execution. Its architecture is 
	explained, as well as how it interacts with the compiled process language 
	executables. The challenges in making it general enough for any process
	language are also discussed.

\section{Overview}

	\textit{Process Viewer} (hereafter referred to simply as \textit{the viewer})
	is an application to allow users to closely monitor
	and affect how compiled process language applications are executed. It enables
	the user to see what processes are currently active, what actions have been
	executed and what actions are ready for execution. It can be run 
	interactively, which allows the user to select the next action for execution.
	If the source code for the application being run is available, then it is also
	possible to see the state of the system in source code form at every stage 
	(e.g. if the original system contained the process $a \ccsdot b \ccsdot P$ and
	action $a$ has been executed then the system now contains the process 
	$b \ccsdot P$). The user interface is simple and consists of only one window.
	Figure~\ref{fig:process_viewer} shows a screenshot of the program during 
	execution of a process language application. The list of active processes is 
	in the upper left of the screen, the trace is on the lower left. The current 
	state of the system is shown in the large text box and under it the next 
	possible actions are shown, as well as some controls to select the action.
	
	\begin{figure}[h!]
		\centering
		\includegraphics[scale=0.4]{process_viewer.png}
		\caption{The Process Viewer application}
		\label{fig:process_viewer}
	\end{figure}
	

\section{Architecture}
  
  \subsection{Class structure}
  The application is written in C\# and is made up of just two main classes. 
  \code{ProcessViewer} is the class for the window itself and contains all
  the actions that have to do with the graphical user interface. The other
  class is \code{ProcessStateVisualization} which is responsible for keeping
  track of how the system looks in source code form at every stage of the
  execution. Since most of what is done in the program has to to with updating
  controls in the window it was not deemed necessary to modularize the code
  further. The only real algorithm in the program is how the process state is
  extracted from the running process language application, that code was 
  clearly not directly tied to the graphical user interface and therefore it
  was put in its own class, the \code{ProcessStateVisualization} class.
  
  \subsection{Interaction with the PLR}
  
  The process language application that is being executed is run in the same 
  operating system process as the process viewer itself. This is done by 
  loading the process language application assembly and simply calling its 
  \code{Main} method on a seperate thread. Doing it this way allows the 
  viewer to interact directly with the process language application and its 
  classes. The interaction with happens mainly through the PLR's 
  \code{Scheduler} class. The scheduler has three useful events that the 
  process viewer subscribes to, these are: \code{ProcessRegistered}, 
  \code{ProcessKilled} and \code{TraceItemAdded}. These notify subscribers 
  when new processes are added to the system, when processes are removed from 
  the systems, and when new items are added to the trace, that is when actions 
  have been executed. The process viewer uses these events to update the 
  controls that show active processes and the trace. 
  
  To allow the user to select which action to execute the viewer makes use of
  a simple abstraction that the \code{Scheduler} class provides. The scheduler
  does not have a special method to select an action from all the candidate 
  actions, instead it has a delegate (function pointer) to a method that takes
  in a list of \code{CandidateAction} classes and return the one that should be
  executed. This function pointer by default points to a simple method that 
  randomly chooses an action to execute, but it can be set to any other method
  that has the correct method signature. The viewer sets this function pointer
  to its own method, that method shows each candidate action in the window and
  returns the one the user chooses. The \code{CandidateAction} class contains
  information about an action and the process or processes that perform it so
  the viewer has enough data to display to the user.

  \subsection{Process visualization}\label{sec:process_visualization}
	
	Showing the state of the process system after each step requires the original
	source file. This is optional, if the source file is not available then the
	viewer can still be used, but the system state in source code form will not
	be shown. The names of the active processes, the trace and the candidate 
	actions all work with just the compiled executable though. The reason for 
	this is that those things can all be shown in a reasonable way using the class 
	names and \code{.ToString()} methods of the runtime classes whereas displaying 
	the system in source code form  requires the abstract syntax tree, which is no 
	longer available after the application has been compiled. 
	
	This is further complicated by the fact that the viewer is not specific to any
	particular process language, it is meant to work with any language that uses
	the PLR. The problem then becomes how can the viewer know which parser to use
	to parse the abstract syntax tree from the provided source file. To solve this
	the viewer has an associated configuration file which lists the filenames of
	all assemblies that contain parsers. The parsers can then implement an 
	\code{IParser} interface that is provided by the PLR. The interface has a
	\code{Parse} method as well as properties for the language name and the file
	extensions used by that language. The viewer inspects the assemblies listed
	in the configuration file and loads all classes that implement \code{IParser}.
	When a source file is selected the viewer can then lookup the correct parser
	for it by filename extension and use that parser to parse the file, or throw
	an error if no suitable parser is found. The \code{IParser} interface also 
	contains a property that returns a \code{BaseFormatter} instance (previously 
	discussed in Section~\ref{sec:visitor}) which can be used to get a source code 
	representation of the whole, or parts of, the abstract syntax tree.
	
	Once the abstract syntax tree and an appropriate formatter for it are loaded
	then the \code{ProcessStateVisualization} class can create the text 
	representation of the system by starting with the initial processes and 
	then keeping track of which actions are performed and removing the 
	corresponding nodes from the syntax tree. The formatter is then used on those
	parts of the syntax tree that are active at a given time and the source
	code for those active processes is shown in the large textbox in the main 
	Process Viewer window.

\section{Summary}

	The Process Viewer application is a useful tool for monitoring and interacting
	with compiled process language applications. Due to the interoperability of
	.NET the application can directly interact with running process language 
	applications and show processes, traces and candidate actions. It also
	allows the user to select which actions are executed in the process language
	application and shows the state of the processes as they change. When creating
	and working with process language applications the Process Viewer really helps
	the user to understand and follow what is happening during execution.