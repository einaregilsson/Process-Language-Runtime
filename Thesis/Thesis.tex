%  PHDTHESIS TEMPLATE FILE
%  Adopted from Thomas Fabricius and Henrik Aalborg Nielsen
%  Jan Larsen, IMM, DTU, Nov 2003 ver 1.0
%  Updated by Finn Kuno Christensen, fkc@imm.dtu.dk Aug 15, 2008

%  COMPILATION STEPS USING INVOLVING A PS FILE
%\documentclass[10pt,twoside,dvips]{book}
%  latex phdthesis.tex
%  dvips -D600 -Pamz -Pcmz -j0 phdthesis.dvi -o phdthesis.ps
%  ps2pdf -sPAPERSIZE=b5 phdthesis.ps phdthesis.pdf (or use Acrobat Distiller)

%  COMPILATION STEP USING PDFLATEX
\documentclass[10pt,twoside]{book}
%  pdflatex phdthesis.tex


%%%%%%%%%%% MODIFY THESE LINES ONLY %%%%%%%%%%%%%%%%%%%%%%%%%%%%%%%%%%%%%%%%%%%%%%%%%%%%%%%%%
\def\thesisyear{2009} % Year thesis submitted
\def\thesisnumber{XX}  % 70 Only number no year
\def\thesisauthor{Einar Egilsson} % Thesis author
\def\thesistitle{A Process Language Runtime for the .NET Platform} 
\def\thesiskeywords{process algebra compiler .net}
\def\thesisISBN{} %OBSOBS provide ISBN number for industrial phd students ONLY
\def\thesisversion{print} %OBSOBS choose this for printed version send to printing
%\def\thesisversion{net} %OBSOBS choose this for the net version for the web and publication database


\newcommand{\netpar}{\mid\mid}
\newcommand{\val}[1]{\langle\,#1\,\rangle}
\newcommand{\aspect}[1]{\{#1\}}
\newcommand{\defn}[3]{[#1(#2)\triangleq#3]}

\newcommand{\ppar}{\mid}
\newcommand{\Let}{{\bf let}}
\newcommand{\Letin}{{\bf in}}

\newcommand{\Out}[2]{{\bf out}({#1})@{\it #2}}
\newcommand{\In}[2]{{\bf in}({#1})@{\it #2}}
\newcommand{\Read}[2]{{\bf read}({#1})@{\it #2}}
\newcommand{\Eval}[2]{{\bf eval}({#1})@{\it #2}}
\newcommand{\Newloc}[1]{{\bf newloc}({\it #1})}
\newcommand{\nil}{{\sf 0}}

\newcommand{\YZl}{l}
\newcommand{\Judgek}[3]{#2 \rightarrow #3}

\newcommand{\Lnt}{\ell^\lambda}
\newcommand{\Ln}{\ell}
\newcommand{\Lct}{\ell^\lambda}
\newcommand{\Lc}{\ell}
\newcommand{\Lbt}{\ell^\lambda}
\newcommand{\Lb}{\ell}
\newcommand{\Lat}{\ell^\lambda}
\newcommand{\La}{\ell}
\newcommand{\Lett}{\ell'^\lambda}
\newcommand{\Le}{\ell'}
\newcommand{\infarrow}[1]{\stackrel{#1}{\rightarrow}}
\newcommand{\Inference}[2]{\begin{array}{@{}c@{}}#1\\[0em]\hline\\[-0.9em]#2\\
\end{array}}
\newenvironment{ARRAY}[1]{%
  \begin{tabular*}{\textwidth}{@{\extracolsep{\fill}}c@{}c@{}c}
  \hline
  &&\\[-5pt]
  &\begin{math}\begin{array}{@{} #1 @{}}
}
{ \end{array}\end{math}&\\
  &&\\[-5pt]
  \hline
  \end{tabular*}
}

\newcommand{\Lbk}[1]{\ell_{b_{#1}}}

\newcommand{\veck}[1]{\overrightarrow{#1}}


%%%%%%%%%%%%%%%%%%%%%%%%%%%%%%%%%%%%%%%%%%%%%%%%%%%%%%%%%%%%%%%%%%%%%%%%%%%%%%%%%%%%%%%%%%%%%

%%%%%%%%%%%%%%% DO NOT MODIFY START %%%%%%%%%%%%%%%%%%%%%%%%%%%%%%%%%%%%%%%%%%
\def\thesisISSN{0909-3192}
\def\ttitle{{\sf\textbf{\thesistitle}}}
\def\thesisdef{IMM-PHD-\thesisyear-\thesisnumber}
\usepackage{hyperref}
\def\printversion{print}
\ifx\thesisversion\printversion
  \special{papersize=176mm,250mm}
  \hypersetup{pdftitle={\thesistitle},
              pdfauthor={\thesisauthor},
              pdfsubject={\thesisdef},
              pdfkeywords={\thesiskeywords},
              breaklinks,
              bookmarksopen,
              bookmarksnumbered}
\else
  \hypersetup{pdftitle={\thesistitle},
              pdfauthor={\thesisauthor},
              pdfsubject={\thesisdef},
              pdfkeywords={\thesiskeywords},
              colorlinks,
              linkcolor=blue,
              breaklinks,
              bookmarksopen,
              bookmarksnumbered}
\fi


%%%%%%%%%%%%%% DO NOT MODIFY BELLOW START%%%%%%%%%%%%
\usepackage[latin1]{inputenc}
\usepackage[english]{babel}
\usepackage{fancyheadings}
\usepackage{amsmath,amssymb,latexsym,epic,eepic,epsfig,graphics,psfrag}
\usepackage{theorem}
\usepackage{immthesislayout}

\newcommand{\papertitle}{}
\setcounter{tocdepth}{1} % 1 in final version 10 for debugging
\setcounter{secnumdepth}{3} % subsubsections get a number when this is 3

% PDF:
%\usepackage[pdftitle={PARAMETRIC AND NON-PARAMETRIC SYSTEM MODELLING},
%            pdfauthor={Henrik Aalborg Nielsen, IMM, DTU},
%            breaklinks,
%            bookmarksopen,
%            bookmarksnumbered]{hyperref}
%\hypersetup{pdftitle={\ttitle},
%%            pdfsubject={\thesisdef},
%            pdfkeywords={\thesiskeywords},
%            breaklinks,
%            bookmarksopen,
%            bookmarksnumbered}

\begin{document}
\thispagestyle{empty}
\vspace*{\fill}
\begin{center}
{\huge\ttitle}\\*[2.5cm]
\Large\sf\thesisauthor\\*[4.5cm]
\small\sf Kongens Lyngby \thesisyear\\
\small\sf IMM-PHD-\thesisyear-\thesisnumber
\end{center}
\vspace*{\fill}
\newpage
\thispagestyle{empty}
\vspace*{11cm}
{\sf Technical University of Denmark}\\
{\sf Informatics and Mathematical Modelling}\\
{\sf Building 321, DK-2800 Kongens Lyngby, Denmark}\\
{\sf Phone +45 45253351, Fax +45 45882673}\\
{\sf reception@imm.dtu.dk}\\
{\sf www.imm.dtu.dk}

\vspace*{2.5cm}
\def\empty{}
\ifx\thesisISBN\empty
  {\sf IMM-PHD: ISSN \thesisISSN}
\else
  {\sf IMM-PHD: ISSN \thesisISSN, ISBN \thesisISBN}
\fi


\frontmatter
\pagenumbering{roman}

\lstdefinelanguage{CSharp}
{
 morecomment = [l]{//}, 
 morecomment = [l]{///},
 morecomment = [s]{/*}{*/},
 morestring=[b]", 
 sensitive = true,
 morekeywords = {abstract,  event,  new,  struct,
   as,  explicit,  null,  switch,
   base,  extern,  object,  this,
   bool,  false,  operator,  throw,
   break,  finally,  out,  true,
   byte,  fixed,  override,  try,
   case,  float,  params,  typeof,
   catch,  for,  private,  uint,
   char,  foreach,  protected,  ulong,
   checked,  goto,  public,  unchecked,
   class,  if,  readonly,  unsafe,
   const,  implicit,  ref,  ushort,
   continue,  in,  return,  using,
   decimal,  int,  sbyte,  virtual,
   default,  interface,  sealed,  volatile,
   delegate,  internal,  short,  void,
   do,  is,  sizeof,  while,
   double,  lock,  stackalloc,   
   else,  long,  static,   
   enum,  namespace,  string}
}	
\lstdefinelanguage{klaim}
{
 morecomment = [l]{//}, 
 morecomment = [l]{///},
 morecomment = [s]{/*}{*/},
 morestring=[b]", 
 sensitive = true,
 morekeywords = {in, out, read}
}	

\lstset{
	frame=trbl,
	basicstyle=\small,
	commentstyle=\textit,
	escapeinside={/*@}{@*/},
	linewidth=\textwidth,
	showtabs=false,
	showspaces=false,
	showstringspaces=false
}
	

\lstnewenvironment{cil}{ \lstset{basicstyle=\scriptsize\ttfamily} }{}
\lstnewenvironment{csharp} { \lstset{language=CSharp} }{}
\lstnewenvironment{code} { \lstset{basicstyle=\scriptsize\ttfamily} }{}
\lstnewenvironment{ccs} { \lstset{basicstyle=\footnotesize\ttfamily} }{}
\lstnewenvironment{klaim} { \lstset{language=klaim,basicstyle=\footnotesize\ttfamily} }{}
\lstnewenvironment{xml} {\lstset{language=xml,basicstyle=\footnotesize} }{}


%%%%%%%%%%%%%%% DO NOT MODIFY END %%%%%%%%%%%%%%%%%%%%%%%%%%%%%%%%%%%%%%%%%%%%

%%%%%%%PREFACE CHAPTERS INCLUDE%%%%%%%%%%%%%%%%%%%%%%%%%%%%%%%%%%%%%%%%%%%%%%

\chapter{Abstract}

Process languages, also known as Process Algebras or Process Calculi, are languages that are built up of distinct processes communicating with each other. Different process languages have different features, but most of the prominent ones have a common subset of features. That subset includes action prefixing, parallel composition of subprocesses and non-deterministic choice between paths.

The .NET platform is a popular development platform. One of its strengths is
that it supports multiple languages running on the same underlying virtual machine. The languages compile down to a common bytecode format which means that the languages can interoperate and different parts of the same application can be built in different languages.

This thesis explores how well Process Languages can be integrated into the .NET environment and how they can interoperate with code written in other languages.  The design and implementation of an extensible compiler backend and a runtime library for process languages are presented, as well as two case studies of languages implemented using the common compiler and runtime. Finally a quick overview is given on how to integrate a process language into a state-of-the-art integrated development environment.


\markboth{}{} 
\chapter{Preface}

This thesis was prepared at Informatics Mathematical Modelling,
the Technical University of Denmark in partial fulfillment of the
requirements for acquiring the Ph.D.\ degree in engineering.

The thesis deals with different aspects of mathematical modeling of
systems using data and partial knowledge about the structure of the
systems.  The main focus is on extensions of non-parametric methods,
but also stochastic differential equations and neural networks are
considered.

The thesis consists of a summary report and a collection of ten
research papers written during the period 1996--1999, and elsewhere
published.

\vspace{20mm}
\mbox{}\hfill
\begin{minipage}[t]{80mm}
  Lyngby, July 2009
  \\ \\
  %%\mbox{} \hspace{-16mm} \includegraphics{signature.eps}
  Einar Egilsson
\end{minipage}

\markboth{}{}

%%%%%%%PREFACE CHAPTERS INCLUDE%%%%%%%%%%%%%%%%%%%%%%%%%%%%%%%%%%%%%%%%%%%%%%


\newpage\mbox{}\newpage
\chaptermark{Contents}
\renewcommand{\sectionmark}[1]{\markright{#1}}
\sectionmark{Contents}
\addtolength{\parskip}{-\baselineskip}
\tableofcontents
\addtolength{\parskip}{\baselineskip}
\renewcommand{\sectionmark}[1]{\markright{\thesection\ #1}}
\newcommand{\TODO}[1]{\textcolor{red}{TODO: #1}}

\mainmatter
% Chapter 1, 2, ... 
%%%%%%%MAIN CHAPTERS INCLUDE%%%%%%%%%%%%%%%%%%%%%%%%%%%%%%%%%%%%%%%%%%%%%%

\chapter{Introduction}

	In the last decade or so, programming languages have increasingly started to 
	target virtual machines instead of specific physical machine architectures. 
	This approach has a number of benefits. Many virtual machines have 
	implementations on different machine architectures, which enables an 
	application developer to write applications in a language that targets the 
	virtual machine and getting the benefit of his application running on 
	multiple machine architectures for free. Another benefit of the virtual 
	machines is that code written in different languages can inter operate, 
	allowing developers to write each part of their application in a language 
	that suits it best. The two most prominent virtual machines used today are 
	the Java Virtual Machine (JVM) and the Common Language Runtime (CLR). The 
	JVM was originally created by Sun Microtechnologies but several 
	implementations are now available by many vendors. The CLR was created by 
	Microsoft for the Windows platform, but an open source version, Mono, which 
	works on Unix and Linux platforms is also available. Both the JVM and CLR 
	have a number of different languages that target them, and can therefore be 
	used to build \textit{cross-platform} applications. 

	Process languages are a class of languages that are made up of distinct 
	processes communicating with each other. Some examples of process languages 
	are \textit{Calculus of Communicating Systems} (CCS), \textit{Communicating 
	Sequential Processes} (CSP), $\pi$ \textit{calculus} and \textit{Kernel 
	Language for Agents Interaction and Mobility} (KLAIM). The process languages 
	mentioned here have a number of traits in common, including action 
	prefixing, non-deterministic choice and parallel composition. This leads to 
	the assumption that maybe some of these common traits can be abstracted away 
	into a common framework, so that each implementation of a process language 
	does not have to implement them separately.

	Implementing a process language, and making it target a popular virtual 
	machine such as the CLR is an interesting proposition. A number of issues 
	can be explored. How well does the instruction set of the virtual machine 
	fit the execution model of the process language? Is there benefit in having 
	access to the large standard library that comes with the CLR? Is it feasible 
	to write parts of a process language application in another language that 
	targets the virtual machine, for instance numerical functions? A number of 
	advanced tools and integrated development environments exist for CLR and JVM 
	languages, can process languages make use of them? 

\section{Thesis Objectives}

	This thesis presents the design and implementation of the \textit{Process 
	Language Runtime}, hereafter referred to as the PLR. The PLR consists of two 
	main components. First, it contains an extensible abstract syntax tree, 
	which models common idioms of process languages, and can compile itself to 
	.NET bytecode. Secondly, it contains a runtime library that is used by 
	process applications that have been compiled from the PLR syntax tree. 

	Integration with the .NET platform is also explored, e.g. how to allow 
	process language applications to call code developed in other .NET languages 
	and whether it is possible to use existing .NET development tools to aid in 
	writing process language applications.

	Finally the thesis presents two case studies of process language 
	implementations that were done using the PLR. The languages implemented were 
	CCS (Calculus of Communicating Systems) and a subset of KLAIM (Kernel 
	Language for Agents Interaction and Mobility).

	The work carried out consists of the following parts:

	\begin{itemize}
  	\item Design and implementation of the PLR abstract syntax tree and 
  				compiler.
  	\item Design and implementation of the runtime library.
  	\item Implementation of the process language CCS.
  	\item Implementation of the process language KLAIM.
  	\item Integration of the CCS language into Visual Studio, a 
  				state-of-the-art integrated development environment for .NET 
  				development.
	\end{itemize}


\section{Thesis Outline}

	The thesis consists of seven chapters. This introduction is the first 
	chapter, the rest are as follows:

	Chapter 2 gives some background on process languages in general and their 
	common properties, and some technical background on the .NET platform. The 
	concepts presented there are useful for understanding the architecture of 
	the PLR.\\
	Chapter 3 describes the Process Language Runtime, both the syntax tree and 
	runtime library and shows how they are designed and implemented. \\
	Chapter 4 is a case study of the implementation of the CCS process 
	language.\\
	Chapter 5 is another case study, this time of the implementation of the 
	KLAIM language.\\
	Chapter 6 gives a quick overview of how the CCS language was integrated into 
	Visual Studio 2008. \\
	Finally, Chapter 7 contains concluding remarks and ideas for further 
	development of the PLR, including how some other common process languages 	
	could be implemented using the PLR.\\

	\TODO{Appendix?}

\chapter{Background}

\section{Process Languages}

bla

\section{The .NET Platform}

bla

\chapter{Process Language Runtime}

	This chapter presents the design and implementation of the Process Language 
	Runtime, an extensible compiler backend and a runtime library for running 	
	process languages on the .NET platform. 

\section{Inspiration}

	The inspiration for the Process Language Runtime comes from the Dynamic 
	Language Runtime \TODO{NEED REF} (DLR), a framework from Microsoft for 	
	developing dynamic languages on top of the Common Language Runtime. Since 
	the .NET Common Intermediate Language is statically typed it is ill-suited
	for dynamic languages such as Python, Ruby or JavaScript. To clarify, 
	statically typed languages are languages where the type of a variable is 
	known at compile time and the type of a variable never changes, while in a 
	dynamically typed language a single variable can contain objects of 	
	different types at different times during program execution. The idea of 
	having a common abstract syntax tree for different languages comes from the 
	DLR. However, the DLR was not directly used for this project since its 
	abstract syntax tree is mainly concerned with traditional constructs for 
	imperative programming languages and constructs to support dynamic typing, 
	while the purpose of this project is to provide constructs common to process 
	languages, and dynamic typing is not an issue we are concerned with.

\section{Overview}

	The PLR is a single .NET assembly, named PLR.dll. This assembly contains the 
	abstract syntax tree and associated helper objects as well as all the 
	classes used at runtime. The PLR does not contain any lexer or parser and 
	generally has no notion of any concrete syntax. Creating a lexer and parser 
	is the responsibility of individual language implementations, the PLR takes 
	over once an input file has been parsed and used to construct a PLR abstract 
	syntax tree. 
	
	It is important to note that the PLR does not, nor is it meant 
	to, support all constructs of all process languages. Creating a superset of 
	all existing process languages has never been the goal, instead the goal is 
	to provide a common subset of the most common constructs found in these 
	languages and to make it easy to extend with specific new constructs needed 
	for specific languages. To enable this, the classes and interfaces in the 
	PLR have been engineered to make them easy to subclass and implement.
	
	As the PLR contains classes needed at runtime it must be distributed with 
	any compiled process language application. However, an option is present in 
	the compilation stage that allows the PLR assembly to be embedded in the 
	final compiled executable program, and can optionally embed any additional 
	runtime libraries that specific languages require. This allows a process 
	language program to be distributed as a single file without any external 
	dependencies other than the .NET framework itself.

	The PLR itself is written in the C\# programming language using Visual 
	Studio 2008 as the development environment. It has a dependency on the NUnit 
	unit test framework, however this dependency is only needed when running 
	internal unit tests and so does not need to be distributed with the PLR 
	assembly. The source code for the PLR is licensed under the General Public 
	License (GPL) v3.0.
	
\section{Abstract Syntax Tree}

	The PLR abstract syntax tree is the component that generates CIL 
	bytecode to run a process language application. Once an abstract syntax tree 
	has been constructed, a call to a \code{Compile} method on the tree's root 
	node with the appropriate parameters will create an executable .NET 
	assembly. 

\subsection{Architecture}

	The architecture of the abstract syntax tree is based on Object Oriented 
	principles, namely that an object contains data and methods to operate on 
	that data. As such, each node in the abstract syntax tree knows how to 
	compile itself, there is no compiler class, the whole syntax tree is the 
	compiler. Every node in the syntax tree inherits from an abstract 
	\code{Node} base class which has an abstract \code{Compile} 
	method. Concrete node classes override the \code{Compile} method and in it 
	emit the appropriate byte codes for the language construct that the node 
	represents. 
	
	The compilation itself is recursive, calling the \code{Compile} method 
	on the root node of the tree will cause it to call the \code{Compile} 
	method of its child nodes, who in turn call \code{Compile} on their child 
	nodes and thus the compilation propagates throughout the entire tree. The 
	reason for choosing this architecture was to make the syntax tree easily 
	extendable by language implementors, who can add new nodes to represent new 
	constructs. 
	
	Concrete nodes typically do not inherit directly from the \code{Node} base 
	class, instead they inherit from one of five intermediate classes, 
	\code{Action}, \code{Process}, \code{Expression}, 
	\code{ActionRestrictions} or \code{PreProcessActions}. Below is a short 
	overview of what each of these classes represents.
	
	\code{Action} represents an action taken by the 
	process. This can for instance be sending on a channel, receiving on a 
	channel or calling an arbitrary method. Concrete descendants of this class 
	typically have either no child nodes of their own, or a list of 
	\code{Expression} nodes, representing parameters to a method call or 
	values passed through a channel.

	\code{Process} is the base class for processes. Its descendants include an 
	\code{ActionPrefix} class, a \code{NonDeterministicChoice} class and a 
	\code{ParallelComposition} class. Implementing a new language construct 
	such as replication could be done be creating a new descendant of this 
	class. Child nodes of \code{Process} classes vary, the 
	\code{ActionPrefix} class for example has one \code{Action} child node 
	representing the action about to be performed and one \code{Process} child 
	node representing the process that the current process turns into after 
	performing the action. Processes that are a composition of other processes 
	such as \code{ParallelComposition} and \code{NonDeterministicChoice} 
	have a list of other \code{Process} instances as childnodes.
	
	\textbf{Expression} represents an expression such as an arithmetic 
	expression, numeric or string constant, a method call or the value of a 
	variable.	Expressions compile in such a way that once they have been 
	evaluated a single value, the result of the expression, is at the top of the 
	evaluation stack. This means that a node that has an expression as a child 
	node can simply call the \code{Compile} method on the expression and then 
	emit bytecodes that operate on its result, without caring whether the 
	expression is a huge expression tree or a single constant value. An example 
	of this is the \code{ArithmeticExpression} node. In its \code{Compile} 
	method it first calls the \code{Compile} method of its left child, then 
	its right child and then emits an \code{Add},\code{Sub},\code{Mul} or 
	\code{Div} bytecode. The child nodes of \code{Expression} nodes are 
	invariably \code{Expression} nodes themselves.

 	\textbf{ActionRestrictions} represents a function that restricts actions 
 	within a process from synchronizing with other actions outside the process. 
 	This is an implementation of the \textit{restriction} process language 
 	construct described in Section~\ref{sec:common_constructs}. It currently has 
 	two concrete descendants. One is \code{ChannelRestrictions} which 
 	restricts channels by name, provided that the names of channels to restrict 
 	are known at compile time. The other is \code{CustomRestrictions}, that 
 	calls a .NET method at runtime for every action and returns \texttt{true} it 
 	it should be restricted. The method can be written in any language available 
 	for the .NET framework, the only requirements are that it takes an 
 	\code{Action} object from the PLR runtime library as a parameter and 
 	returns a boolean value.
	 
 	\textbf{PreProcessActions} represents a function that is called for every 
 	action that is performed in a process and returns another action. This is 
 	used to implement the \textit{re-labelling} process language construct 
 	described in Section~\ref{sec:common_constructs}. Simple re-labelling of 
 	channels with names known at compile time is done with a 
 	\code{RelabelActions} class. It has the names to re-label as child nodes 
 	and compiles down to a method that takes in an \code{Action} runtime class 
 	and performs simple string substitution on its name. More complicated 
 	pre-processing of actions can be achieved with another descendant class, 
 	\code{CustomPreprocess}. That class compiles down to a method call to a 
 	.NET method that takes an \code{Action} as a parameter and returns an 
 	\code{Action} as well. This method can be written in any language 
 	available for .NET.
 	
 	Besides all the descendant classes of those five main classes, there are a 
 	few classes that inherit directly from the \code{Node} class. 
 	\code{ProcessSystem} is the root node of the entire syntax tree and has a 
 	more complicated \code{Compile} method than most other nodes, since it 
 	takes care of setting up the necessary context for the compilation and 
 	creating the actual compiled file, giving it a name and so forth. 
 	\code{ProcessDefinition} is a simple class that just has a 
 	\code{Process} child node and a name for the process. Finally, 
 	\code{ExpressionList} is a convenience class to hold a list of 
 	\code{Expression} instances.
 	
	We now look at a simple example of a coffee machine, CM. The coffee machine 
	accepts a coin as input, then outputs coffee and then makes a non 
	deterministic choice between turning into itself again or turning into a 
	process representing a failure of the coffee machine, CMFAIL. The CMFAIL 
	process accepts a coin and then turns into the nil process without ever
	returning coffee for the inserted coin.

	\begin{Exa}
	\label{ex:coffee_machine_syntax}
	\begin{align*}
			\mathrm{CM} \defeq & coin \ccsdot \out{coffee} \ccsdot (\mathrm{CM}+\mathrm{CMFAIL})\\
			\mathrm{CM} \defeq & coin \ccsdot \mathrm{0}
	\end{align*}	
	\end{Exa}

	The syntax tree for the processes in Example~\ref{ex:coffee_machine_syntax} 
	is shown in Figure~\ref{fig:syntax_tree_example}. The names shown in 
	boldface are the names of the node classes, while the text in 
	parentheses shows properties of the nodes. Note that although this 
	particular tree is strictly binary it does not mean that all PLR trees
	are binary trees. \code{ProcessSystem}, \code{ParallelComposition} and 
	\code{NonDeterministicChoice} nodes can all have 1-n child nodes.
	
	
	\begin{figure}[h!]
		\centering
		\includegraphics[scale=0.7]{syntax_tree_example.jpg}
		\caption{PLR abstract syntax tree}
		\label{fig:syntax_tree_example}
	\end{figure}
 	

\subsection{Processing the syntax tree before compilation}\label{sec:visitor}
	
	There are many reasons why a language implementation might need to process 
	the abstract syntax tree in some way before compilation. An example might be 
	optimization; to fold constant expressions or prune branches of the tree 
	that are sure to never be executed. To support scenarios like this, the PLR 
	makes use of the \textit{Visitor} design pattern. The pattern is a way of 
	separating an algorithm from an object structure upon which it operates. As 
	a result, new operations can be added to existing object structures without 
	modifying those structures. A visitor interface contains one \code{Visit} 
	method for each of the classes in the object structure, each class in the 
	object structure then contains an \code{Accept} method that takes the 
	visitor interface as a parameter and does nothing except call the visitor's 
	\code{Visit} method with itself as a parameter. An example of this is 
	shown in Figure~\ref{fig:visitor_example}. This technique, calling the 
	objects \code{Visit} methods that immediately calls the visitors 
	\code{Accept} methods is known as \textit{double dispatch}. This 
	essentially mimics virtual method overloading, but the added benefit is that 
	the methods can be defined outside the object structure, making it easy to 
	plug in different implementations of the visitor as needed. The visitor 
	class can then contain different implementations of traversing the object 
	structure while calling the \code{Accept} method on each of its nodes. A 
	more detailed explanation of the visitor pattern and its benefits 
	can be found in \cite{design_patterns, visitor}.
	
	\begin{figure}
	\begin{csharp}
  
//Each node in the tree contains this method
public override void Accept(AbstractVisitor visitor) {
  visitor.Visit(this);
}
  
//The Visit method for ActionPrefix node in a subclass
//of AbstractVisitor. 
public override void Visit(ActionPrefix node) {
  //...process the ActionPrefix node here
}
\end{csharp}
\caption{Examples of Visit and Accept methods}
\label{fig:visitor_example}
	\end{figure}
  
  The PLR contains an \code{AbstractVisitor} class which is the base class
  of all visitor implementations. This was implemented as an abstract class
  rather than an interface for convenience reasons; the 
  \code{AbstractVisitor} provides empty implementations of the \code{Visit}
  method for each of the nodes in the abstract syntax tree, subclasses only
  need to override the \code{Visit} methods for nodes which they
  are interested in processing. Depth first traversal is a very common method
  of working with tree structures, to account for that common case the 
  \code{AbstractVisitor} contains a \code{VisitRecursive(Node node)} 
  method which performs a depth first recursive traversal of the tree, calling 
  each nodes \code{Accept} method along the way. A  boolean property on the 
  \code{AbstractVisitor} named \code{VisitParentBeforeChildren} controls 
  whether the parent or child nodes \code{Accept} methods are called first 
  during the traversal.  
  
  Process algebras are the subject of many academic papers, and it is likely 
  that future users of the PLR might be in the process of writing research 
  papers themselves. For that reason it could be quite useful to be able to 
  get a text representation of the abstract syntax tree, formatted in the
  LaTeX typesetting format (it certainly has been very useful for this 
  author!). The PLR contains three different formatter classes that generate 
  text representations of the syntax tree in different formats. 
  
  \begin{itemize}
  	\item \code{BaseFormatter} generates unformatted process algebra text, 
  	using the common symbols for constructs like non deterministic choice and 
  	parallel composition. 
  	
  	\item \code{LaTeXFormatter} generates LaTeX source, suitable for copying 
  	directly into a LaTeX document.
  	
  	\item \code{HTMLFormatter} generates HTML formatted text, suitable for 
  	displaying on the web.
  \end{itemize}
  
  The concrete syntax used by all of these formatters is that of CCS, however, 
  since many of the common process algebras use the same symbols for things 
  such as parallel composition and non deterministic choice, formatters for 
  other languages could be implemented simply by inheriting from one of the 
  three aforementioned classes and overriding the formatting methods only for 
  those constructs whose syntax differs from CCS syntax.
  
  The formatter classes are all implemented using the visitor pattern. The 
  benefits of the pattern here are obvious, it is easy to add new formats at a 
  later date without having to alter anything in the syntax tree itself, all 
  code for a particular format is kept in one place and formatter classes do 
  not need to implement tree traversal algorithms themselves.

	Figure~\ref{fig:expression_folder} shows another example of how the 
	visitor pattern can be useful. It is a class that takes binary 
	expressions that contain constants on both the left and right hand side, 
	calculates their results and replaces the \code{ArithmeticBinOpExpression} 
	node (which contains two child nodes) with a single \code{Number} node 
	containing the result of the expression. The code listing shows the entire 
	\code{ExpressionFolder} class, all it needs to do is inherit from 
	\code{AbstractVisitor} and override the \code{Visit} method that takes 
	\code{ArithmeticBinOpExpression} as a parameter. The visitor is executed 
	by calling \texttt{folder.VisitRecursive(tree)} where \texttt{folder} is an 
	instance of \code{ExpressionFolder} and \texttt{tree} is any \code{Node} 
	in the syntax tree, usually the root. Note that the folding only happens if 
	both the left and right node are \code{Number} nodes, however, the fact 
	that the tree is traversed depth first and child nodes are visited before 
	parent nodes ensures that even deeply nested expressions are folded as much 
	as possible. The leafs are visited first and folded if possible, by the time 
	the upper level nodes in the expression tree are visited they will have 
	\code{Number} nodes as children where previously were
	\code{ArithmeticBinOpExpression} nodes, and thus they can be replaced 
	as well. (Of course this example is fairly contrived, it is hard to think of
	a legitimate reason for writing down a large expression where all elements
	are constants. It does however show how the visitor pattern can be used to 
	implement classes that alter the syntax tree with a minimal amount of code).
	
	\begin{figure}[h!]
	\begin{csharp}
class ExpressionFolder : AbstractVisitor{

  public override void Visit(ArithmeticBinOpExpression exp) {
    if (exp.Left is Number && exp.Right is Number) {
      int result = 0, leftVal, rightVal;
      leftVal = ((Number)exp.Left).Value;
      rightVal = ((Number)exp.Right).Value;
		
      if (exp.Op == ArithmeticBinOp.Plus) {
        result = rightVal + leftVal;
      } else if (exp.Op == ArithmeticBinOp.Minus) {
        result = rightVal - leftVal;
      } else if (exp.Op == ArithmeticBinOp.Multiply) {
        result = rightVal * leftVal;
      } else if (exp.Op == ArithmeticBinOp.Divide) {
        result = rightVal / leftVal;
      }
      int pos = exp.Parent.ChildNodes.IndexOf(exp);
      exp.Parent.ChildNodes[pos] = new Number(result);
    }
  }
}
\end{csharp}
\caption{Expression folder implemented using Visitor pattern} \label{fig:expression_folder}
\end{figure}

\subsection{Extensibility}
	
	As stated before, one of the goals of the PLR is extensibility, allowing
	for language implementors to add features and constructs not included in
	the PLR itself. There are three main methods of extending the PLR. Firstly,
	language implementors can add new nodes to the abstract syntax tree. These
	nodes just have to implement the \code{Compile} method and then they can
	be seamlessly integrated with the built in PLR nodes. Of course it is also
	possible to inherit from one of the existing nodes and thereby re-using some
	of the compilation work they do, and simply adding extra code before or after
	the base classes compilation step. 
	
	Secondly, the root node of the abstract	syntax tree, \code{ProcessSystem} 
	exposes the following four events. 
	
	\begin{enumerate}
		\item \code{BeforeCompile} occurs before the PLR has performed any 
		compilation. At this point only the \code{AssemblyBuilder} and 
		\code{ModuleBuilder} have been defined, no types or methods exist yet.
		
		\item \code{AfterCompile} occurs after the PLR has finished all its
		compilation but before it creates the executable file. At this point
		subscribers to this event can access any types or methods created during
		compilation.
		
		\item \code{MainMethodStart} occurs just after the main method of the
		application has been defined but before any bytecodes have been emitted
		into it. Subscribers of this event can then inject their own bytecodes at 
		the beginning of the main method if they wish.
		
		\item \code{MainMethodEnd} occurs after all bytecodes of the main method
		have been emitted, except for the final \code{Ret} instruction. Again, 
		subscribers of this event can inject their own bytecodes at this point.
	\end{enumerate}
	
	All these events have the same signature, they require a 
	\code{CompileEventHandler} delegate, which takes a \code{CompileContext} 
	as a parameter. The event subscribers then use the compile context to create 
	types, methods and emit bytecodes at different points in the compilation 
	process.
	
	Finally, the third way to extend the PLR is to write supporting code in
	another .NET language, writing a seperate runtime library. The KLAIM 
	implementation described in Chapter~\ref{ch:klaim} takes this approach. When 
	a language implementation requires large amounts of supporting code it is 
	inconvenient and error prone to generate all that code by emitting CIL 
	bytecode at compilation time. By creating a runtime library instead, 
	language implementors can get the benefit of programming languages and tools 
	such as C\# and Visual Studio when writing the common, re-usable parts of 
	their languages. Then, at compilation time, they can simply emit bytecodes 
	to call code in the runtime library. This is also the approach taken by the 
	PLR itself, which has runtime classes written in C\# and emits bytecodes 
	during compilation that interact with these classes.	

\subsection{Code generation}
	
	To generate a valid .NET assembly the PLR uses a set of classes that are a 
	part of the .NET framework Base Class Library. These classes are located in 
	the \code{System.Reflection.Emit} namespace. Before explaining more about 
	these classes it is worth going over how a .NET assembly is structured. A 
	.NET \textit{assembly} is an executable file (.exe) or a dynamic link 
	library (.dll). The assembly contains one more \textit{modules}, typically 
	just one. Each module contains one or more \textit{types} (or classes). 
	Types have \textit{fields}, \textit{constructors} and \textit{methods}. At 
	the lowest level, constructors and methods contain CIL bytecodes. 
	Figure~\ref{fig:assembly} shows the structure.

	\begin{figure}[h!]
		\centering
		\includegraphics{assembly2.jpg}
		\caption{The structure of a CIL assembly}
		\label{fig:assembly}
	\end{figure}
	
	The classes in the \code{System.Reflection.Emit} namespace match the 
	structure of an assembly. There is an \code{AssemblyBuilder}, 
	\code{ModuleBuilder}, \code{TypeBuilder}, \code{FieldBuilder}, 
	\code{ConstructorBuilder} and a \code{MethodBuilder}. These are instantiated 
	by giving them names and other properties as parameters. The 
	\code{ConstructorBuilder} and \code{MethodBuilder} have a 
	\code{GetILGenerator} method that returns an object of type 
	\code{ILGenerator}. That object has direct access to the bytecode stream of 
	the method being created, and contains various overloads of an \code{Emit} 
	method that emits bytecodes and their associated arguments. An 
	\code{OpCodes} class contains constants for all possible bytecodes that can 
	be emitted by the \code{ILGenerator}.
	
	The nodes of the syntax tree gain access to these classes through a 
	\code{CompileContext} class which is part of the PLR, and is a parameter 
	to the \code{Compile} method implemented by all nodes. The 
	\code{CompileContext} class has a number of useful properties that the 
	nodes can access. It exposes the \code{TypeBuilder} object of the type 
	currently being built, the \code{ILGenerator} object of the method or 
	constructor being built and a symbol table for variables currently in scope. 
	The node can then emit its bytecodes, create new variables or otherwise 
	alter the \code{CompileContext} before passing it on to its child nodes 
	\code{Compile} methods. Essentially this is a form of distributed 
	compiling, no one node has a complete picture of what is being compiled, 
	each node only has enough information to add its own code to the correct 
	type or method.

\subsection{Debugging support}\label{debug_support}
	
	One of the benefits of targeting a common virtual machine such as the CLR is
	is that both a free command line and graphical debugger exist that can be 
	used for any programming language that compiles down to the Common 
	Intermediate Language format. The \code{System.Reflection.Emit} API offers 
	functionality to emit the necessary debugging symbols to be able to use 
	these debuggers. Emitting debug symbols consists of the following five steps:
 	
 	\begin{enumerate}
 		\item When the \code{ModuleBuilder} objects is defined with a call to 
 		the \code{DefineDynamicModule} method on the \code{AssemblyBuilder} 
 		object, a parameter named \code{emitSymbols} should be passed as 
 		\texttt{true}.
 		
		\item An item of the type \code{ISymbolDocumentWriter} needs to be 
		defined. This is done with a call to a \code{DefineDocument} method on 
		the \code{ModuleBuilder} object which returns a 
		\code{ISymbolDocumentWriter} object. The parameters to this method call 
		include the name of the source file that is being compiled, this is 
		neccessary so that the debugger can prompt for the source file when 
		debugging the compiled file. The \code{ISymbolDocumentWriter} object is 
		passed with the \code{CompileContext} to all nodes during compilation.

		\item Local variables in methods are created with a \code{LocalBuilder} 
		object. In a non debug build these locals are not stored by name in the 
		compiled file, but simply given a number and referred to by that number.
		To be able to map variables in the compiled file to variable names in the
		source file a method, \code{SetLocalSymInfo} is called on the 
		\code{LocalBuilder} object. The method takes the name of the variable as
		a parameter and stores that information for later use by the debugger.
		
		\item The method \code{SetUserEntryPoint} must be called on the 
		\code{ModuleBuilder} object to enable the debugger to know what the entry
		method of the assembly is. The method takes a \code{MethodBuilder} object
		as a parameter.
		
		\item The most important part of emitting the debug symbols is marking 
		\textit{sequence points} in the CIL bytestream. A sequence point is a 
		point in the bytecode that tells the debugger to stop at that point during 
		code execution and highlight a particular section in the source code file. 
		To be able to do this the sequence point contains information about a 
		start position and end position in the source file, given as line and 
		column numbers. A sequence point is marked with a call to a 
		\code{MarkSequencePoint} method on an \code{ILGenerator} object, the 
		methods parameters are an instance of \code{ISymbolDocumentWriter} and 
		four integers, startLine, startColumn, endLine and endColumn. 
		Figure~\ref{fig:sequence_points} shows a few lines of CIL bytecode 
		interspersed with sequence points. (Note: the CIL file format does not 
		store sequence points in exactly this manner, the figure is simply meant 
		to clarify the concept). It is worth noting that the CIL has no notion of 
		statements, expressions or other programming language constructs, it is 
		perfectly legal to insert a sequence point in the middle of an expression
		or anywhere else in the bytecode. It is completely up to the programmer to
		insert sequence points at meaningful points in the bytestream according to
		the semantics of the language being implemented.
		
 	\end{enumerate}
 	
 	The PLR handles these five steps, so an implementation of a language that 
 	uses the PLR as its backend compiler does not need to concern itself with 
 	them directly. However, since the PLR does not handle parsing of source 
 	files it can not determine itself the line and column numbers needed for 
 	marking sequence points. For that purpose, every node in the PLR abstract 
 	syntax tree has an instance of a class named \code{LexicalInfo}. This 
 	class is simply a wrapper around the four integers that a sequence point 
 	needs, startLine, startColumn, endLine and endColumn. This information is 
 	easily available during parsing and so the individual language parsers 
 	should store this information for each node. The PLR will then automatically 
 	emit a sequence point before every \textit{action} taken by a process, as 
 	well as for expressions evaluated in \texttt{if-then-else} processes and for 
 	process invocations.

	\begin{figure}
	\begin{cil} 
.method private hidebysig static void  Main() cil managed
{
  .entrypoint
  // Code size       22 (0x16)
  .maxstack  8
  IL_0000:  nop
//debugger stops, highlights line 12, col 9-36
/*@\textbf{[SEQUENCEPOINT (12, 9, 12, 36)]}@*/
  IL_0001:  ldstr      "Hello"
  IL_0006:  call       void Program::WriteLine(string)
  IL_000b:  nop
//debugger stops, highlights line 13, col 9-26
/*@\textbf{[SEQUENCEPOINT (12, 9, 12, 36)]}@*/
  IL_000c:  ldc.i4.s   27
  IL_000e:  ldc.i4.4
  IL_000f:  call       void Program::Power(int32, int32)
  IL_0014:  nop
  IL_0015:  ret
} // end of method Program::Main
  \end{cil}
  \caption{CIL bytecode with sequence points}
  \label{fig:sequence_points}
	\end{figure}
	
\section{Runtime Library}

	The PLR has a small runtime library consisting of nine classes. These classes
	reside in the \code{PLR.Runtime} namespace. Applications compiled using 
	the PLR must have access to this library at runtime in order to execute 
	successfully. Figure~\ref{fig:runtime_library} shows a class diagram of the 
	runtime library. Below is an overview of each of the nine classes.

	\begin{figure}[ht!]
		\centering
		\includegraphics[scale=0.5]{RuntimeLibrary.png}
		\caption{PLR runtime library classes}
		\label{fig:runtime_library}
	\end{figure}

	\textbf{ProcessBase} is an abstract base class for any compiled processes.
	It contains methods used to interact with the \code{Scheduler},
	for instance method to synchronize on channels, methods for startup and
	termination as well as methods to suspend and resume the process thread.
	Since processes are built up into a tree-like structure at runtime (further 
	explained in Section~\ref{sec:cil_architecture}) it also contains a field 
	for its parent \code{ProcessBase} instance. Finally, it contains a list of 
	\code{IAction} instances, this list will hold all actions that occur in 
	the process instance or any of its subprocesses and are restricted by the 
	instances restriction clause.
	
	\textbf{BuiltIns} is a small static class that contains utility methods
	that can be called by processes, such as to print to the console.
	
	\textbf{IAction} is an interface that all runtime actions must implement.
	It contains four methods. \code{ProcessID} returns the id of the
	process performing the action, \code{IsAsynchronous} returns true if
	the action can be executed without synchronizing with another action, 
	this applies for instance to arbitrary method calls to .NET methods.
	\code{CanSyncWith(IAction other)} determines whether the action can be 
	synced with another action, in the case of asynchronous actions this
	method always returns false. Finally, \code{Sync(IAction other)} is 
	called on those actions that have been chosen for execution and is used
	for example to pass values from one process to another through channels.

	\textbf{ChannelSyncAction} is a class representing synchronization on a
	channel. It contains the channel name, the id of the process performing
	the action and information about whether the process is attempting to send
	or receive on the channel. In the case of a send operation it can optionally
	contain a list of values that are being sent, and in the case of a receive
	operation it can contain a list of variables that should be bound to the 
	values being sent from the other side. The \code{ChannelSyncAction}
	implements the \code{IAction} interface as all runtime actions must, it is
	a synchronous action and its \code{CanSyncWith} method will only return 
	true for other \code{ChannelSyncAction} instance that have the same
	channel name, are performing the opposing operation and have the same
	number of values being passed through the channel. When two actions are
	synchronized, the \code{Sync(IAction other)} method on both the actions
	are called with the other actions as a parameter. In the case of channel 
	synchronizations the action instance which represents the receiving end
	of the operation will bind its variables to the values passed through
	the channel in its \code{Sync} method, the sending action will do nothing
	in its own \code{Sync} method.
	
	\textbf{MethodCallAction} is a runtime action which can be used to call
	an arbitrary .NET method, either a built-in method from the .NET base
	class library or a method from any .NET assembly. It is an asynchronous 
	action and as such does not need to synchronize with another process to be 
	executed. Currently	the PLR provides support for calling static methods 
	that have integers or	strings as parameters. This allows for instance most
	of the methods from the \code{System.Math} class to be accessible. To
	gain access to instance methods, for example the \code{NextInt} method
	of the \code{System.Random} class, it is necessary to write static wrappers
	around the methods.

	\textbf{Logger} is a utility class for handling process output to the screen.
	Its main feature is assigning a different color to each process to 
	easily distinguish between them in the console output.

	\textbf{GlobalScope} is a small class whose only purpose is to be a 
	repository of possible actions that are not restricted by any process. As
	explained in Section~\ref{sec:cil_architecture}, candidate actions are 
	propagated up the process tree, and at each process it is checked whether 
	the process restricts them, if so they are stored within that process so 
	that they do not synchronize with actions outside the process. In the case 
	where no process restricts the action and it can synchronize with any other 
	action that is not otherwise restricted then the action is stored in the 
	global scope, while it waits to see whether it was chosen for execution.
	
	\textbf{ProcessKilledException} is an exception class used when processes
	are killed. As an example, when a process is a candidate in non deterministic
	choice and is not chosen then it must be killed. To bypass all the subsequent
	actions of the process, a \code{ProcessKilledException} is thrown and then
	caught at the end of the processes code. There the process will unregister
	itself from the \code{Scheduler}, print a message to the console and then
	terminate.
	
	\textbf{Scheduler} is the real execution engine of the PLR. It follows the
	\textit{Singleton} \cite{design_patterns} design pattern so it is trivial 
	for all	processes in the application to gain access to the same 
	\code{Scheduler} instance. When processes are activated they register 
	themselves with the scheduler, which keeps a list of active processes. The 
	scheduler then monitors the processes and waits until all processes have 
	generated all their candidate actions and are waiting for an action to be 
	executed so they can continue. At that point the scheduler goes through all 
	the possible actions, figures out which actions can sync with each other and 
	then randomly chooses an action to execute. It then executes the action and 
	wakes up the processes involved in the action so that they can resume 
	execution. It also terminates certain processes (or rather instructs them to 
	terminate themselves). These are generally candidates of non deterministic 
	choice who were not chosen. Once the scheduler has finished one such round 
	it again waits until all the processes it woke up are again suspended and 
	then chooses the next action to execute, and so on. 
	Figure~\ref{fig:scheduler} shows the workings of the scheduler in 
	pseudocode. Other responsibilities of the scheduler are thread locking and 
	synchronization, and keeping track of the \textit{trace}, that is the list 
	of actions executed during the duration of the program.
	 
	\begin{figure}
	\begin{codeblock}
// active_procs contains all the running processes

do forever:
  all_blocked = true
  
  for every process p in active_procs
    if state(p) != STATE_BLOCKED
      all_blocked = false
    
  if all_blocked = true
    //Find matches...      
    candidate_matches = []
    for every process p in active_procs:
      //p.restricted_actions contains those actions
      //being performed within p that are restricted 
      //by p and so can not go into the global_scope 
      //and sync with any other action
      for every action a in p.restricted_actions
        for every action b in p.restricted_actions
          if a can sync with b
            candidate_matches.add( (a,b) )
		
    for every action x in global_scope
      for every action y in global_scope
        if x can sync with y
          candidate_matches.add( (x,y) )
    
    if length(candidate_matches) = 0
    	DEADLOCK, program finishes
    else
      set match = random(candidate_matches)
      execute(match)
      wake up processes that had actions in the match
      kill processes that were not selected 
        in non deterministic choice
	
	\end{codeblock}
	\caption{The scheduler algorithm}
	\label{fig:scheduler}
	\end{figure}
	

\section{CIL Structure of a Process Language Application}\label{cil_structure}

	A process language application is in many ways different from an application
	written in a traditional programming language. One of the goals of this  
	project was to investigate how well process languages are suited to the .NET
	virtual machine. We now look at how a process language system looks once it 
	has been compiled to the Common Intermediate Language.
	
	\subsection{Architecture choices}\label{sec:cil_architecture}
	One of the hardest decisions during the design of the PLR was whether or not 
	to represent each process as a separate thread. Writing multi-threaded code 
	is hard, and it is subject to subtle errors, race conditions and other 
	problems that are easily avoided when using only a single thread. However, a 
	single threaded implementation would be forced to represent the processes as 
	datastructures, rather than as independent programs in their own right. 
	That approach, that the processes are datastructures and the program is a 
	single thread operating on those datastructures is certainly worthwhile, and 
	in fact an early prototype was implemented as an interpreter that did just 
	that. It even makes certain things easier, such as the visualizing the state 
	of the processes after each round. However, using multiple threads allowed 
	for compiling each process relatively independently of other processes, and 
	conceptually seemed closer to the semantics of process algebra. Another
	benefit of the multi-threaded approach was that it made emitting debug 
	symbols fairly simple, while doing the same in a single threaded way would 
	have been problematic. For these reasons the multi-threaded approach was 
	taken.
	
	Another concern when deciding how to structure a process language 
	application was the semantics of restriction and relabeling. When these are 
	applied to a process they keep applying to any subsequent process that it 
	may invoke. E.g. in $(a . P)[d/c]$ the relabeling of $c$ to $d$ has to be 
	applied to everything that happens in the invoked process $P$. To handle 
	this each process has a reference to the process that spawned it in a 
	\code{Parent} property. At runtime $P$ would have $(a . P)[d/c]$ in its 
	\code{Parent} property and whenever it performs an action it will first 
	check whether it restricts or relabels it itself, if not it will pass the 
	action up to its parent which can check again if the action is restricted at 
	that level, and and so it propagates up the chain of parent processes. If an 
	action is restricted at a particular level then it can only synchronize with 
	other actions that are at the same level, those actions that are not 
	restricted at all go on up to the global scope, where they become observable 
	from the outside.
	
	The one problem with that approach is that a lot of processes do not 
	restrict or relabel anything at all, and would then be kept alive for no 
	reason, they would simply take up memory and make the process tree 
	unnecessarily complicated. To avoid this, processes only set themselves as 
	the parent of a spawned process if they actually have some restrictions or 
	relabellings. If they do not then they set their own parent as the parent of 
	their spawned processes. This is perhaps best explained by an example.
	
	\begin{verbatim}
			                      A = a . b . B
			                      B = (b . c . C) \{b,c}
			                      C = c . d . A
			                      D = 0
	\end{verbatim}
	
	In the system above the initial process is $A$. It performs two actions and 
	then turns into $B$. A is not restricted in any way so there is no reason 
	for it to set itself as $B$'s parent. A then checks if itself has a parent, 
	it does not and so it sets $B$'s parent to \code{null} before starting it.
	$B$ on the other hand is restricted, so after it has performed its actions 
	it starts $C$ and sets itself as $C$'s parent. When $C$ performs its $c$ 
	action it is passed up to its parent and is restricted in the $B$ process. 
	When $C$ eventually turns into $D$ it needs to decide what $D$'s parent will 
	be. $C$ itself has no restrictions and so does not need to live on, so it 
	sets $D$'s parent as its own parent, which was $B$. $C$ can now die and be 
	removed from memory, $D$ has $B$ as its parent so the restrictions of $B$ 
	will continue to be applied correctly.
	
	\subsection{Processes}
	Each process in process algebra maps to a class in CIL. The process classes 
	all inherit from a \code{ProcessBase} class in the PLR runtime library and 
	override a \code{RunProcess} method. When a process has been instantiated, a 
	\code{Run} method is called on it, and it will then run the 
	\code{RunProcess} method on a new thread.

	The top level processes, those that are defined as named process constants, 
	are compiled to classes named after the process constant. A top level class 
	can have multiple inner classes however, and each of those can itself have 
	multiple inner classes. This happens for instance when a process starts by 
	performing an action and then turns into a process that is a parallel 
	composition of 	other processes, e.g. $a \ccsdot (b \ccsdot 0 \mid c \ccsdot 
	0)$. In that case each of those parallel processes is an inner class of the 
	original top level process. The same thing happens when a process makes a 
	non-deterministic choice; each of the choices is its own inner class. The 
	reason for making these inner classes was that in a fairly large system the 
	number of classes quickly becomes large, and instead of polluting the top 
	level namespace with dozens of generated class names, they are confined 
	within their owning process. It is also easier to understand what each 
	process represents, the class name \code{PC+Parallel1} represents the 
	first parallel process within the \code{PC} process, which is a 
	parallel composition process. Figure~\ref{fig:struct_parallelcomp} shows the 
	assembly structure of a simple parallel composition process, $PC \defeq a 
	\ccsdot (b \ccsdot 0 \mid c \ccsdot 0)$. (This is a screenshot from a tool 
	called ILDASM from Microsoft, the large boxes with three pins represent 
	classes, the triangles represent metadata about the classes and the small 
	rectangles represent methods).
	
	\begin{figure}
		\begin{center}\includegraphics[scale=0.7]{ex_pc.png}\end{center}
		\caption{Assembly structure of process  $PC \defeq a \ccsdot (b \ccsdot 0 \mid c \ccsdot 0)$}
		\label{fig:struct_parallelcomp}
	\end{figure}
	
	The semantics of action prefixing ($a \ccsdot P$) state that action $a$ is 
	performed and the process then behaves like $P$. Considering how parallel 
	composition and non deterministic choice were implemented with inner classes 
	it might then seem natural to perform $a$ and then invoke an inner class 
	$P$. That is not the case however. The reason is that it is very common to 
	have a list of actions performed, e.g. $a \ccsdot b \ccsdot c \ccsdot d 
	\ccsdot P$, and creating a new class after every single action would result 
	in a large number of classes for no purpose. So for a process such as this, 
	all the actions are performed in the same class, in its \code{RunProcess} 
	method. There are exceptions to this however, if a process performs some 
	actions and then becomes another action prefixed process that is restricted 
	or has relabellings then it will have an inner class at that point. For 
	example the process $a \ccsdot b \ccsdot (c \ccsdot P) \backslash\{c\}$ will 
	perform actions $a$ and $b$ in the main class, but have an inner class for 
	$(c \ccsdot P) \backslash\{c\}$. The reason for this is that restrictions 
	and relabellings always have process scope, and since the restriction on $c$ 
	does not apply to the first actions $a$ and $b$ then a new process is needed 
	after $a$ and $b$ have been performed. Figure~\ref{fig:struct_actionprefix} 
	shows the assembly structure of the process $AP \defeq a \ccsdot b \ccsdot 
	(c \ccsdot P) \backslash\{c\}$. Note that the class \code{AP+Inner} 
	contains \code{get\_Restrict} and \code{RestrictByName} methods, while the 
	outer class \code{AP} does not.

	\begin{figure}
		\begin{center}\includegraphics[scale=0.75]{ex_ap.png}\end{center}
		\caption{Assembly structure of process  $AP \defeq a \ccsdot b \ccsdot (c 
	\ccsdot P) \backslash\{c\}$}
		\label{fig:struct_actionprefix}
	\end{figure}
	
	Process invocations, that is when a process turns into a named process, is 
	implemented simply by creating a new instance of the named process and 
	starting it. The process $c \ccsdot Proc$ performs action $c$ and then 
	creates a new instance of \code{Proc} and starts it, after that has been 
	done $c \ccsdot Proc$ simply exits, \code{Proc} has been started in its 
	place. The only complication here is if a restriction or relabeling is 
	applied to \code{Proc} but not the rest of the process. If the process was 
	written as $c \ccsdot (Proc \backslash\{a\})$ then an inner class would be 
	needed, whose only purpose was to create a new instance of \code{Proc} and 
	start it. This might seem a wasteful inner class, but keep in mind that this 
	restriction lives on and applies to everything that happens in \code{Proc} 
	and any other process that \code{Proc} might turn into. The instance of 
	\code{Proc} that is started will have a reference to the process $(Proc 
	)\backslash\{a\}$ in its \code{Parent} property, any action performed in 
	\code{Proc} will be passed up along the parent chain to see if it is 
	restricted or relabeled at some point.
	
	\subsection{Restrictions and relabellings}
	Restrictions and relabellings map to methods in CIL. The methods do not
	have to be member methods of a process class, there is an extra layer
	of indirection to allow for calling methods in external assemblies. The
	\code{ProcessBase} class has two methods, \code{get\_PreProcess} and 
	\code{get\_Restrict} \footnote{The odd naming stems from the fact that these 
	are defined as \textit{properties} in the \code{ProcessBase} class, where 
	they are named \code{Restrict} and \code{PreProcess}. Properties in C\# 
	simply compile down to getter and setter methods with get\_ and set\_ 
	prefixed to the property name)}. These methods return delegates (otherwise 
	known as function pointers) to a preprocess function (such as a relabelling 
	function) and a restriction function, respectively. The base class versions 
	of \code{get\_PreProcess} and \code{get\_Restrict} return function pointers 
	to methods that do not alter or restrict any actions. Concrete process 
	classes can override these methods and return function pointers to other 
	methods, either methods in the process class itself, or methods in some 
	external assembly. 
	
	\begin{figure}
		\begin{center}\includegraphics[scale=0.7]{ex_rr.png}\end{center}
		\caption{Assembly structure of process  $RR \defeq (a \ccsdot b \ccsdot c \ccsdot 0)_{[x/a,y/b]} \backslash \{ c \}$}
		\label{fig:struct_restrictrelabel}
	\end{figure}

	The PLR can compile simple restrictions and relabellings directly into the 
	class where they are used. (Simple meaning that they only use constant 
	channel names, like $[a/b]$ or $\backslash\{c,d\}$, and do not require any 
	additional logic). Figure~\ref{fig:struct_restrictrelabel} shows the 
	assembly structure of a simple restricted and relabeled process, $RR \defeq 
	(a \ccsdot b \ccsdot c \ccsdot 0)_{[x/a,y/b]} \backslash \{ c \}$. The 
	restriction method is named \code{RestrictByName} and the relabeling method 
	is named \code{RelabelAction}. The base class methods 
	\code{get\_Restrict} and \code{get\_PreProcess} have then been overriden to 
	return function pointers to \code{RestrictByName} and \code{RelabelAction}, 
	respectively.

	\subsection{Variables and scope}
	
	Variable scope becomes a bit tricky due to the possibly many inner classes
	of a single process. A simple process like $in(x) \ccsdot (\overline{out}(x) 
	\ccsdot 0)\backslash\{out\}$ is actually two classes and the $in$ and $out$ 
	actions are performed in different methods. The variable $x$ still needs to 
	be passed on so that the inner process has access to the value that was 
	bound in the outer process. An additional complication is that the processes 
	main logic has to be defined in an overridden method, \code{RunProcess}, so 
	changing the parameters of that method is not an option. And finally, a 
	variable may be bound and the process may then become two or more parallel 
	processes, each of which must have access to the variable. Consider the 
	process $in(x) . (\overline{left}(x) . 0 \mid \overline{right}(x).0)$. Both 
	of the parallel processes use the variable $x$, but an important thing to 
	note is that they each have their own instance of it. Changing it in one 
	parallel process does not affect it in another. This may seem 
	counterintuitive, but consider if changing the variable in one process did 
	affect it in the other process, then we would have created a new method of 
	communication between processes, shared variables!
	
	The PLR handles variables by passing them as parameters to the inner 
	processes constructor. The inner process then assigns each of the 
	constructor parameters to member variables. The PLR variables are either 
	integers or strings, integers are passed by value and strings are immutable, 
	so there is no danger of two processes changing the same value in memory. At 
	the start of a process's \code{RunProcess} method, it defines local 
	variables with the same names as its member variables and assigns the member 
	variables to local variables. This was done for two reasons, it simplifies 
	working with variables in the method since all variable lookups are done on 
	local variables (keep in mind that new local variables might also be 
	defined), and it helps during debugging, the debugger will display the value 
	of local variables when the mouse cursor hovers over their names.
	
  A process may split into many processes and not all of them might need to 
  use all variables. As an optimization, the PLR examines the syntax tree 
  during compilation and only defines member variables and constructor 
  parameters in processes where it is possible that the variable will be used 
  in the process. For example, in the process
  
  \begin{center}$\mathrm{VAR} \defeq in(x) \ccsdot in(y)\ccsdot (\overline{out}(x) \ccsdot 0 \mid \overline{out}(y) \ccsdot 0)$\end{center}
  
	the main $VAR$ process will define two local variables, $x$ and $y$. The 
	first parallel process, $\overline{out}(x) \ccsdot 0$, will have a 
	constructor with only one parameter, $x$ and one member variable $x$, the 
	second parallel process, $\overline{out}(y) \ccsdot 0$, will only have $y$ 
	as a member variable and constructor parameter.

\section{Summary}

	The Process Language Runtime is implemented as a .NET library, which 
	contains both the abstract syntax tree used during compilation as well as 
	runtime classes used during execution. The syntax tree is the most important 
	part of the PLR, it is a rich datastructure where each node knows how to 
	compile itself and emit the correct bytecodes. Debugging support is also 
	provided by the PLR, although individual language parsers must provide the 
	PLR with information about line and column numbers for the process 
	constructs. Implementers of process algebras can extend the PLR by creating 
	additional syntax tree nodes, subscribing to compilation events and writing 
	their own runtime libraries. When a process language system is compiled then 
	individual processes become classes in .NET, actions become method calls and 
	restrictions and relabellings are implemented as methods, with function 
	pointers used as an abstraction to allow for calling methods in external 
	assemblies.
  
\chapter{Analysis and Optimization}\label{ch:analysis_and_optimization}

	In this chapter we look at the static analyses performed on the PLR syntax 
	tree before compilation, these include some classical dataflow analyses 
	as well as some analyses that are more specific to the process algebra 
	domain. The results of these analyses can be used to optimize the 
	compilation process, the optimizations are presented and explained, and 
	their implications for debugging support are discussed.
	
\section{Analysing process algebra}
	
	\subsection{Traditional data flow analysis}
	Many of the most useful static analyses that compilers perform are based on 
	\textit{data flow analysis}. For these types of analyses it is necessary to 
	build up a \textit{control flow graph} of the program being analysed. The 
	control flow graph shows which program points lead directly to which other 
	program points, in some cases the reverse control flow graph is needed, 
	which shows for a program point $p$ what its immediate predecessors are. The 
	following snippet of typical imperative code is an example:
	\begin{Exa}
	\label{ex:control_flow}
	\begin{verbatim}
	
                     1: y := 2
                     2: x := 1
                     3: while x < 6 do
                     4: 	  x := x + 1   
                        od
                     5: print x
	\end{verbatim}
	\end{Exa}
	
	In this example the program points are labelled from 1-5. Program points are 
	those points in the program where something happens, expression are 
	evaluated, variables are assigned, etc. The control flow for this example 
	would be (1,2), (2,3), (3,4), (3,5), (4,3) and (4,5). The analysis is then 
	typically performed by having each program point have an input set and an 
	output set, the input set represents the state of the program as the point 
	is reached and is based on the state of its predecessors, the output set 
	represents the state of the program after the program point has been 
	evaluated, and is based on the points input set with some modifications 
	based on what happened at the program point. 
	
	To give a concrete example suppose we have an analysis which is determining 
	for the code snippet in Example~\ref{ex:control_flow} which variables have 
	been assigned at each point in the program. For the program point labelled 1 
	(x := 1), its input set would be $\{y\}$ as $y$ is the only variable that 
	has been assigned at that point. Since program point 1 assigns to the 
	variable $x$ then its output set would be the union of its input set and 
	$\{x\}$, or $\{x,y\}$. The functions used to modify the input set to create 
	the output set are commonly called \textit{Kill} and \textit{Gen}, the 
	\textit{Kill} function removes items from the input set and the \textit{Gen} 
	function adds new items to the output set. This can be shown as (for program 
	point $p$):
	
	\begin{center}
	$p_{output} = p_{input}\ \cup\ Gen(p)\ \setminus\ Kill(p) $
	\end{center}
	
	To get the final result of the analysis these calculations must be repeated 
	for each point in the program until the input and output sets of each become 
	stable. The result is an approximation, either overapproximation (something 
	\textit{may} happen on the path to the program point) or an underapproximation
	(something \textit{must} have happened on the path to the program point).  To 
	calculate the input and output sets of each program points an iterative 
	worklist algorithm is used. There are many variations of these algorithms, 
	they keep track of which program points change and which other program points 
	must then be re-calculated. There is a lot more to data flow analysis than 
	explained here, for instance whether output sets of predecessors are combined 
	using the union or intersection operator, and what the initial content of the 
	input sets are, but we will not go into more detail on how data flow analysis 
	generally works here, this is meant only as a brief introduction before 
	explaining the analyses performed in the PLR. A more formal introduction to 
	the subject of data flow analysis can be found in \cite{program_analysis}.
	
	\subsection{Data flow analysis in process algebra}
	
	Two properties of process algebra make it different from imperative 	
	languages when it comes to data flow analysis.
	
	\begin{enumerate}
		\item There are no looping constructs. The reason that data flow analysis 
		on imperative languages needs iterative worklist algorithms is because of 
		looping. When looping is not a part of the language then the whole program 
		can be analyzed from top to bottom (or bottom to top) in one pass, each 
		program point only needs to be calculated once.
		
		\item There are branches, but they are never re-joined later in the 
		program. This implies that at every point in the program is is possible to 
		know exactly what path was taken to get to that point. Note that this only 
		holds for forward analysis. Backwards analysis can be seen as multiple 
		branches joining, and so in a backward analysis it is not possible to know 
		at program point $p$ from which branch a particular item in $p$'s input 
		set originates.
		
	\end{enumerate}

	These two properties might not hold for all process algebras in existence, 
	but they certainly hold for both CCS and KLAIM, as well as some other 
	prominent algebras such as CSP, and they do simplify the implementation of 
	these analyses for process algebra. Sections \ref{live_variables} and 
	\ref{reaching_definitions} discuss one backward and one forward data flow 
	analysis and how they were implemented in an extensible way for the PLR.
	
\section{Analyses}

\subsection{Live Variables}\label{live_variables}

  \textit{Live Variables Analysis} is a classic data flow analysis. Its purpose 
  is to identify at each program point which variables are \textit{live}, that 
  is which variables will be read later on in the program in the paths that 
  follow the program point in question. This is a backward analysis and is 
  traditionally used to identify assignments to variables that have no effect,
  for instance if the variable $x$ is assigned at program point $p$, but along 
  all paths that follow $p$ the variable is either never read, or assigned to 
  again before being read then the assignment at $p$ has no effect and can be 
  eliminated. The analysis is an over-approximation, we cannot safely eliminate 
  the assignment to $x$ if there \textit{may} be a path after $p$ where $x$ is
  read. This implies that the input set of $p$ will be the \textit{union} of the
  output sets of its pre-decessors (here the pre-decessors actually mean the 
  program points that follow $p$ as it is a backward analysis).
  
  In the analysis of the PLR syntax tree, a process is considered a program 
  point. Additionally, syntax nodes that represent a process constant being 
  defined are considered program points, as they may contain defining occurences
  of variables. For instance, in $ CS(x,y) \defeq a . 0$ we would consider 
  $CS(x,y)$ a program point, since it defines the variables $x$ and $y$, and it
  can be beneficial at that point to know whether the initial value of those 
  variables was ever used in the process. We have two helper functions, 
  \textit{assigned(x)} takes in an action and returns a set of the variables 
  assigned in that action. An action in this case can be the receiving of 
  values through a channel or sending values through a channel. The other 
  helper function, \textit{read(x)} takes in an expression and returns all 
  variables evaluated in that expression, or takes in an action and returns
  all variables evaluated in that action. Figure~\ref{fig:read_assigned} 
  shows some examples of the use of \textit{read} and \textit{assigned}, 
  Figure~\ref{fig:killgen_livevariables} shows how the \textit{Kill} and
  \textit{Gen} functions are defined for each of the process types in the
  PLR syntax tree.
  
	\begin{figure}[h!]
	\label{fig:read_assigned}
	\caption{Examples of the \textit{read} and \textit{assigned} functions}
	\begin{ARRAY}{r l l}
  assigned(\mathrm{channel}(x,y)) & = & \{x,y\} \vspace{5pt}\\
  read(\overline{(\mathrm{channel}(z+1,a-b)}) & = & \{z,a,b\} \vspace{5pt}\\
  read(x+1/z) & = & \{x,z\} \\
	\end{ARRAY}
	\end{figure}
  
  \begin{figure}[h!]
  \label{fig:killgen_livevariables}
  \caption{\textit{Kill} and \textit{Gen} functions for Live Variables}
  \begin{ARRAY}{r l l}
  		
  		Kill(a.P) & = & assigned(a) \\
  		Gen(a.P) & =& read(a) \vspace{7pt}\\
      
      Kill(P \mid Q) & = & \emptyset \\
      Gen(P \mid Q) & = & \emptyset \vspace{7pt}\\
   
      Kill(P + Q) & = & \emptyset \\
      Gen(P + Q) & = & \emptyset \vspace{7pt}\\
  
      Kill(0) & = & \emptyset \\
      Gen(0) & = & \emptyset \vspace{7pt}\\
      
      Kill(\mathrm{if}\ bexp\ \mathrm{then}\ P\ \mathrm{else}\ Q) & = & \emptyset\\
      Gen(\mathrm{if}\ bexp\ \mathrm{then}\ P\ \mathrm{else}\ Q) & = & read(bexp)\vspace{7pt}\\

      (\mathrm{process\ definitions}) & &\\
      Kill(K(x,y,z) \defeq) & = & \{x,y,z\} \\
      Gen(K(x,y,z) \defeq) & = & \emptyset \vspace{7pt}\\

      (\mathrm{process\ invocations}) & &\\
      Kill(K(exp_1,...,exp_n)) & = & \emptyset \\
      Gen(K(exp_1,...,exp_n)) & = & read(exp_1) \cup ... \cup\ read(exp_n) \\

  
  \end{ARRAY} 
  \end{figure}
  
  The actual implementation of the \textit{assigned()} and \textit{read()} 
  functions is done with two properties that are present on all syntax nodes
  that derive from either \textsf{Process} or \texts{Action}. These properties
  are \textsf{AssignedVariables} and \textsf{ReadVariables} and return the
  assigned and read variables of a process or action. Having these as properties
  of the syntax nodes instead of as part of the implementation of the analysis 
  allows for greater extensibility. Languages implemented using the PLR that 
  have their own custom syntax nodes simply need to override these properties
  and can then use the analysis without further modifications. This is  
  the case in the KLAIM implementation discussed in Chapter~\ref{ch:klaim}, 
  it has its own custom actions and a custom process type and they simply 
  implement these properties.
  
  Constructing the flow graph is trivial, since there are no loops or joining
  branches. In fact, a special flow graph is not constructed, instead the PLR
  syntax tree is used directly. To make it simpler, each \textsf{Process} node
  is required to implement a \textsf{FlowsTo} property which is a list of all
  the processes it can flow to. For an \texttt{if-else} process these are the 
  if and else branches, for parallel composition these are the composed 
  processes, etc. In addition each node in the syntax tree has a \textsf{Parent} 
  property which references its parent and can be used as a reverse flow graph.
  Again, having these properties directly on the syntax tree is useful so
  that additional process types can be added without having to modify the
  analysis code. The input and output sets are stored in a a \textsf{Tag} property
  which is also on each syntax node, this is a generic object reference which
  analyses may use temporarily to store data associated with each node.
  

\subsection{Reaching Definitions}\label{reaching_definitions}

\subsection{Constant Expressions}

\subsection{Nil Process Elimination}

\subsection{Unmatched Channels}

\subsection{Unused Processes}


	
\chapter{CCS Implementation}

In this chapter we look at the Calculus of Communicating System process language, its history, formal syntax and semantics, and its implementation with the PLR as the back end compiler. CCS was the first language implemented using the PLR, and it was implemented at the same time as the PLR was being designed. For that reason the PLR contains all the main constructs of CCS so that the CCS specific parts of the implementation are fairly small and mostly have to do with the front end, that is the lexer and parser. A description of a another language, one that extends the PLR further, is found in Chapter~\ref{ch:klaim}.

\section{History}

\section{Formal syntax and semantics}\label{sec:ccs_syntax}

The collection $P$ of CCS expressions is given by the following grammar

\begin{math}
	P,Q ::= K \bigm\vert \alpha.P \bigm\vert \sum_{i \in I} P_i \bigm\vert P \mid Q \bigm\vert P[f] \bigm\vert P\ \backslash\ L
\end{math}

where

\begin{itemize}
	\item $K$ is a process name in $\mathcal{K}$,
	\item $\alpha$ is an action in \textsf{Act},
	\item $I$ is a possibly infinite index set,
	\item $f$ : \textsf{Act} $\rightarrow$ \textsf{Act} is a \textit{relabelling function} satisfying the constraints
	
	\begin{tabular{r c l}
	f(\tau) & = & \tau, \\
	f(\overline{a})& = & \overline{f(a)} for each label a, \\
	\end {tabular}
	\item $L$ is a set of labels from $\mathcal{L}$
	\begin{math}
	
	\end{math}
	
\end{itemize}

\section{Value passing syntax and semantics}\label{ccs_value_syntax}

\section{Implementation}

	The CCS implementation was written in C\# using Visual Studio 2008 as the 
	development environment. Since the PLR includes abstract syntax tree nodes 
	for all constructs of CCS, no additional nodes were created specifically for 
	the CCS implementation. There are only a handful of classes used, below is a 
	short summary of each one.

	\begin{itemize}
	
	\item \textsf{Program} is the entry point into the compiler. It contains a 
	\textsf{Main} method that parses command line parameters, validates 
	parameters such as input and output file names and then creates a 
	\textsf{Parser} object. It then calls the parser's \textsf{Parse} method, 
	receives a PLR \textsf{ProcessSystem} node and calls \textsf{Compile} on it. 
	
	\item \textsf{Scanner} is the lexer class whose responsibility it is to tokenize a CCS source code file into valid CCS terminals. Listing~\ref{lst:ccsterminals} shows the more complicated terminals of CCS defined by regular expressions. The simpler terminals, who are just string constants, are given directly in quotes in the parser definition.
	
  \item \textsf{Parser} is a recursive-descent parser for CCS. It uses the 
  tokens generated by \textsf{Scanner} and applies a number of 
  \textsf{productions} to recognize the language. The Extended Backus-Naur 
  Form (EBNF) of the productions is given in Listing~\ref{lst:ccs_ebnf}. The 
  parser constructs a PLR abstract syntax tree as it parses, if the parsing is 
  without errors the syntax tree can then be compiled. Both the 
  \textsf{Parser} and \textsf{Scanner} are generated by the parser generator 
  Coco/R \cite{cocor}. It takes as input a EBNF specification of the language 
  interspersed with C\# source code and from that it generates the 
  \textsf{Scanner} and \textsf{Parser} classes. 
	
	\item \textsf{ParserTest} contains unit tests for the parser and scanner, to 
	be run with the NUnit unit testing framework.
	
	\end{itemize}
	\begin{figure}
\lstset{showtabs=false,showspaces=false,showstringspaces=false}
\begin{lstlisting}[caption=Terminals of CCS scanner,label=lst:ccsterminals,frame=trbl,basicstyle=\scriptsize\ttfamily,showtabs=false,showspaces=false]
  PROCNAME   = [A-Z][A-Za-z0-9]*
  LCASEIDENT = [a-z][A-Za-z0-9]*
  CLASSNAME  = [A-Z][A-Za-z0-9]*(\\.[A-Z][A-Za-z0-9]*)*
  OUTACTION  = _[a-z][A-Za-z0-9]*_
  METHOD     = :[a-zA-Z][A-Za-z0-9]*
  NUMBER     = [0-9]+
  STRING     = "[^"]*"'
  \end{lstlisting}
	\end{figure}
	
	\begin{figure}
\lstset{showtabs=false,showspaces=false,showstringspaces=false}
\begin{lstlisting}[caption=EBNF Productions of CCS parser,frame=trbl,label=lst:ccs_ebnf,basicstyle=\scriptsize\ttfamily,showtabs=false,showspaces=false]
CCS = { ClassImport } ProcessDefinition { ProcessDefinition } 

ClassImport = "use" CLASSNAME 

ProcessDefinition = 
  PROCNAME [ "(" LCASEIDENT {"," LCASEIDENT } ")" ] "=" Process 

Process = NonDeterministicChoice

NonDeterministicChoice = ParallelComposition { "+" ParallelComposition }

ParallelComposition = ActionPrefix { "|" ActionPrefix } .

ActionPrefix =
  { Action "." }
  (
    "(" Process ")"
    | "0"
    | ProcessConstantInvoke
    | BranchProcess
  )
  [ Relabelling ]
  [ Restriction ]

BranchProcess = "if" Expression "then" Process "else" Process .

ProcessConstantInvoke = 
  PROCNAME [ "(" ArithmeticExpression {"," ArithmeticExpression } ")" ]

Action =
  LCASEIDENT [ "(" LCASEIDENT { "," LCASEIDENT } ")" ]
  | OUTACTION [ "(" ArithmeticExpression { "," ArithmeticExpression } ")" ]
  | METHOD "(" [ CallParam { "," CallParam } ")"

CallParam = ArithmeticExpression | STRING

Relabelling =
  "[" METHOD "]"
  | "[" LCASEIDENT "/" LCASEIDENT { "," LCASEIDENT "/" LCASEIDENT } "]"

Restriction = 
  "\" 
  (
    LCASEIDENT 
    | "{" LCASEIDENT {"," LCASEIDENT } "}"
    | METHOD
  )        

Expression = OrTerm { "or" OrTerm }

OrTerm = AndTerm { "and" AndTerm }

AndTerm = RelationalTerm { "xor" RelationalTerm }

RelationalTerm = 
  ArithmeticExpression [ ("=="|"!="|">"|">="|"<"|"<=") ArithmeticExpression ]
		
ArithmeticExpression = PlusMinusTerm { ("+" | "-") PlusMinusTerm }

PlusMinusTerm = UnaryMinusTerm { ("*"|"/"|"\%") UnaryMinusTerm }

UnaryMinusTerm =
  [ "-" ]                                   
  (
    "(" ArithmeticExpression ")"
    | NUMBER 
    | "0"
    | "true"
    | "false"
    | LCASEIDENT
  )
	\end{lstlisting}
	\end{figure}
	
	Section~\ref{sec:ccs_syntax} shows the formal syntax for CCS, it however does not account for integrating with the .NET environment to allow arbitrary method calls to be made as actions and .NET methods to be used as relabelling functions and/or restriction functions. The complete allowed syntax can be seen in Listing~\ref{lst:ccs_ebnf}, but to quickly summarize the changes from formal CCS, they are as follows:
	
	\begin{itemize}
		
		\item A CCS source code file can start with one or more \texttt{use} 
		statements, which consist of the token \texttt{use} followed by the fully 
		qualified name of a .NET class. This class can be in the .NET core 
		library, \texttt{mscorlib}, or in any arbitrary .NET assembly. During 
		compilation the filenames of assemblies containing classes used in the 
		source code must be passed to the compiler so that it knows where to look 
		for classnames found in \texttt{use} statements.
	
		\item Actions can be calls to .NET methods in addition to synchronization 
		and value passing on channels. A .NET method call consists of a colon 
		followed by the method name and parantheses around expressions passed as 
		parameters to the method, e.g. \texttt{:Power(2, 3*2)}. At compile time 
		the PLR resolves which class the method belongs to by looking at the 
		classes imported with \texttt{use} statements and inspecting their 
		methods. If more than one imported class has a candidate method an 
		exception is thrown.
		
		\item Relabelling functions can be specified as .NET methods in addition 
		to be constant replacements. To use a .NET method for relabelling its name 
		prefixed with a colon is put inside the square brackets that usually 
		define relabelling functions in CCS, e.g. \texttt{[:MyRelabelMethod]}. The 
		method is resolved to an imported class at compile time and is required to 
		be a method that takes a single parameter, an instance of the 
		\textsf{IAction} interface from the PLR runtime library.
		
		\item Restrctions functions can be specified as .NET methods in much the 
		same way as relabelling functions and must resolve to a method that takes 
		a \textsf{IAction} instance as a parameter. An example of a process which 
		uses a .NET method for restriction could be \texttt{(a . 0) \ :MyRestrict}.
	\end{itemize}
\chapter{KLAIM Implementation}\label{ch:klaim}s

\section{First Section}

\chapter{Interactive Process Viewer}

	To fully understand the issues involved in creating the Process Language 
	Runtime, a little background knowledge is required. First, Process 
	languages are explained, their history, common properties and practical 
	applications. Secondly, the .NET framework is presented and its technology
	explained.

\section{Process Languages}

\chapter{Integrated Development Environment}\label{ch:ide}

	One of the biggest differences between academic programming languages and 
	industrial programming languages is the level of tool support. Academic 
	languages traditionally are edited in text editors and compiled with command 
	line tools, while industrial languages usually have integrated development 
	environments (IDE's). These environments offer a wide range of features such 
	as syntax highlighting, instant visual warnings about syntax errors, 
	background compilation, automatic listing of available methods and variables 
	(IntelliSense), built in debuggers and refactoring. One of the goals of this 
	project was to explore how well the CCS language could be integrated into 
	one of these environments and how it could benefit from the features they 
	have to offer. This chapter presents the results of this exploration. 
	
\section{Choice of Integrated Development Environment}
	
	When choosing which IDE would be most suitable for CCS two environments 
	stood out, Microsoft Visual Studio 2008 and SharpDevelop. Eclipse 
	was briefly considered as well since it provides good plugin support, but 
	was dismissed since it is built on Java and since this project is focused on 
	integrating with the .NET framework it seemed natural to go with a .NET 
	development environment. Below the two candidate environments are described 
	and reasons given for choosing one of them.
	
	\subsection{Microsoft Visual Studio 2008}
	Microsoft's Visual Studio is the most popular environment for .NET 
	development. Visual Studio versions are released alongside new versions of 
	the .NET framework itself and each new version takes full advantage of and 
	supports all the new features in the .NET framework. The latest version as 
	of this writing is Visual Studio 2008 which supports the .NET framework 3.5. 
	Visual Studio has an extensive extensibility API based on COM technology, an 
	older technology for interaction between programs written in different 
	programming languages. Languages are integrated by creating \textit{language 
	services}, the Visual Studio program itself is simply a host for these 
	services. The languages that come with the .NET Framework, C\# and Visual 
	Basic.NET have their own language services that do not have any special 
	access to Visual Studio, this implies that a language service for a new 
	language can be made to offer all the same features as those supported by 
	the built in languages. The downside of Visual Studio is that its 
	extensibility API is fairly complicated and hard to work with. Another 
	drawback is that even though there exist a fair number of samples for 
	language services, the professional level services for languages like C\# 
	and Visual Basic.NET are not available as open source so it is not possible 
	to look at them for inspiration.
	
	\subsection{SharpDevelop}
	The second candidate for a CCS development environment was SharpDevelop, an 
	open source IDE written entirely in C\#. SharpDevelop is very similar to 
	Visual Studio in look and feel, and offers many of the same features. Its 
	extensibility API is entirely in .NET and is in many ways cleaner and 
	clearer than Visual Studio's API. It also has the benefit of being open 
	source software, so it is easy to get a clearer picture of its architecture, 
	and view the source for other language services, even the built in ones for 
	C\# and Visual Basic.NET. The architecture of SharpDevelop itself has even 
	been the subject of a book, \cite{sharpdevelop}. The drawbacks are that its 
	debugger is inferior to Visual Studio's, the application itself is slower, 
	and it is not as well known as Visual Studio. It also does not offer all the 
	same features as Visual Studio, a notable feature that is missing is the 
	ability to highlight syntax errors as the user types in code.
	
	\subsection{Chosen environment}
	After researching both environments, Visual Studio 2008 was chosen as the one
	to implement CCS's language service in. This was based primarily on the fact
	that Visual Studio is the IDE of choice for most .NET developers, it is 
	fast, offers great debugging support and real time syntax checking. While 
	the standard versions of Visual Studio are not free, it is possible to 
	download only the Visual Studio shell and distribute it for free. The shell 
	is the Visual Studio program itself without any language services. This 
	makes it possible to offer the CCS development environment free of charge to 
	anyone who wishes to use it. 
	
\section{Building a language service}
	
	\subsection{Goal}
	The goal of this integration was to be able to use Visual Studio to manage 
	all aspects of working with the CCS language. To achieve that the following 
	features needed to be implemented:
	
	\begin{enumerate}
		\item \textbf{CCS Projects}. The ability to create new projects 
		specifically for CCS applications.

		\item \textbf{Syntax highlighting}. To have different tokens of the 
		language colored differently so that it is easier to see and understand 
		the structure of the code.
		
		\item \textbf{Real time syntax checking}. Display visual warnings about 
		incorrect syntax in the code as it is being written. 
	
		\item \textbf{IntelliSense}. Allow the developer to press a keyboard 
		shortcut and get a list of all available channel names, keywords, variable 
		names and process names in the current application, and insert them at the 
		current location in code. 
		
		\item \textbf{Match braces}. When working with large expressions it 
		can be hard to see which braces (parentheses, curly braces, angle 
		brackets) match, and missing parentheses are a common syntax error. Visual 
		Studio can highlight the matching braces automatically, if the language
		service provides it with the necessary information about which braces 
		match each other.
		
		\item \textbf{Build support}. To be able to compile the code being written 
		from within Visual Studio, using its \textit{Build} menu items and 
		commands. A part of that is being able to use Visual Studio's built in 
		mechanism to search for and add references to other .NET assemblies that 
		the application uses, and pass those references to the compiler at compile 
		time.
		
		\item \textbf{Debugger support}. Launching the application after it had 
		been built and attaching the Visual Studio debugger to the running 
		executable. Also the ability to set breakpoints and step through the code 
		as it is executing.
		
	\end{enumerate}
	
	Visual Studio offers the front-end for all these features, that is the 
	graphical user interface to display them and the infrastructure that calls 
	into the language service's code to get the data necessary for the features 
	to work. However, Visual Studio of course has no knowledge of specific 
	languages and so it is the responsibility of each language service to 
	provide the back-end, the code that understands the language, its tokens, 
	its syntax, which items to display in IntelliSense and so on. The 
	implementation of these features is described in the following sections.
	
	\subsection{Visual Studio API}	
	The COM extensibility API for Visual Studio is fairly complicated and 
	unfriendly to use. It is also poorly documented. Fortunately Microsoft has 
	recently released a framework called the Managed Package Framework, or MPF 
	for short. This comes as part of the Visual Studio SDK (Software Development 
	Kit) and is a collection of .NET classes that wrap a lot of the underlying 
	COM interface, making it easier to work with in .NET languages. 
	The MPF classes implement much of the tedious boilerplate code which is 
	necessary and common to all language services. Parts of the Managed Package 
	Framework are released only as source code and are meant to be included 
	directly in language service projects when they are being built. In the 
	source code repository for this project, these files have been marked 
	specifically with a header stating that they are supplied by Microsoft to 
	avoid confusion about which code is original work and which code is 
	borrowed. 

	\subsection{CCS Projects}
	To allow the user to create a new project of type CCS project four main 
	things needed to be implemented.
	
	\begin{enumerate}
		\item A class named \textsf{CCSProjectPackage}, this class inherits from 
		a \textsf{ProjectPackage} that is provided in the Managed Package 
		Framework. It provides information about the project file ending, the 
		project name and a path to the project templates described in item 4 in 
		this list. It doesn't contain any real code, it only provides the 
		necessary properties for Visual Studio to recognize that a new type of 
		project has been registered. To ensure uniqueness of the package it has a 
		globally unique identifier (GUID).
		
		\item A class named \textsf{CCSProjectFactory}, this class inherits from 
		a \textsf{ProjectFactory} class from the Managed Package Framework. It 
		also contains a globally unique identifier and overrides only one method, 
		\textsf{CreateProject()} which returns a \textsf{ProjectNode} instance.
		
		\item A class named \textsf{CCSProjectNode}. This class represents the 
		project once it has been created, it inherits from \textsf{ProjectNode}, 
		again from Managed Package Framework. In this class it is possible to 
		override a lot of behavior, such as what happens when build dependencies 
		are added to the project, which items can be deleted from the project, how 
		to clean the project and many more. This implementation did not require a 
		lot of overrides, since each project only contains one source file so most 
		project possibilities are simply not enabled.
		
		\item Templates for the project needed to be created. The main template is 
		for the project file. The project file defines which items are included 
		and which MSBuild targets (see Section~\ref{msbuild}) should be used to
		build the project. Another template is for a default CCS source file that 
		is included in every project, and contains some simple sample code to get 
		people started. Finally there is a file named \textit{CCS 
		Project.vstemplate}, this contains some metadata about the templates and 
		is the file used by Visual Studio to determine which items to show when 
		new projects are created.
	
	\end{enumerate}
	
	In addition to these items that needed to be implemented, a lot of source 
	code from the Managed Package Framework is necessary to build the project 
	package successfully. These are standard implementations of a number of 
	interfaces Visual Studio requires, they can be overridden for more complex 
	projects than the CCS projects. It was surprising how much source code is 
	needed to do a relatively simple thing like creating a new project type.
	
	\subsection{Syntax Highlighting}
	Syntax highlighting is when different tokens in a language are given 
	different color to help differentiate them when looking at code. This
	is very helpful when looking at code to quickly sense the structure and
	identify problems, and has been a standard feature of development 
	environments as well as most advanced text editors for many years. In many
	common text editors this is simply implemented as lists of tokens and 
	colors for them. The Managed Package Framework however requires that the 
	language service provides an implementation of an interface named 
	\textsf{IScanner}, shown in Figure~\ref{fig:iscanner}. 

	\begin{figure}
	\begin{csharp}
  public interface IScanner {
    bool ScanTokenAndProvideInfoAboutIt(
    		TokenInfo tokenInfo, ref int state);
    void SetSource(string source, int offset);
  }	
\end{csharp}
	\caption{IScanner interface}
	\label{fig:iscanner}
\end{figure}

	The \textsf{SetSource} method of the interface is called by Visual Studio 
	and provides the \textsf{IScanner} with one line of source code at a time, 
	Visual Studio then repeatedly calls \textsf{ScanTokenAndProvideInfoAboutIt} 
	to get information about each of the tokens in that line. This is done on a 
	line-by-line basis so that only lines that change need to be re-colored, as 
	it is a relatively expensive operation. For this project the 
	\textsf{IScanner} interface was implemented using a lexer class, how that 
	class was generated is described further in 
	Section~\ref{sec:syntax_checking}. The tokens of the language were divided 
	up into eight distinct color classes and colored as follows:
	
	\begin{itemize}
		\item Process constants - Greenblue
		\item Output actions on channels	- Dark gray
		\item Method calls - Magenta
		\item Comments - Dark green
		\item Strings and class names - Maroon
		\item Keywords - Blue
		\item Numbers - Red
		\item All other tokens - Black
	\end{itemize}

	An example of the syntax highlighting can be seen in 
	Figure~\ref{fig:syntaxcheck}.
	
	\subsection{Real time syntax checking}\label{sec:syntax_checking}
	A very useful feature of Visual Studio is its ability to show the user 
	errors in their code in real time, as they are typing. These errors (or 
	warnings) are shown both as text error messages in an error message window, 
	as well as red curvy lines under the places in code where the syntax errors 
	occurs. This makes it extremely easy to look at a page of code and determine 
	whether it is syntactically correct. Figure~\ref{fig:syntaxcheck} shows an 
	example of this feature in action.
	
	\begin{figure}[h!]
		\centering
		\includegraphics[scale=0.6]{syntaxcheck.jpg}
		\caption{Syntax checking for CCS in Visual Studio}
		\label{fig:syntaxcheck}
	\end{figure}

	The component in Visual Studio that is responsible for syntax highlighting 
	and syntax checking is named Babel \cite{babel}. As part of the Managed 
	Package Framework there is a collection of classes to wrap this component, 
	these are called the Managed Babel System. With the Managed Babel System 
	come two programs called MPLex.exe and MPPG.exe, these acronyms stand for 
	\textit{Managed Package Lex} and \textit{Managed Package Parser Generator}. 
	These are .NET implementations of the well known parser generator tools Lex 
	and YACC and derive directly from the \textit{Garden Point Parser Generator} 
	\cite{gppg} developed at the Queensland University of Technology. These 
	tools take as input a \textsf{lexer.lex} file and \textsf{parser.y} file. 
	The lexer file defines the tokens of the language with regular expressions, 
	and the parser file describes the syntax of the language in extended 
	Backus-Naur form, or EBNF. From these input files MPLex.exe and MPPG.exe 
	generate C\# code for a lexer and a parser for the language. A more detailed 
	explanation of Lex and YACC-like tools is outside the scope of this paper 
	but a useful book on the subject is \cite{lexyacc}. 
	
	Once the generated lexer and parser have been built, Visual Studio is 
	responsible for calling them repeatedly in the background while the user is 
	typing code. Visual Studio calls a method named 
	\textsf{ParseSource(ParseRequest req)} which is a method of the 
	\textsf{CCSLanguage} class. That class inherits from 
	\textsf{BabelLanguageService} and overrides its \textsf{ParseSource} 
	method. Inside this method the parser and lexer are instantiated and parse 
	the source code which is a part of the \textsf{ParseRequest} instance passed 
	to the method. The parser then logs every error it encounters, with file 
	name and line numbers, and Visual Studio is responsible for displaying these 
	errors to the user.

	As we saw in Chapter~\ref{ch:ccs_implementation} the parser used by the CCS 
	compiler itself was written using the Coco/R parser generator. It would have 
	been preferable to re-use that parser directly instead of defining a new 
	parser for the same input language. While it would have been possible, the 
	fact is that MPLex.exe and MPPG.exe are optimized for generating parsers 
	that work well with the Managed Babel System, and the Managed Babel 
	infrastructure expects parsers and lexers that conform to a certain 
	interface. For that reason MPLex and MPPG were used to create a new parser 
	and lexer instead of re-using the existing ones. The drawback to this is of 
	course that two implementations of the same language need to be maintained 
	and kept in sync. However, the input language for both these parser 
	generators is based on EBNF syntax and so it is fairly trivial to port from 
	one to the other. In hindsight the best approach would have been to use MPPG 
	and MPLex for the CCS compiler as well as the language service.
	
	\subsection{IntelliSense}
	IntelliSense (or automatic word completion) is one of the most useful 
	features of Visual Studio. When the programmer presses the keyboard 
	combination CTRL+SPACE at some point in the source code, the environment 
	shows a list of items that the programmer might wish to insert at that 
	point, and a description of them. This includes (in CCS's case) channel 
	names, process names, imported .NET method names and CCS keywords. If the 
	programmer invokes IntelliSense once the caret is positioned after a half 
	completed word, such as \textit{cof} and there is only one candidate that 
	starts with \textit{cof}, namely the channel \textit{coffee}, then the word 
	is completed automatically and the list of possibilities is not even shown.
	
	This behaviour is implemented through two main classes, 
	\textsf{AuthoringScope} and \textsf{Resolver}. \textsf{AuthoringScope} is 
	the class that is returned from the \textsf{ParseSource} method we learned 
	about in Section~\ref{sec:syntax_checking}, part of its interface is the 
	method \textsf{GetDeclarations}. This method is called when the user has 
	pressed CTRL+SPACE and it in turns calls a method named 
	\textsf{FindCompletions} on the \textsf{Resolver} class, which returns a 
	list of all the items to display to the user, along with descriptions and 
	icons. 
	
	To create the list of items the resolver uses the generated scanner class we 
	discussed previously and scans all tokens in the source file to find 
	process and channel names. For every channel name it adds an item for the 
	output action and input action on that channel, e.g. both \texttt{coffee} 
	and \texttt{\_coffee\_}. The resolver also adds all the language keywords 
	(if,then,use etc.) to the list so they can be completed automatically, and 
	to provide a reference for the user about what is possible to do. Finally 
	the resolver tries to find all methods that the application could call, and 
	add them along with descriptions that include parameter names and types. To 
	do this the resolver tries to look up all classes that are imported in the 
	source code using the \texttt{use} keyword, as well as look for methods in 
	the class \textsf{PLR.Runtime.BuiltIns} which contains methods that can be 
	called without a \texttt{use} statement. The method names and parameters are 
	found through reflection. Since the PLR only supports static methods and 
	only strings and integers as parameters, the resolver only considers methods 
	that fulfill that criteria. Once this has been done the list of names, 
	keywords and methods is returned so that Visual Studio can display them to 
	the user. Figure~\ref{fig:intellisense} shows an example of this feature in 
	action. Note how different types of items have different icons to 
	distinguish them from each other.

	\begin{figure}[h!]
		\centering
		\includegraphics[scale=0.5]{intellisense.png}
		\caption{IntelliSense for CCS in Visual Studio}
		\label{fig:intellisense}
	\end{figure}
	
	\subsection{Match braces}
	
	This feature was fairly trivial to implement as most of it is provided by 
	the Managed Package Framework. To get it working the generated parser needs 
	to store information about every matching parentheses pair while it is 
	parsing the source. This is simply done by calling a \textsf{Match} method 
	in the parser definition. Figure~\ref{fig:matchbrace} shows how the brace 
	matching for expressions is achieved in the parser definition.

\begin{figure}
\begin{codeblock}	
UnaryMinusTerm
    : '-' UnaryMinusTerm
    | NUMBER
    | '(' Expr ')'  { Match(@1, @3); }
    | LCASEIDENT
    | KWTRUE
    | MethodCall
    | KWFALSE
    ;
\end{codeblock}
\caption{Matching braces for expressions}
\label{fig:matchbrace}
\end{figure}

	The macros @1 and @3 are converted when the parser is generated and mean 
	that the first and third token on the lines match. The only additional thing 
	needed to get the feature working was to check in the \textsf{ParseSource} 
	method if the reason for the parsing was to highlight braces, and if so then 
	use the braces found by the parser earlier and call a \textsf{MatchPair} 
	method on the \textsf{ParseRequest} object that was passed to the method. 
	Figure~\ref{fig:matchbraces} shows how this feature is useful when working 
	with large expressions.
	
	\begin{figure}[h!]
		\centering
		\includegraphics[scale=0.75]{matchbraces.png}
		\caption{Brace matching for CCS in Visual Studio}
		\label{fig:matchbraces}
	\end{figure}
		
	
	\subsection{Build support}\label{msbuild}
	
	Visual Studio has menu items and keyboard shortcuts for a \textit{Build} 
	command. This command is used to build (compile) the source files of the 
	current project into an executable file. The underlying system that does the 
	actual building is named MSBuild, and it is a command line build tool, 
	similar to \textit{make} which is commonly used on Unix and Linux platforms, 
	and \textit{NAnt}, an popular open source build tool for the .NET framework. 
	The input files for MSBuild are XML files that define what the tool should 
	do. The three most important elements in these files are \textit{targets}, 
	\textit{tasks} and \textit{properties}. 
	
	A \textit{target} is a particular action to take, for instance there can be 
	a \textit{build} target which builds an entire project and a \textit{clean} 
	target which deletes all intermediate files. Targets can depend on each 
	other, for example a \textit{rebuild} target can depend on the 
	\textit{clean} and \textit{build} targets, so that whenever \textit{rebuild} 
	is executed the tasks it depends on are automatically executed first. 
		
	\textit{Tasks} are the operations performed by the targets. Each task is 
	usually a single distinct action, for example one task might be to call a 
	compiler, another task might be to copy files. Tasks can take parameters, 
	for instance the names of the files to compile. Custom tasks can be written 
	in .NET to achieve any operation and integrate it into the build process.
	
	\textit{Properties} are essentially variables to use in the build process, 
	and are often passed as parameters to the tasks. A property might for 
	instance be named \textit{Debug} and have either the value \texttt{true} or 
	\texttt{false}. It could then be passed to a task as a parameter.
	
	Each language in Visual Studio has its own \textit{.targets} files which 
	defines all the targets and tasks specific to that language. Additionally 
	there is a common targets file, \textit{Microsoft.Common.targets} which all 
	languages use. These \textit{.targets} files are then referenced in the 
	project files for the languages. To get CCS working with the build system it 
	was necessary to create one custom task, to call the compiler. The task is 
	defined in a class named \textsf{CompileTask} which inherits from MSBuild's 
	\textsf{ToolTask} class. It has an \textsf{Execute} method which calls the 
	CCS compiler and logs all errors that the compiler emits, and four 
	properties that can be set, \textsf{Debug}, \textsf{InputFile}, 
	\textsf{OutputFile} and \textsf{References}. To make that task available in 
	Visual Studio the template CCS project file references a file called 
	\textsf{CCS.targets}. That file in its entirety is shown in 
	Figure~\ref{fig:ccstargets}.
	
	\begin{figure}
	\begin{xml}
<?xml version="1.0" encoding="utf-8" ?>
<Project DefaultTargets="Build" 
  xmlns="http://schemas.microsoft.com/developer/msbuild/2003">
    <UsingTask TaskName="CCS.BuildTasks.CompileTask" 
      AssemblyFile="$(CCS_PATH)CCS.BuildTasks.dll"/>
    <Target Name="CoreCompile">
        <CompileTask 
            Debug="$(DebugSymbols)" 
            OutputFile="@(IntermediateAssembly)" 
            InputFile="@(Compile)" 
            References="@(ReferencePath)"/>
    </Target>
    <Target Name="CreateManifestResourceNames"></Target>
    <Target Name="Build"></Target>
    <Target Name="Compile"></Target>
    <Import Project="$(MSBuildBinPath)\Microsoft.Common.targets" />
</Project>
\end{xml}
\caption{The CCS.targets file}
\label{fig:ccstargets},
\end{figure}
	
	All that is necessary to do in the \textit{CCS.targets} file is to create a 
	target named \textit{CoreCompile}, and in that target call the custom 
	compile task that was created for the CCS compiler. Empty implementations of 
	the targets \textit{CreateManifestResourceNames}, \textit{Build} and 
	\textit{Compile} are also provided, since otherwise these targets try to 
	perform actions that are not necessary for building CCS applications. The 
	\textit{Debug} parameter can be set within Visual Studio and the names of 
	the input and output files are determined by the name given to the project 
	when it is created. References to other .NET assemblies can be added in 
	Visual Studio and they will be passed to the compile task through the 
	\textit{References} parameter. Having the CCS project file reference the 
	CCS.targets file, and that file call the custom compile task is enough to 
	get full build support from within Visual Studio.
	
	\subsection{Debugger support}
	
	Once the build support described in Section~\ref{msbuild} is in place, 
	getting debugger support within Visual Studio is trivial. All that is needed 
	is to select the \textit{Debug} configuration inside Visual Studio, this 
	passes the paramater \textit{Debug=true} to the compile task and on to the 
	compiler. We already saw how the PLR supports emitting debugging symbols in 
	Section~\ref{debug_support}. Pressing the F5 key, or the play button, in 
	Visual Studio will then build the project, launch the executable if the 
	build is successful and attach the debugger. This works due to the fact that 
	the \textit{Microsoft.Common.targets} file defines a \textit{Run} target 
	that calls the \textit{CoreCompile} target (which \textit{CCS.targets} 
	defines) and then launches the debugger, this action is the same for all 
	languages, although the compile step itself may be different.
	
	The only additional thing done to make the debugging experience more user 
	friendly was to implement a method named \textsf{ValidateBreakpointLocation} 
	in the \textsf{CCSLanguage} class. When the user tries to set a breakpoint 
	on a particular line, this method is called and is responsible for looking 
	at the line and determining whether a breakpoint can be set there, and if 
	so, which part of the line should be highlighted. For CCS the valid 
	locations are actions, method calls, process invocations and expressions in 
	\texttt{if} statements. Using the generated \textsf{Scanner} class once 
	again, it is possible to find out which tokens are on a particular line and 
	return their locations if they are valid points for a breakpoint. 
	Figure~\ref{fig:breakpoint} shows how a breakpoint is highlighted. (Note 
	that this is only relevant for the highlighting done at compile time, at run 
	time each line is highlighted according to the sequence points in the actual 
	compiled file.)

	\begin{figure}[h!]
		\centering
		\includegraphics[scale=0.5]{breakpoint.png}
		\caption{Setting breakpoints in Visual Studio}
		\label{fig:breakpoint}
	\end{figure}
	
	\section{Summary}
	
	There is no doubt that the tool support for programming languages is a 
	factor in whether those languages become popular, how productive programmers 
	are when working with them and how enjoyable they are to work with. 
	Programming languages used in industry have for a long time now had great 
	tool support while academic languages often suffer from bad or incomplete 
	tools. As this chapter demonstrates, this need not be the case. Once a 
	programming language (or any tool that takes some text files as input) has 
	been developed, taking the extra time to develop a language service for it 
	can be very beneficial to the end users of that language, and is relatively 
	easy to do. One approach might even be to develop the language service 
	first, thus enabling the author to benefit from it himself while writing the 
	compiler. Visual Studio might not be the perfect environment for all 
	languages, but in any case it is a good idea for the language author to 
	research what environments are out there and evaluate if one of them could 
	be used to offer the end user of the language a better user experience.
\chapter{Final Considerations}

	The first and second chapters of the thesis presented the objectives and 
	goals of the project. They also introduced the subject matter, process 
	algebras, their purpose and structure and what they had in common. The .NET 
	framework was introduced, its history, relation to other virtual 
	machine environments and current status was discussed, as well as the 
	benefits of using virtual machines in general.
	
	The third chapter then documented the design and implementation of the 
	Process Language Runtime itself. The PLR self-compiling syntax tree was 
	explained in detail and reasons given for why that was an optimal design for 
	further extensibility. An overview of the PLR runtime classes was given, as 
	well as an insight into how debugging support is enabled in compiled process 
	language applications. The chapter ended by explaining in detail the 
	structure of a .NET assembly compiled from the PLR syntax tree.
	
	Chapter 4 presented a number of static analyses that are part of the PLR, 
	both classical data flow analyses and more specific analyses for process 
	algebra. How these analyses could be re-used for more than one process 
	algebra, even when they have different constructs, was also explained.
	
	Chapters 5 and 6 documented two separate implementations of process algebras 
	using the PLR, the languages were CCS and KLAIM. The syntax and semantics of 
	the languages was shown, as well as an overview of the most important 
	classes in their implementation. The chapter on KLAIM then explored how 
	additional features that are not included in the PLR could be added to new 
	languages, by using custom syntax tree nodes, PLR compilation events and 
	runtime libraries. Both chapters finished with sizable example systems to 
	give an idea of what a real application of these languages might be.
	
	In chapter seven we looked at a graphical user interface tool that allows
	the user to monitor and interact with running process language applications.
	The challenges faced when integrating with the PLR were discussed as well as
	how the program was kept general enough to work with any process language.
	
	The eighth chapter was about how the CCS language was integrated into a 
	professional development environment, Visual Studio 2008. Some alternative 
	development environments were presented and reasons given for why Visual 
	Studio was chosen. A number of useful features were implemented in the 
	Visual Studio integration, including CCS project support, syntax 
	highlighting, real-time syntax checking, IntelliSense, brace matching and 
	full integration with the debugger and build tool.
	
	Finally, in this chapter related work is discussed, including some 
	alternative implementations of CCS and KLAIM. We explore the potential 
	future work that could be done with the PLR, and end with some concluding 
	remarks.
	
	
\section{Related work}\label{sec:related_work}
	
	There are quite a few other projects that have implemented process algebras, 
	or languages inspired by process algebras, however most of these projects 
	take a different approach than that taken by the PLR. Below is a short 
	summary of some of the notable ones, especially those that focus on CCS or 
	KLAIM.
	
	\textit{KLAVA} \cite{klava} is an implementation of KLAIM. It is a Java 
	library which represents the KLAIM constructs as Java classes. KLAIM 
	applications can then be written in Java using the KLAVA library. The main 
	difference between KLAVA and the PLR implementation of KLAIM (hereafter 
	referred to as PLR KLAIM) is that in KLAVA it is not possible to write the 
	applications using the actual syntax of KLAIM. KLAVA is a much more feature 
	rich implementation of KLAIM than PLR KLAIM, it supports nodes running on 
	different machines, and additional constructs such as non blocking input 
	operations.
	
	\textit{X-Klaim} \cite{xklaim} (which stands for eXtended KLAIM) is another 
	implementation of KLAIM from the authors of KLAVA. It is at a higher level 
	of abstraction than KLAVA and has its own syntax, which is a superset of
	the original KLAIM syntax. X-Klaim code uses the KLAVA library as its 
	runtime library, the X-Klaim compiler compiles X-Klaim code down to Java 
	code that uses the KLAVA library. X-Klaim is similar to PLR KLAIM in that 
	both have a KLAIM syntax and both use a runtime library, the difference is 
	that PLR KLAIM directly emits bytecodes, whereas X-Klaim emits Java source 
	code, which means that X-Klaim code cannot be debugged using the original
	X-Klaim source files. X-Klaim is feature rich and supports many constructs
	not in the original KLAIM.
	
	\textit{AspectK} is an aspect oriented version of KLAIM. Originally 
	introduced in \cite{aspectk}, a full virtual machine for the language was 
	subsequently developed in \cite{giordano}. The KLAIM subset used in AspectK 
	is the same as that used in PLR KLAIM, AspectK then adds aspects on top of 
	that. The difference (aside from the aspect orientation) is that AspectK has 
	its own virtual machine, with bytecodes for common process algebra tasks 
	whereas PLR KLAIM uses an existing virtual machine. An advantage of having a 
	process algebra focused virtual machine is that generated code can be 
	smaller, since each bytecode instruction can perform more work. The PLR does 
	get a similar reduction in code size by generating bytecodes that call 
	methods defined in the PLR runtime library.
	
	\textit{JACK} \cite{jack} is a process algebra implementation written in 
	Java. It is similar to the PLR in that it aims to be a framework that can be
	used for implementing different types of process algebra, although its main
	focus is Communicating Sequential Processes (CSP). The difference is that 
	JACK, like KLAVA, represents algebra constructs as Java classes, and the  
	systems are written using Java code instead of the native syntax of the 
	process algebra being implemented. That is to say, it is a framework, but 
	not a compiler.
	
	CCS has at least two implementations, in \cite{build_ccs_interpreter} a 
	method is presented for how to build a sound CCS interpreter by following 
	the semantics of the language, and \cite{ccs_interpreter} shows how the 
	functional programming language Haskell can be used to build a CCS 
	interpreter with minimal amount of code. Both of these differ from the PLR 
	in that they are interpreters rather than compilers.
	
	None of the above related work aims to do exactly what the PLR attempts, 
	which is to build compilers for process algebras that operate on the 		
	algebra's standard syntax and integrate tightly with an existing virtual 
	machine. The PLR is also the only one of these that explores how existing 
	infrastructure can be used to add features to process algebras, such as 
	allowing CCS to call .NET methods written in another .NET language. A 
	further look at that topic might prove interesting, specifically how it 
	affects the original semantics of the algebra being implemented, what side 
	effects it might produce and what sort of interesting things could be
	modeled in this way.

\section{Further work}
\label{sec:further}

	The PLR could be improved and built upon in several ways. Here we will look 
	at some of them.
	
	\subsection{Additional process algebra constructs}
	
	Perhaps the most obvious improvement to the PLR would be to add support for 
	some of the constructs that are common in process algebras but are not 
	currently included in the PLR. This would make it even simpler to use the 
	PLR as the basis for implementing other process algebras. To get a sense of 
	which constructs would be most beneficial to add, we look shortly at two
	of the most prominent process algebras, CSP and $\pi$-calculus, and 
	what would be needed for them to run on the PLR.
	
	\textit{$\pi$-calculus} was introduced in \cite{Milner89acalculus} and is 
	described by its author as an extension of CCS. There are many variants of 
	$\pi$-calculus with additional features but the core calculus has two 
	noteworthy additions to CCS:
	
	\begin{enumerate}
		\item The \textit{match} construct $[a = b] P$ which compares the values 
		of $a$ and $b$ and behaves as $P$ if they are equal but otherwise turns 
		into the nil process. This is simply a more restricted version of the 
		\texttt{if-then-else} construct which is already implemented in the PLR, 
		the boolean condition is restricted to equality comparison and the 
		\texttt{else} branch is simply the nil process. 
		
		\item The generalization of channel and variable names. In $\pi$-calculus 
		both variable and channel names are seen simply as \textit{names}, and can 
		be passed along channels as data. For example a process $P$ could send the 
		channel name $y$ to process $Q$, which could then send or receive on that 
		channel.
		
	  \begin{align*}
			\mathrm{P} \defeq & \overline{x}y . y(j) . 0\\
			\mathrm{Q} \defeq & x(z) . \overline{z}3 . 0\\
		\end{align*} 
		
		Here we see that $P$ first outputs the name $y$ on channel $x$. Process 
		$Q$ receives the channel name on channel $x$, after which it substitutes
		$z$ with $y$ and becomes $\overline{y}3 \ccsdot 0$. It then sends the value
		3 on the received channel and $P$ accepts it and binds it to the name $j$.
		
		This could fairly easily be added to the PLR. At compilation time each 
		channel name being used could be checked to see whether it was defined as 
		a variable or not. If it had previously been defined as a variable then 
		the synchronization would happen on the channel whose name was stored in 
		the variable, if no variable with that name exists then the name of the 
		channel would be considered a constant. For example in the term $x(z) . 
		\overline{z}3$ we see that $z$ is bound in the first action, and so when 
		we process the $\overline{z}3$ action we know that we should use the name
		stored in $z$ as opposed to the literal name $z$. However in the term 
		$x(z) . \overline{y}3$ we see that $y$ has never been bound as a variable 
		and so when the $\overline{y}3$ action is processed the literal name $y$ 
		is used for the channel.
		
	\end{enumerate}
	
	\textit{CSP} is very similar to CCS. It can synchronize on channels, with or 
	without message passing, it uses action prefixing, restriction, choice and 
	parallel composition. Its definitions of choice and parallel composition are 
	a little bit more complex than those of CCS. It also distinguishes between 
	an inactive process, which is called $STOP$ and is equivalent to the nil 
	process in CCS, and a $SKIP$ process which signals that a process has 
	completed successfully. Finally, as its name \textit{Communicating 
	Sequential Processes} suggests, it offers sequential composition of 
	processes. We now look at how these differences might be implemented in the 
	PLR.
	
	\begin{enumerate}
		\item Implementing the $SKIP$ process is trivial since it does nothing. 
		The main issue is distinguishing it from the $STOP$ process, this can be 
		done by letting them output different text when they are invoked.
		
		\item In CSP a distinction is made between internal non deterministic 
		choice (written $P \sqcap Q$) and external non deterministic choice 
		(written $P\ \square\ Q$). External choice will synchronize with the first 
		event offered by  the environment, so in $a . STOP\ \square\ b . STOP$ it 
		depends on which of $a$ and $b$ is offered first. Internal non 
		deterministic choice however can refuse to participate in an event even 
		though no alternative is offered. The current implementation in the PLR is 
		equivalent to CSP's external choice, if an event (or channel 
		synchronization) is offered on one of the paths then it will be taken. If 
		two or more candidates are offered then the selection between them is made 
		randomly. To enable a simulation of internal choice it could be possible 
		to add some randomness to whether or not an offered channel 
		synchronization is accepted. When the scheduler is finding out which 
		synchronizations are possible it calls a \textsf{CanSyncWith} method on 
		each candidate action. It could be possible to make that call randomly 
		return true or false, which would make sure that a path in an internal 
		choice was not forced to be taken, even if it was the only candidate for 
		synchronization.
		
		\item CSP also handles parallel composition in a slightly different way 
		from CSS. It defines two versions of parallel composition. The first one, 
		which is sometimes called \textit{interleaving} is written as $P 
		\mid\mid\mid Q$. Here $P$ and $Q$ run in parallel and are independent of 
		each other. They can interact but are not forced to. This is equivalent to 
		the PLR's parallel composition construct. 
		
		The other type of parallel composition available in CSP is written as $P 
		\mid\mid_A Q$ or sometimes as $P\ |[\{A\}]|\ Q$. $P$ and $Q$ are said 
		to be \textit{interface parallel}. Here $A$ is a set of channel names that 
		$P$ and $Q$ must synchronize on, e.g. in $P\ |[\{a,b\}]|\ Q$ the 
		processes $P$ and $Q$ must synchronize with each others on channels $a$ 
		and $b$. If one of them has arrived at an $a$ or $b$ action it cannot 
		continue until the other one is ready to synchronize with it. This is 
		similar to parallel composition with restriction in CCS, e.g. $(P \mid Q) 
		\backslash\{a,b\}$. In that expression $P$ and $Q$ have to synchronize on 
		$a$ and $b$ because they are invisible from the outside and so for either 
		process the other process is the only candidate to synchronize with. It is 
		slightly different though because in CSP's version the events $a$ and $b$ 
		are not unobservable from the outside. This could still be implemented 
		much like the CCS expression shown above, the channels $a$ and $b$ would 
		be locally scoped to the $P\ |[\{a,b\}]|\ Q$ process so that no 
		external processes could participate in the synchronization, and the 
		channels would be shown as part of the trace of the system, which would 
		not normally be done in an expression like $(P \mid Q) \backslash\{a,b\}$
	
		\item Sequential composition is written as $P;Q$, it is a process that 
		behaves like $P$ until $P$ terminates and then behaves like $Q$.	This 
		would be easy to implement in the PLR, any finite process will end up as 
		the nil process (or the $STOP$ or $SKIP$ process in CSP), once the nil 
		process has been reached in $P$ a new instance of $Q$ could be 
		instantiated and started. 
	
	\end{enumerate}
	
	In addition to these features, variants of CSP sometimes contain additional 
	features such as interrupts and timeouts which we shall not go into here. 
	
	\subsection{Richer datatypes and expressions}
	
	The initial version of the PLR supports two types of variables, integers and 
	strings. For expressions it supports constants for integers, booleans and 
	strings, as well as simple arithmetic and relational operators for integers 
	and logical operators for booleans. One way to extend the PLR would be to 
	add more support for using other data types from .NET. An example would be 
	to allow passing of .NET objects through channels and calling instance 
	methods on those objects in the receiving process. This would require some 
	additional syntax nodes for the PLR tree, an expression node for 
	constructing a new object and a node for a method call on an object (as 
	opposed to a static method call which is already supported). Other useful 
	features to add might include support for floating point numbers, string 
	formatting and basic string expressions using the + operator.
	
	\subsection{Additional analysis}
	
	Another potential improvement would be to add additional analyses before 
	compilation. This is where the benefit of having a shared syntax tree for 
	multiple algebras becomes apparent, as many of the analyses could be re-used 
	for multiple process algebras (although probably not all of them). This 
	could include common compiler optimization techniques such as constant 
	propagation and Very Busy Expressions analysis, or analyses more directly 
	related to process algebra, such as finding channels that are never used
	and identifying processes that will always block.
	
	\subsection{Bi-similarity of processes}
	One of the interesting things that could be added, for CCS and maybe others, 
	would be an analysis to compare two processes and see if they are 
	\textit{behaviorally equivalent}, also known as \textit{bi-similar}. 
	B-similarity is a congruence, if processes $P$ and $Q$ are bi-similar then 
	it means that if $P$ is a component in a system then it can be replaced with 
	$Q$ and the system will continue to work in the same way, since $P$ and $Q$ 
	exhibit the same behavior. This can for instance be used to write a 
	specification as a simple process expression and then write an 
	implementation for that specification. If the specification and 
	implementation are bi-similar then the implementation is a correct 
	implementation of the specification. An example could be the specification
  \begin{align*}
			\mathrm{CoffeeMachine} \defeq & coin  \ccsdot \out{coffee} \ccsdot CoffeeMachine \\
	\end{align*} and the implementation
  \begin{align*}
			\mathrm{CoffeeMachineImpl} \defeq & ( \mathrm{CoinReceiver} \mid \mathrm{CoffeePourer} ) \backslash \{pour\} \\
			\mathrm{CoinReceiver} \defeq & coin \ccsdot \out{pour} \ccsdot \mathrm{CoinReceiver} \\
			\mathrm{CoffeePourer} \defeq & pour \ccsdot \out{coffee} \ccsdot \mathrm{CoffeePourer}
	\end{align*}	

	Here we see that according to the specification of \textsf{CoffeeMachine} 
	the observable events are an endless stream of $coin$ and $coffee$. This 
	however tells us nothing about how this machine is implemented. The second 
	process \textsf{CoffeeMachineImpl} is the implementation of this coffee 
	machine, it is composed of two components, a receiver for the coins and a 
	component that pours the coffee. They communicate between themselves on the 
	$pour$ channel. Since that channel is hidden (or restricted) it is not 
	observable from the outside, what is observable from the outside is again an 
	endless stream of $coin$ and $coffeee$. In this trivial case it is obvious 
	that \textsf{CoffeeMachineImpl} is a valid implementation of 
	\textsf{CoffeeMachine}.
	
	Bi-similarity can be analyzed by converting a CCS process expression into a 
	\textit{labelled transition system}, which is a state machine where the 
	transitions between states are the actions performed in the process. Weak 
	bi-simulation, or observational equivalence, is perhaps the most interesting 
	bi-simulation to verify. In general terms it states that if $P$ and $Q$ are 
	weakly bi-similar then they will behave exactly the same when observed from 
	the outside, they will offer the same synchronizations or events. However 
	before and after these public events they can perform any number of internal 
	actions or $\tau$ actions which do not have to match between the two 
	processes since they do not affect their behavior as seen from the outside. 
	
	The PLR syntax tree is a rich data structure and would be well suited for 
	this type of analysis. This might not be the type of feature that belongs in 
	a process algebra compiler, instead it might be incorporated into some kind 
	of analysis tool for CCS (or other process algebras) that could make use of 
	the syntax tree of the compiler and the parser and scanner of the CCS 
	compiler. 
	
	This section has only briefly touched on the possibilities of verifying 
	process behavior using bi-simulation, for a more comprehensive explanation 
	see \cite{reactive}.
	
\section{Conclusions}

	The initial goal of this project was to explore how process algebras could 
	fit in with the .NET framework and how the constructs these algebras had in 
	common could be abstracted into common building blocks that could be re-used 
	in many implementations. The implementations of two process algebras using 
	those building blocks was necessary to prove their generality. Integrating a 
	process algebra into Visual Studio was more of an afterthought, but once I 
	started on it I saw how useful it was, and how much more enjoyable it was to 
	write CCS in this integrated environment, and so I was inspired to explore 
	exactly how well the language could be integrated and what services were 
	available.
	
	After finishing this project I believe that the .NET framework is a
	good platform for implementing process algebras. The main benefit is the 
	ease of interacting with other .NET languages. It is easy to add additional 
	features to a language by writing runtime libraries in languages like C\# 
	and it is easy to make a process algebra call into any arbitrary .NET 
	assembly. Another great benefit is being able to debug the compiled 
	applications, and in general great tool support in programs like Visual 
	Studio. The built in API to emit bytecode is well structured and easy to 
	use, and the bytecode itself is well designed and surprisingly readable once 
	you have gotten used to it. 
	
	There are downsides to .NET as well. The only real support for concurrency 
	is by using threads. I had thought beforehand that there might be some low 
	level support for that in the actual bytecode but that was not the case. 
	Threading can only be done through the standard class library, using the 
	\code{Thread} class, which shows that the class library is at least as 
	important as the virtual machine itself. It is also my conclusion that while
	.NET is a good platform for implementing process algebras, it is not 
	\textit{specifically} good for process algebras, it is more that it is a 
	good platform for implementing programming languages in general. 

	The fact that there exist other implementations of the CLI (Common Language 
	Infrastructure) specification is very useful. The PLR library, CCS compiler 
	and KLAIM compiler all work flawlessly on Mono, the CLI implementation that 
	runs on the Linux and Unix family of operating systems. The compiled process 
	applications also work on Mono without problems. I had expected that the 
	compilers and the PLR would work, since they are command line tools/libraries
	and do not depend on the underlying operating system much, except for reading
	and writing files. What I did not expect was that the Process Viewer 
	application would work, since it has a graphical user interface which uses
	the underlying graphic system of the Windows operating system. To my surprise
	the Process Viewer ran without any problems on Mono the first time I tried 
	it. It is worth noting that no special consideration was needed to support 
	Mono, I simply wrote the whole thing using .NET on Windows and once it was 
	ready I compiled and ran it with Mono 2.4.2 on an Ubuntu Linux 8.10 
	operating system and it worked flawlessly. This means that the PLR is truly 
	cross-platform, which should make it accessible to more people, especially 
	since many researchers in computer science do not use the Windows operating 
	system at all. 

	Building re-usable components for process algebras went very well. The PLR 
	takes care of a lot of the basic work that anyone implementing a process 
	algebra would otherwise have to do themselves. Not just the basic process 
	algebra constructs themselves, but also expression trees for arithmetic 
	expressions, static analyses, emitting debugging symbols, and helper methods 
	for calling into .NET code. An implementor can use this and focus his energy 
	on what matters, implementing specific constructs and emitting the bytecodes 
	for them.
	
	Overall I consider the project a success, as it has resulted in two working 
	process algebra implementations (one of which is fully integrated into 
	Visual Studio), a graphical tool for working with compiled process language 
	applications, as well as the main software product: a library/compiler that 
	can (and hopefully will) be used by future implementers of other process 
	algebras.
\appendix

%%%%%%%APPENDIX CHAPTERS INCLUDE%%%%%%%%%%%%%%%%%%%%%%%%%%%%%%%%%%%%%%%%%%%%%%

% Appendix A, B, ...
\appendix

\chapter{Software}

	During the course of this project three distinct software packages were 
	developed, a CCS compiler, a CCS integration package for Visual Studio and a 
	KLAIM compiler. Here we look at the practical aspects of this software, 
	where it can be downloaded, how it is licensed and how it can be configured 
	and used.

\section{Licensing and availability}

	All the software developed during this project can be downloaded from the 
	url http://einaregilsson.com/plr. The source code for the entire project is 
	available in a zip file. The binaries for each of the software packages can 
	be downloaded seperately. 
	
	The source code for the project is licensed under the General Public License 
	(GPL) v3.0. In brief, this allows anyone to download and modify the source or
	use it as a basis for something else, as long as the source code for that 
	modified version is also made available under a GPL compatible license. For 
	further information see http://www.gnu.org/licenses/gpl.html.

\section{CCS Compiler}

  The CCS compiler is an executable file named \texttt{ccs.exe}. It has one
  dependency which is the PLR itself, it is in a file named \texttt{PLR.dll}.
  The compiler is a command line tool and is invoked as 
  \begin{verbatim}ccs.exe [options] <filename>\end{verbatim} It accepts one 
  input file (\texttt{<filename>}) and can accept a number of optional command 
  line switches (\texttt{[options]}). By default the generated executable file 
  will have the same name as the input file, except ending with \texttt{.exe} 
  instead of \texttt{.ccs}. The compiled file will have a dependency on the 
  PLR for the runtime system. To get a guide to the available command line 
  options the compiler can be invoked as \texttt{ccs.exe /?} . The output of 
  that command is shown below:
	\begin{footnotesize}
	\begin{verbatim}
CCS Compiler
Copyright (C) 2009 Einar Egilsson

Usage: CCS [options] <filename>

Available options:

    /reference:<files>   The assemblies that this program requires. It is
    /r:<files>           not neccessary to specify the PLR assembly.
                         Other assemblies should be specified in a comma
                         seperated list, e.g. /reference:Foo.dll,Bar.dll.

    /optimize            If specified then the generated assembly will be
    /op                  optimized, dead code eliminated and expressions
                         pre-evaluated where possible. Do not combine this
                         with the /debug switch.

    /embedPLR            Embeds the PLR into the generated file, so it can
    /e                   be distributed as a stand-alone file.
  
    /debug               Emit debugging symbols in the generated file,
    /d                   this allows it to be debugged in Visual Studio, or
                         in the free graphical debugger that comes with the
                         .NET Framework SDK.

    /out:<filename>      Specify the name of the compiled executable. If
    /o:<filename>        this is not specified then the name of the input
                         file is used, with .ccs replaced by .exe.

    /print:<format>      Prints a version of the program source in the
    /p:<format>          specified format. Allowed formats are ccs, html
                         and latex. The generated file will have the same
                         name as the input file, except with the format
                         as extension.	
\end{verbatim}
\end{footnotesize}  
  
\section{CCS Visual Studio Integration Package}


\section{KLAIM Compiler}

  The KLAIM compiler is an executable file named \texttt{kc.exe}. It has two 
  dependencies, the KLAIM runtime (\texttt{KlaimRuntime.dll}) and the PLR 
  itself (\texttt{PLR.dll}).
  dependency which is the PLR itself, it is in a file named \texttt{PLR.dll}.
  The compiler is a command line tool and is invoked as 
  \begin{verbatim}ccs.exe [options] <filename>\end{verbatim} It accepts one 
  input file (\texttt{<filename>}) and can accept a number of optional command 
  line switches (\texttt{[options]}). By default the generated executable file 
  will have the same name as the input file, except ending with \texttt{.exe} 
  instead of \texttt{.ccs}. To get a guide to the available command line 
  options the compiler can be invoked as \texttt{ccs.exe /?} . The output of 
  that command is shown below:

\chapter{Generated bytecode}

  For those that are interested in the actual code generation that takes place 
  in the PLR, we now look at a sample system and its generated code. The 
  system we look at is the following:
  
  \begin{verbatim}
  use PLR.Runtime.BuiltIns

  StartProc = ActionPrefix | NonDeterministicChoice 

  ActionPrefix = a .  0

  ValuePassSend(x) = _a_(x+3 /:Rand(2)) . 0
  ValuePassReceive = a(y) . 0

  NonDeterministicChoice = _a_ . 0 + NDC2
  NDC2 = b . 0
 
  ParallelComposition = a . 0 | PC2 | 0 
  PC2 = b . 0

  MethodCall = :Print("Hello") . 0

  Restrict = ( a . (d . 0)\ d ) \{a}

  Relabel = ( a . (d . 0)[dnew/d] )[anew/a]  
  \end{verbatim}

  This system obviously is not a model of anything special, it is simply 
  composed to show the available features of the PLR, and the processes are 
  named accordingly, e.g. ActionPrefix and Restrict. In this appendix each 
  activity type will have its own section, where the relevant process is first 
  shown in CCS and then the generated bytecode is shown. The text 
  representation of the bytecode was generated using the intermediate language 
  disassembler tool from Microsoft, \textit{ILDASM}.
  
  \section{Header}
  
  The header of the CIL file is as follows:

	\begin{cil}}

//  Microsoft (R) .NET Framework IL Disassembler.  Version 3.5.21022.8
//  Copyright (c) Microsoft Corporation.  All rights reserved.

// Metadata version: v2.0.50727
.assembly extern PLR
{
  .ver 1:0:0:0
}
.assembly extern mscorlib
{
  .publickeytoken = (B7 7A 5C 56 19 34 E0 89 )        // .z\V.4..
  .ver 2:0:0:0
}
.assembly disassemble.exe
{
  .hash algorithm 0x00008004
  .ver 0:0:0:0
}
.module disassemble.exe
// MVID: {A23EA4EB-468E-4E2B-A4CA-AA68D8C2320E}
.imagebase 0x00400000
.file alignment 0x00000200
.stackreserve 0x00100000
.subsystem 0x0003       // WINDOWS_CUI
.corflags 0x00000001    //  ILONLY
// Image base: 0x00970000

\end{cil}

  \section{Main method}
  
	The main method of the executable, which is the entry point, is as follows:
	
	\begin{cil}
// ================== GLOBAL METHODS =========================

.method public static int32  Main() cil managed
{
  .entrypoint
  // Code size       18 (0x12)
  .maxstack  1
  .locals init (class StartProc V_0)
  IL_0000:  newobj     instance void StartProc::.ctor()
  IL_0005:  stloc.0
  IL_0006:  call       class [PLR]PLR.Runtime.Scheduler 
                       [PLR]PLR.Runtime.Scheduler::get_Instance()
  IL_000b:  call       instance void [PLR]PLR.Runtime.Scheduler::Run()
  IL_0010:  ldc.i4.0
  IL_0011:  ret
} // end of global method Main


// =============================================================

	\end{cil}	  
  \section{StartProc}
	\begin{verbatim}
  StartProc = ActionPrefix | NonDeterministicChoice 
	\end{verbatim}
	
	is compiled as follows:
	
	\begin{cil}

.class public auto ansi beforefieldinit StartProc
       extends [PLR]PLR.Runtime.ProcessBase
{
  .method public specialname rtspecialname 
          instance void  .ctor() cil managed
  {
    // Code size       7 (0x7)
    .maxstack  2
    IL_0000:  ldarg.0
    IL_0001:  call       instance void [PLR]PLR.Runtime.ProcessBase::.ctor()
    IL_0006:  ret
  } // end of method StartProc::.ctor

  .method public virtual instance void  RunProcess() cil managed
  {
    .override [PLR]PLR.Runtime.ProcessBase::RunProcess
    // Code size       63 (0x3f)
    .maxstack  2
    .locals init ([0] class [PLR]PLR.Runtime.ProcessBase V_0,
             [1] class [PLR]PLR.Runtime.ProcessBase V_1)
    IL_0000:  ldarg.0
    IL_0001:  call       instance void [PLR]PLR.Runtime.ProcessBase
                         ::InitSetID()
    IL_0006:  nop
    IL_0007:  newobj     instance void ActionPrefix::.ctor()
    IL_000c:  stloc.0
    IL_000d:  ldloc.0
    IL_000e:  ldarg.0
    IL_000f:  call       instance class [PLR]PLR.Runtime.ProcessBase 
                         [PLR]PLR.Runtime.ProcessBase::get_Parent()
    IL_0014:  call       instance void [PLR]PLR.Runtime.ProcessBase
                         ::set_Parent(class [PLR]PLR.Runtime.ProcessBase)
    IL_0019:  ldloc.0
    IL_001a:  call       instance void [PLR]PLR.Runtime.ProcessBase::Run()
    IL_001f:  nop
    IL_0020:  newobj     instance void NonDeterministicChoice::.ctor()
    IL_0025:  stloc.1
    IL_0026:  ldloc.1
    IL_0027:  ldarg.0
    IL_0028:  call       instance class [PLR]PLR.Runtime.ProcessBase 
                         [PLR]PLR.Runtime.ProcessBase::get_Parent()
    IL_002d:  call       instance void [PLR]PLR.Runtime.ProcessBase
                         ::set_Parent(class [PLR]PLR.Runtime.ProcessBase)
    IL_0032:  ldloc.1
    IL_0033:  call       instance void [PLR]PLR.Runtime.ProcessBase::Run()
    IL_0038:  ldarg.0
    IL_0039:  call       instance void [PLR]PLR.Runtime.ProcessBase::Die()
    IL_003e:  ret
  } // end of method StartProc::RunProcess

} // end of class StartProc

\end{cil}

\section{ActionPrefix}
	
	\begin{verbatim}
  ActionPrefix = a .  0
	\end{verbatim}
	
	is compiled as follows:
	
	\begin{cil}

.class public auto ansi beforefieldinit ActionPrefix
       extends [PLR]PLR.Runtime.ProcessBase
{
  .method public specialname rtspecialname 
          instance void  .ctor() cil managed
  {
    // Code size       7 (0x7)
    .maxstack  2
    IL_0000:  ldarg.0
    IL_0001:  call       instance void [PLR]PLR.Runtime.ProcessBase::.ctor()
    IL_0006:  ret
  } // end of method ActionPrefix::.ctor

  .method public virtual instance void  RunProcess() cil managed
  {
    .override [PLR]PLR.Runtime.ProcessBase::RunProcess
    // Code size       84 (0x54)
    .maxstack  10
    .locals init ([0] class [PLR]PLR.Runtime.ChannelSyncAction V_0)
    IL_0000:  ldarg.0
    IL_0001:  call       instance void [PLR]PLR.Runtime.ProcessBase
                         ::InitSetID()
    .try
    {
      IL_0006:  ldarg.0
      IL_0007:  ldstr      "Preparing to sync now..."
      IL_000c:  call       instance void [PLR]PLR.Runtime.ProcessBase
                           ::Debug(string)
      IL_0011:  ldarg.0
      IL_0012:  ldstr      "a"
      IL_0017:  ldarg.0
      IL_0018:  ldc.i4     0x0
      IL_001d:  ldc.i4.1
      IL_001e:  newobj     instance void [PLR]PLR.Runtime.ChannelSyncAction
                           ::.ctor(string, class [PLR]PLR.Runtime.ProcessBase
                                   ,int32, bool)
      IL_0023:  stloc.0
      IL_0024:  ldloc.0
      IL_0025:  call       instance void [PLR]PLR.Runtime.ProcessBase
                           ::Sync(class [PLR]PLR.Runtime.IAction)
      IL_002a:  nop
      IL_002b:  nop
      IL_002c:  ldarg.0
      IL_002d:  ldstr      "Turned into 0"
      IL_0032:  call       instance void [PLR]PLR.Runtime.ProcessBase
                           ::Debug(string)
      IL_0037:  leave      IL_004d

    }  // end .try
    catch [PLR]PLR.Runtime.ProcessKilledException 
    {
      IL_003c:  pop
      IL_003d:  ldarg.0
      IL_003e:  ldstr      "Caught ProcessKilledException"
      IL_0043:  call       instance void [PLR]PLR.Runtime.ProcessBase
                           ::Debug(string)
      IL_0048:  leave      IL_004d

    }  // end handler
    IL_004d:  ldarg.0
    IL_004e:  call       instance void [PLR]PLR.Runtime.ProcessBase::Die()
    IL_0053:  ret
  } // end of method ActionPrefix::RunProcess

} // end of class ActionPrefix

\end{cil}

\section{ValuePassSend and ValuePassReceive}

	\begin{verbatim}
  ValuePassSend(x) = _a_(x+3 /:Rand(2)) . 0
  ValuePassReceive = a(y) . 0
	\end{verbatim}
	
	are compiled as follows:

\begin{cil}
.class public auto ansi beforefieldinit ValuePassSend_1
       extends [PLR]PLR.Runtime.ProcessBase
{
  .field assembly object x
  .method public specialname rtspecialname 
          instance void  .ctor(object x) cil managed
  {
    // Code size       19 (0x13)
    .maxstack  4
    IL_0000:  ldarg.0
    IL_0001:  call       instance void [PLR]PLR.Runtime.ProcessBase::.ctor()
    IL_0006:  ldarg.0
    IL_0007:  ldarg      x
    IL_000b:  nop
    IL_000c:  nop
    IL_000d:  stfld      object ValuePassSend_1::x
    IL_0012:  ret
  } // end of method ValuePassSend_1::.ctor

  .method public virtual instance void  RunProcess() cil managed
  {
    .override [PLR]PLR.Runtime.ProcessBase::RunProcess
    // Code size       125 (0x7d)
    .maxstack  10
    .locals init ([0] object x,
             [1] class [PLR]PLR.Runtime.ChannelSyncAction V_1)
    IL_0000:  ldarg.0
    IL_0001:  call       instance void [PLR]PLR.Runtime.ProcessBase
                         ::InitSetID()
    .try
    {
      IL_0006:  ldarg.0
      IL_0007:  ldfld      object ValuePassSend_1::x
      IL_000c:  stloc.0
      IL_000d:  ldarg.0
      IL_000e:  ldstr      "Preparing to sync now..."
      IL_0013:  call       instance void [PLR]PLR.Runtime.ProcessBase
                           ::Debug(string)
      IL_0018:  ldarg.0
      IL_0019:  ldstr      "a"
      IL_001e:  ldarg.0
      IL_001f:  ldc.i4     0x1
      IL_0024:  ldc.i4.0
      IL_0025:  newobj     instance void [PLR]PLR.Runtime.ChannelSyncAction
                           ::.ctor(string,class [PLR]PLR.Runtime.ProcessBase,
                           int32, bool)
      IL_002a:  stloc.1
      IL_002b:  ldloc.1
      IL_002c:  ldloc.0
      IL_002d:  unbox.any  [mscorlib]System.Int32
      IL_0032:  ldc.i4     0x3
      IL_0037:  ldc.i4     0x2
      IL_003c:  call       int32 [PLR]PLR.Runtime.BuiltIns::Rand(int32)
      IL_0041:  div
      IL_0042:  add.ovf
      IL_0043:  box        [mscorlib]System.Int32
      IL_0048:  call       instance void [PLR]PLR.Runtime.ChannelSyncAction
                           ::AddValue(object)
      IL_004d:  ldloc.1
      IL_004e:  call       instance void [PLR]PLR.Runtime.ProcessBase
                           ::Sync(class [PLR]PLR.Runtime.IAction)
      IL_0053:  nop
      IL_0054:  nop
      IL_0055:  ldarg.0
      IL_0056:  ldstr      "Turned into 0"
      IL_005b:  call       instance void [PLR]PLR.Runtime.ProcessBase
                           ::Debug(string)
      IL_0060:  leave      IL_0076

    }  // end .try
    catch [PLR]PLR.Runtime.ProcessKilledException 
    {
      IL_0065:  pop
      IL_0066:  ldarg.0
      IL_0067:  ldstr      "Caught ProcessKilledException"
      IL_006c:  call       instance void [PLR]PLR.Runtime.ProcessBase
                           ::Debug(string)
      IL_0071:  leave      IL_0076

    }  // end handler
    IL_0076:  ldarg.0
    IL_0077:  call       instance void [PLR]PLR.Runtime.ProcessBase::Die()
    IL_007c:  ret
  } // end of method ValuePassSend_1::RunProcess

} // end of class ValuePassSend_1

.class public auto ansi beforefieldinit ValuePassReceive
       extends [PLR]PLR.Runtime.ProcessBase
{
  .method public specialname rtspecialname 
          instance void  .ctor() cil managed
  {
    // Code size       7 (0x7)
    .maxstack  2
    IL_0000:  ldarg.0
    IL_0001:  call       instance void [PLR]PLR.Runtime.ProcessBase::.ctor()
    IL_0006:  ret
  } // end of method ValuePassReceive::.ctor

  .method public virtual instance void  RunProcess() cil managed
  {
    .override [PLR]PLR.Runtime.ProcessBase::RunProcess
    // Code size       96 (0x60)
    .maxstack  10
    .locals init ([0] class [PLR]PLR.Runtime.ChannelSyncAction V_0,
             [1] object y)
    IL_0000:  ldarg.0
    IL_0001:  call       instance void [PLR]PLR.Runtime.ProcessBase
                         ::InitSetID()
    .try
    {
      IL_0006:  ldarg.0
      IL_0007:  ldstr      "Preparing to sync now..."
      IL_000c:  call       instance void [PLR]PLR.Runtime.ProcessBase
                           ::Debug(string)
      IL_0011:  ldarg.0
      IL_0012:  ldstr      "a"
      IL_0017:  ldarg.0
      IL_0018:  ldc.i4     0x1
      IL_001d:  ldc.i4.1
      IL_001e:  newobj     instance void [PLR]PLR.Runtime.ChannelSyncAction
                           ::.ctor(string,class [PLR]PLR.Runtime.ProcessBase,
                           int32, bool)
      IL_0023:  stloc.0
      IL_0024:  ldloc.0
      IL_0025:  call       instance void [PLR]PLR.Runtime.ProcessBase
                           ::Sync(class [PLR]PLR.Runtime.IAction)
      IL_002a:  nop
      IL_002b:  ldloc.0
      IL_002c:  ldc.i4     0x0
      IL_0031:  call       instance object [PLR]PLR.Runtime.ChannelSyncAction
                           ::GetValue(int32)
      IL_0036:  stloc.1
      IL_0037:  nop
      IL_0038:  ldarg.0
      IL_0039:  ldstr      "Turned into 0"
      IL_003e:  call       instance void [PLR]PLR.Runtime.ProcessBase
                           ::Debug(string)
      IL_0043:  leave      IL_0059

    }  // end .try
    catch [PLR]PLR.Runtime.ProcessKilledException 
    {
      IL_0048:  pop
      IL_0049:  ldarg.0
      IL_004a:  ldstr      "Caught ProcessKilledException"
      IL_004f:  call       instance void [PLR]PLR.Runtime.ProcessBase
                           ::Debug(string)
      IL_0054:  leave      IL_0059

    }  // end handler
    IL_0059:  ldarg.0
    IL_005a:  call       instance void [PLR]PLR.Runtime.ProcessBase::Die()
    IL_005f:  ret
  } // end of method ValuePassReceive::RunProcess

} // end of class ValuePassReceive

\end{cil}

\section{NonDeterministicChoice}

	\begin{verbatim}
  NonDeterministicChoice = _a_ . 0 + NDC2
  NDC2 = b . 0
	\end{verbatim}
	
	are compiled as follows:

\begin{cil}
.class public auto ansi beforefieldinit NonDeterministicChoice
       extends [PLR]PLR.Runtime.ProcessBase
{
  .class auto ansi nested public beforefieldinit NonDeterministic1
         extends [PLR]PLR.Runtime.ProcessBase
  {
    .method public virtual instance void 
            RunProcess() cil managed
    {
      .override [PLR]PLR.Runtime.ProcessBase::RunProcess
      // Code size       84 (0x54)
      .maxstack  10
      .locals init ([0] class [PLR]PLR.Runtime.ChannelSyncAction V_0)
      IL_0000:  ldarg.0
      IL_0001:  call       instance void [PLR]PLR.Runtime.ProcessBase
                           ::InitSetID()
      .try
      {
        IL_0006:  ldarg.0
        IL_0007:  ldstr      "Preparing to sync now..."
        IL_000c:  call       instance void [PLR]PLR.Runtime.ProcessBase
                             ::Debug(string)
        IL_0011:  ldarg.0
        IL_0012:  ldstr      "a"
        IL_0017:  ldarg.0
        IL_0018:  ldc.i4     0x0
        IL_001d:  ldc.i4.0
        IL_001e:  newobj     instance void [PLR]PLR.Runtime.ChannelSyncAction
                             ::.ctor(string,
                             class [PLR]PLR.Runtime.ProcessBase, int32, bool)
        IL_0023:  stloc.0
        IL_0024:  ldloc.0
        IL_0025:  call       instance void [PLR]PLR.Runtime.ProcessBase
                             ::Sync(class [PLR]PLR.Runtime.IAction)
        IL_002a:  nop
        IL_002b:  nop
        IL_002c:  ldarg.0
        IL_002d:  ldstr      "Turned into 0"
        IL_0032:  call       instance void [PLR]PLR.Runtime.ProcessBase
                             ::Debug(string)
        IL_0037:  leave      IL_004d

      }  // end .try
      catch [PLR]PLR.Runtime.ProcessKilledException 
      {
        IL_003c:  pop
        IL_003d:  ldarg.0
        IL_003e:  ldstr      "Caught ProcessKilledException"
        IL_0043:  call       instance void [PLR]PLR.Runtime.ProcessBase
                             ::Debug(string)
        IL_0048:  leave      IL_004d

      }  // end handler
      IL_004d:  ldarg.0
      IL_004e:  call       instance void [PLR]PLR.Runtime.ProcessBase::Die()
      IL_0053:  ret
    } // end of method NonDeterministic1::RunProcess

    .method public specialname rtspecialname 
            instance void  .ctor() cil managed
    {
      // Code size       7 (0x7)
      .maxstack  2
      IL_0000:  ldarg.0
      IL_0001:  call       instance void [PLR]PLR.Runtime.ProcessBase
                           ::.ctor()
      IL_0006:  ret
    } // end of method NonDeterministic1::.ctor

  } // end of class NonDeterministic1

  .method public specialname rtspecialname 
          instance void  .ctor() cil managed
  {
    // Code size       7 (0x7)
    .maxstack  2
    IL_0000:  ldarg.0
    IL_0001:  call       instance void [PLR]PLR.Runtime.ProcessBase
                         ::.ctor()
    IL_0006:  ret
  } // end of method NonDeterministicChoice::.ctor

  .method public virtual instance void  RunProcess() cil managed
  {
    .override [PLR]PLR.Runtime.ProcessBase::RunProcess
    // Code size       85 (0x55)
    .maxstack  2
    .locals init (class [PLR]PLR.Runtime.ProcessBase V_0,
             class [PLR]PLR.Runtime.ProcessBase V_1)
    IL_0000:  ldarg.0
    IL_0001:  call       instance void [PLR]PLR.Runtime.ProcessBase
                         ::InitSetID()
    IL_0006:  newobj     instance void 
                         NonDeterministicChoice/NonDeterministic1::.ctor()
    IL_000b:  stloc.0
    IL_000c:  ldloc.0
    IL_000d:  ldarg.0
    IL_000e:  call       instance class [PLR]PLR.Runtime.ProcessBase 
                         [PLR]PLR.Runtime.ProcessBase::get_Parent()
    IL_0013:  call       instance void [PLR]PLR.Runtime.ProcessBase
                         ::set_Parent(class [PLR]PLR.Runtime.ProcessBase)
    IL_0018:  ldloc.0
    IL_0019:  ldarg.0
    IL_001a:  call       instance valuetype [mscorlib]System.Guid 
                         [PLR]PLR.Runtime.ProcessBase::get_SetID()
    IL_001f:  call       instance void [PLR]PLR.Runtime.ProcessBase
                         ::set_SetID(valuetype [mscorlib]System.Guid)
    IL_0024:  ldloc.0
    IL_0025:  call       instance void [PLR]PLR.Runtime.ProcessBase::Run()
    IL_002a:  newobj     instance void NDC2::.ctor()
    IL_002f:  stloc.1
    IL_0030:  ldloc.1
    IL_0031:  ldarg.0
    IL_0032:  call       instance class [PLR]PLR.Runtime.ProcessBase 
                         [PLR]PLR.Runtime.ProcessBase::get_Parent()
    IL_0037:  call       instance void [PLR]PLR.Runtime.ProcessBase
                         ::set_Parent(class [PLR]PLR.Runtime.ProcessBase)
    IL_003c:  ldloc.1
    IL_003d:  ldarg.0
    IL_003e:  call       instance valuetype [mscorlib]System.Guid 
                         [PLR]PLR.Runtime.ProcessBase::get_SetID()
    IL_0043:  call       instance void [PLR]PLR.Runtime.ProcessBase
                         ::set_SetID(valuetype [mscorlib]System.Guid)
    IL_0048:  ldloc.1
    IL_0049:  call       instance void [PLR]PLR.Runtime.ProcessBase::Run()
    IL_004e:  ldarg.0
    IL_004f:  call       instance void [PLR]PLR.Runtime.ProcessBase::Die()
    IL_0054:  ret
  } // end of method NonDeterministicChoice::RunProcess

} // end of class NonDeterministicChoice

.class public auto ansi beforefieldinit NDC2
       extends [PLR]PLR.Runtime.ProcessBase
{
  .method public specialname rtspecialname 
          instance void  .ctor() cil managed
  {
    // Code size       7 (0x7)
    .maxstack  2
    IL_0000:  ldarg.0
    IL_0001:  call       instance void [PLR]PLR.Runtime.ProcessBase::.ctor()
    IL_0006:  ret
  } // end of method NDC2::.ctor

  .method public virtual instance void  RunProcess() cil managed
  {
    .override [PLR]PLR.Runtime.ProcessBase::RunProcess
    // Code size       84 (0x54)
    .maxstack  10
    .locals init ([0] class [PLR]PLR.Runtime.ChannelSyncAction V_0)
    IL_0000:  ldarg.0
    IL_0001:  call       instance void [PLR]PLR.Runtime.ProcessBase
                         ::InitSetID()
    .try
    {
      IL_0006:  ldarg.0
      IL_0007:  ldstr      "Preparing to sync now..."
      IL_000c:  call       instance void [PLR]PLR.Runtime.ProcessBase
                           ::Debug(string)
      IL_0011:  ldarg.0
      IL_0012:  ldstr      "b"
      IL_0017:  ldarg.0
      IL_0018:  ldc.i4     0x0
      IL_001d:  ldc.i4.1
      IL_001e:  newobj     instance void [PLR]PLR.Runtime.ChannelSyncAction
                           ::.ctor(string,class [PLR]PLR.Runtime.ProcessBase,
                           int32, bool)
      IL_0023:  stloc.0
      IL_0024:  ldloc.0
      IL_0025:  call       instance void [PLR]PLR.Runtime.ProcessBase
                           ::Sync(class [PLR]PLR.Runtime.IAction)
      IL_002a:  nop
      IL_002b:  nop
      IL_002c:  ldarg.0
      IL_002d:  ldstr      "Turned into 0"
      IL_0032:  call       instance void [PLR]PLR.Runtime.ProcessBase
                           ::Debug(string)
      IL_0037:  leave      IL_004d

    }  // end .try
    catch [PLR]PLR.Runtime.ProcessKilledException 
    {
      IL_003c:  pop
      IL_003d:  ldarg.0
      IL_003e:  ldstr      "Caught ProcessKilledException"
      IL_0043:  call       instance void [PLR]PLR.Runtime.ProcessBase
                           ::Debug(string)
      IL_0048:  leave      IL_004d

    }  // end handler
    IL_004d:  ldarg.0
    IL_004e:  call       instance void [PLR]PLR.Runtime.ProcessBase::Die()
    IL_0053:  ret
  } // end of method NDC2::RunProcess

} // end of class NDC2

\end{cil}

\section{ParallelComposition}

	\begin{verbatim}
  ParallelComposition = a . 0 | PC2 | 0 
  PC2 = b . 0
	\end{verbatim}
	
	are compiled as follows:

\begin{cil}

.class public auto ansi beforefieldinit ParallelComposition
       extends [PLR]PLR.Runtime.ProcessBase
{
  .class auto ansi nested public beforefieldinit Parallel1
         extends [PLR]PLR.Runtime.ProcessBase
  {
    .method public virtual instance void 
            RunProcess() cil managed
    {
      .override [PLR]PLR.Runtime.ProcessBase::RunProcess
      // Code size       84 (0x54)
      .maxstack  10
      .locals init ([0] class [PLR]PLR.Runtime.ChannelSyncAction V_0)
      IL_0000:  ldarg.0
      IL_0001:  call       instance void [PLR]PLR.Runtime.ProcessBase
                           ::InitSetID()
      .try
      {
        IL_0006:  ldarg.0
        IL_0007:  ldstr      "Preparing to sync now..."
        IL_000c:  call       instance void [PLR]PLR.Runtime.ProcessBase
                             ::Debug(string)
        IL_0011:  ldarg.0
        IL_0012:  ldstr      "a"
        IL_0017:  ldarg.0
        IL_0018:  ldc.i4     0x0
        IL_001d:  ldc.i4.1
        IL_001e:  newobj     instance void [PLR]PLR.Runtime.ChannelSyncAction
                             ::.ctor(string,class 
                             [PLR]PLR.Runtime.ProcessBase, int32, bool)
        IL_0023:  stloc.0
        IL_0024:  ldloc.0
        IL_0025:  call       instance void [PLR]PLR.Runtime.ProcessBase
                             ::Sync(class [PLR]PLR.Runtime.IAction)
        IL_002a:  nop
        IL_002b:  nop
        IL_002c:  ldarg.0
        IL_002d:  ldstr      "Turned into 0"
        IL_0032:  call       instance void [PLR]PLR.Runtime.ProcessBase
                             ::Debug(string)
        IL_0037:  leave      IL_004d

      }  // end .try
      catch [PLR]PLR.Runtime.ProcessKilledException 
      {
        IL_003c:  pop
        IL_003d:  ldarg.0
        IL_003e:  ldstr      "Caught ProcessKilledException"
        IL_0043:  call       instance void [PLR]PLR.Runtime.ProcessBase
                             ::Debug(string)
        IL_0048:  leave      IL_004d

      }  // end handler
      IL_004d:  ldarg.0
      IL_004e:  call       instance void [PLR]PLR.Runtime.ProcessBase::Die()
      IL_0053:  ret
    } // end of method Parallel1::RunProcess

    .method public specialname rtspecialname 
            instance void  .ctor() cil managed
    {
      // Code size       7 (0x7)
      .maxstack  2
      IL_0000:  ldarg.0
      IL_0001:  call       instance void [PLR]PLR.Runtime.ProcessBase
                           ::.ctor()
      IL_0006:  ret
    } // end of method Parallel1::.ctor

  } // end of class Parallel1

  .class auto ansi nested public beforefieldinit Parallel3
         extends [PLR]PLR.Runtime.ProcessBase
  {
    .method public virtual instance void 
            RunProcess() cil managed
    {
      .override [PLR]PLR.Runtime.ProcessBase::RunProcess
      // Code size       25 (0x19)
      .maxstack  2
      IL_0000:  ldarg.0
      IL_0001:  call       instance void [PLR]PLR.Runtime.ProcessBase
                           ::InitSetID()
      IL_0006:  nop
      IL_0007:  ldarg.0
      IL_0008:  ldstr      "Turned into 0"
      IL_000d:  call       instance void [PLR]PLR.Runtime.ProcessBase
                           ::Debug(string)
      IL_0012:  ldarg.0
      IL_0013:  call       instance void [PLR]PLR.Runtime.ProcessBase
                           ::Die()
      IL_0018:  ret
    } // end of method Parallel3::RunProcess

    .method public specialname rtspecialname 
            instance void  .ctor() cil managed
    {
      // Code size       7 (0x7)
      .maxstack  2
      IL_0000:  ldarg.0
      IL_0001:  call       instance void [PLR]PLR.Runtime.ProcessBase
                           ::.ctor()
      IL_0006:  ret
    } // end of method Parallel3::.ctor

  } // end of class Parallel3

  .method public specialname rtspecialname 
          instance void  .ctor() cil managed
  {
    // Code size       7 (0x7)
    .maxstack  2
    IL_0000:  ldarg.0
    IL_0001:  call       instance void [PLR]PLR.Runtime.ProcessBase
                         ::.ctor()
    IL_0006:  ret
  } // end of method ParallelComposition::.ctor

  .method public virtual instance void  RunProcess() cil managed
  {
    .override [PLR]PLR.Runtime.ProcessBase::RunProcess
    // Code size       86 (0x56)
    .maxstack  2
    .locals init ([0] class [PLR]PLR.Runtime.ProcessBase V_0,
             [1] class [PLR]PLR.Runtime.ProcessBase V_1,
             [2] class [PLR]PLR.Runtime.ProcessBase V_2)
    IL_0000:  ldarg.0
    IL_0001:  call       instance void [PLR]PLR.Runtime.ProcessBase+
                         ::InitSetID()
    IL_0006:  newobj     instance void ParallelComposition/Parallel1
                         ::.ctor()
    IL_000b:  stloc.0
    IL_000c:  ldloc.0
    IL_000d:  ldarg.0
    IL_000e:  call       instance class [PLR]PLR.Runtime.ProcessBase 
                         [PLR]PLR.Runtime.ProcessBase::get_Parent()
    IL_0013:  call       instance void [PLR]PLR.Runtime.ProcessBase
                         ::set_Parent(class [PLR]PLR.Runtime.ProcessBase)
    IL_0018:  ldloc.0
    IL_0019:  call       instance void [PLR]PLR.Runtime.ProcessBase::Run()
    IL_001e:  nop
    IL_001f:  newobj     instance void PC2::.ctor()
    IL_0024:  stloc.1
    IL_0025:  ldloc.1
    IL_0026:  ldarg.0
    IL_0027:  call       instance class [PLR]PLR.Runtime.ProcessBase 
                         [PLR]PLR.Runtime.ProcessBase::get_Parent()
    IL_002c:  call       instance void [PLR]PLR.Runtime.ProcessBase
                         ::set_Parent(class [PLR]PLR.Runtime.ProcessBase)
    IL_0031:  ldloc.1
    IL_0032:  call       instance void [PLR]PLR.Runtime.ProcessBase::Run()
    IL_0037:  newobj     instance void ParallelComposition/Parallel3::.ctor()
    IL_003c:  stloc.2
    IL_003d:  ldloc.2
    IL_003e:  ldarg.0
    IL_003f:  call       instance class [PLR]PLR.Runtime.ProcessBase 
                         [PLR]PLR.Runtime.ProcessBase::get_Parent()
    IL_0044:  call       instance void [PLR]PLR.Runtime.ProcessBase
                         ::set_Parent(class [PLR]PLR.Runtime.ProcessBase)
    IL_0049:  ldloc.2
    IL_004a:  call       instance void [PLR]PLR.Runtime.ProcessBase::Run()
    IL_004f:  ldarg.0
    IL_0050:  call       instance void [PLR]PLR.Runtime.ProcessBase::Die()
    IL_0055:  ret
  } // end of method ParallelComposition::RunProcess

} // end of class ParallelComposition

.class public auto ansi beforefieldinit PC2
       extends [PLR]PLR.Runtime.ProcessBase
{
  .method public specialname rtspecialname 
          instance void  .ctor() cil managed
  {
    // Code size       7 (0x7)
    .maxstack  2
    IL_0000:  ldarg.0
    IL_0001:  call       instance void [PLR]PLR.Runtime.ProcessBase::.ctor()
    IL_0006:  ret
  } // end of method PC2::.ctor

  .method public virtual instance void  RunProcess() cil managed
  {
    .override [PLR]PLR.Runtime.ProcessBase::RunProcess
    // Code size       84 (0x54)
    .maxstack  10
    .locals init ([0] class [PLR]PLR.Runtime.ChannelSyncAction V_0)
    IL_0000:  ldarg.0
    IL_0001:  call       instance void [PLR]PLR.Runtime.ProcessBase
                         ::InitSetID()
    .try
    {
      IL_0006:  ldarg.0
      IL_0007:  ldstr      "Preparing to sync now..."
      IL_000c:  call       instance void [PLR]PLR.Runtime.ProcessBase
                           ::Debug(string)
      IL_0011:  ldarg.0
      IL_0012:  ldstr      "b"
      IL_0017:  ldarg.0
      IL_0018:  ldc.i4     0x0
      IL_001d:  ldc.i4.1
      IL_001e:  newobj     instance void [PLR]PLR.Runtime.ChannelSyncAction
                           ::.ctor(string, 
                           class [PLR]PLR.Runtime.ProcessBase, int32, bool)
      IL_0023:  stloc.0
      IL_0024:  ldloc.0
      IL_0025:  call       instance void [PLR]PLR.Runtime.ProcessBase
                           ::Sync(class [PLR]PLR.Runtime.IAction)
      IL_002a:  nop
      IL_002b:  nop
      IL_002c:  ldarg.0
      IL_002d:  ldstr      "Turned into 0"
      IL_0032:  call       instance void [PLR]PLR.Runtime.ProcessBase
                           ::Debug(string)
      IL_0037:  leave      IL_004d

    }  // end .try
    catch [PLR]PLR.Runtime.ProcessKilledException 
    {
      IL_003c:  pop
      IL_003d:  ldarg.0
      IL_003e:  ldstr      "Caught ProcessKilledException"
      IL_0043:  call       instance void [PLR]PLR.Runtime.ProcessBase
                           ::Debug(string)
      IL_0048:  leave      IL_004d

    }  // end handler
    IL_004d:  ldarg.0
    IL_004e:  call       instance void [PLR]PLR.Runtime.ProcessBase::Die()
    IL_0053:  ret
  } // end of method PC2::RunProcess

} // end of class PC2
\end{cil}

\section{MethodCall}

	\begin{verbatim}
	MethodCall = :Print("Hello") . 0
	\end{verbatim}
	
	is compiled as follows:

\begin{cil}

.class public auto ansi beforefieldinit MethodCall
       extends [PLR]PLR.Runtime.ProcessBase
{
  .method public specialname rtspecialname 
          instance void  .ctor() cil managed
  {
    // Code size       7 (0x7)
    .maxstack  2
    IL_0000:  ldarg.0
    IL_0001:  call       instance void [PLR]PLR.Runtime.ProcessBase::.ctor()
    IL_0006:  ret
  } // end of method MethodCall::.ctor

  .method public virtual instance void  RunProcess() cil managed
  {
    .override [PLR]PLR.Runtime.ProcessBase::RunProcess
    // Code size       86 (0x56)
    .maxstack  8
    IL_0000:  ldarg.0
    IL_0001:  call       instance void [PLR]PLR.Runtime.ProcessBase
                         ::InitSetID()
    .try
    {
      IL_0006:  ldarg.0
      IL_0007:  ldstr      "Preparing to sync now..."
      IL_000c:  call       instance void [PLR]PLR.Runtime.ProcessBase
                           ::Debug(string)
      IL_0011:  ldarg.0
      IL_0012:  ldstr      "Print(\"Hello\")"
      IL_0017:  ldarg.0
      IL_0018:  newobj     instance void [PLR]PLR.Runtime.MethodCallAction
                           ::.ctor(string, 
                           class [PLR]PLR.Runtime.ProcessBase)
      IL_001d:  call       instance void [PLR]PLR.Runtime.ProcessBase
                           ::Sync(class [PLR]PLR.Runtime.IAction)
      IL_0022:  nop
      IL_0023:  ldstr      "Hello"
      IL_0028:  call       void [PLR]PLR.Runtime.BuiltIns::Print(object)
      IL_002d:  nop
      IL_002e:  ldarg.0
      IL_002f:  ldstr      "Turned into 0"
      IL_0034:  call       instance void [PLR]PLR.Runtime.ProcessBase
                           ::Debug(string)
      IL_0039:  leave      IL_004f

    }  // end .try
    catch [PLR]PLR.Runtime.ProcessKilledException 
    {
      IL_003e:  pop
      IL_003f:  ldarg.0
      IL_0040:  ldstr      "Caught ProcessKilledException"
      IL_0045:  call       instance void [PLR]PLR.Runtime.ProcessBase
                           ::Debug(string)
      IL_004a:  leave      IL_004f

    }  // end handler
    IL_004f:  ldarg.0
    IL_0050:  call       instance void [PLR]PLR.Runtime.ProcessBase
                         ::Die()
    IL_0055:  ret
  } // end of method MethodCall::RunProcess

} // end of class MethodCall

\end{cil}

\section{Restrict}

	\begin{verbatim}
  Restrict = ( a . (d . 0)\ d ) \{a}
	\end{verbatim}
	
	is compiled as follows:

\begin{cil}

.class public auto ansi beforefieldinit Restrict
       extends [PLR]PLR.Runtime.ProcessBase
{
  .class auto ansi nested public beforefieldinit Inner
         extends [PLR]PLR.Runtime.ProcessBase
  {
    .method public static bool  RestrictByName(
        class [PLR]PLR.Runtime.IAction A_0) cil managed
    {
      // Code size       37 (0x25)
      .maxstack  3
      .locals init ([0] class [PLR]PLR.Runtime.ChannelSyncAction V_0)
      IL_0000:  ldarg.0
      IL_0001:  isinst     [PLR]PLR.Runtime.ChannelSyncAction
      IL_0006:  brtrue     IL_000d

      IL_000b:  ldc.i4.0
      IL_000c:  ret

      IL_000d:  ldarg.0
      IL_000e:  castclass  [PLR]PLR.Runtime.ChannelSyncAction
      IL_0013:  stloc.0
      IL_0014:  ldloc.0
      IL_0015:  call       instance string [PLR]PLR.Runtime.ChannelSyncAction
                           ::get_Name()
      IL_001a:  ldstr      "d"
      IL_001f:  call       bool [mscorlib]System.String::op_Equality(string,
                                                                     string)
      IL_0024:  ret
    } // end of method Inner::RestrictByName

    .method public virtual instance class [PLR]PLR.Runtime.RestrictAction 
            get_Restrict() cil managed
    {
      .override [PLR]PLR.Runtime.ProcessBase::get_Restrict
      // Code size       13 (0xd)
      .maxstack  3
      IL_0000:  ldnull
      IL_0001:  ldftn      bool Restrict/Inner::RestrictByName(class 
                           [PLR]PLR.Runtime.IAction)
      IL_0007:  newobj     instance void [PLR]PLR.Runtime.RestrictAction
                           ::.ctor(object, native int)
      IL_000c:  ret
    } // end of method Inner::get_Restrict

    .method public virtual instance void 
            RunProcess() cil managed
    {
      .override [PLR]PLR.Runtime.ProcessBase::RunProcess
      // Code size       84 (0x54)
      .maxstack  10
      .locals init ([0] class [PLR]PLR.Runtime.ChannelSyncAction V_0)
      IL_0000:  ldarg.0
      IL_0001:  call       instance void [PLR]PLR.Runtime.ProcessBase
                           ::InitSetID()
      .try
      {
        IL_0006:  ldarg.0
        IL_0007:  ldstr      "Preparing to sync now..."
        IL_000c:  call       instance void [PLR]PLR.Runtime.ProcessBase
                             ::Debug(string)
        IL_0011:  ldarg.0
        IL_0012:  ldstr      "d"
        IL_0017:  ldarg.0
        IL_0018:  ldc.i4     0x0
        IL_001d:  ldc.i4.1
        IL_001e:  newobj     instance void [PLR]PLR.Runtime.ChannelSyncAction
                             ::.ctor(string, class 
                             [PLR]PLR.Runtime.ProcessBase, int32, bool)
        IL_0023:  stloc.0
        IL_0024:  ldloc.0
        IL_0025:  call       instance void [PLR]PLR.Runtime.ProcessBase
                             ::Sync(class [PLR]PLR.Runtime.IAction)
        IL_002a:  nop
        IL_002b:  nop
        IL_002c:  ldarg.0
        IL_002d:  ldstr      "Turned into 0"
        IL_0032:  call       instance void [PLR]PLR.Runtime.ProcessBase
                             ::Debug(string)
        IL_0037:  leave      IL_004d

      }  // end .try
      catch [PLR]PLR.Runtime.ProcessKilledException 
      {
        IL_003c:  pop
        IL_003d:  ldarg.0
        IL_003e:  ldstr      "Caught ProcessKilledException"
        IL_0043:  call       instance void [PLR]PLR.Runtime.ProcessBase
                             ::Debug(string)
        IL_0048:  leave      IL_004d

      }  // end handler
      IL_004d:  ldarg.0
      IL_004e:  call       instance void [PLR]PLR.Runtime.ProcessBase::Die()
      IL_0053:  ret
    } // end of method Inner::RunProcess

    .method public specialname rtspecialname 
            instance void  .ctor() cil managed
    {
      // Code size       7 (0x7)
      .maxstack  2
      IL_0000:  ldarg.0
      IL_0001:  call       instance void [PLR]PLR.Runtime.ProcessBase
                           ::.ctor()
      IL_0006:  ret
    } // end of method Inner::.ctor

  } // end of class Inner

  .method public specialname rtspecialname 
          instance void  .ctor() cil managed
  {
    // Code size       7 (0x7)
    .maxstack  2
    IL_0000:  ldarg.0
    IL_0001:  call       instance void [PLR]PLR.Runtime.ProcessBase
                         ::.ctor()
    IL_0006:  ret
  } // end of method Restrict::.ctor

  .method public static bool  RestrictByName(class [PLR]PLR.Runtime.IAction 
      A_0) cil managed
  {
    // Code size       37 (0x25)
    .maxstack  3
    .locals init ([0] class [PLR]PLR.Runtime.ChannelSyncAction V_0)
    IL_0000:  ldarg.0
    IL_0001:  isinst     [PLR]PLR.Runtime.ChannelSyncAction
    IL_0006:  brtrue     IL_000d

    IL_000b:  ldc.i4.0
    IL_000c:  ret

    IL_000d:  ldarg.0
    IL_000e:  castclass  [PLR]PLR.Runtime.ChannelSyncAction
    IL_0013:  stloc.0
    IL_0014:  ldloc.0
    IL_0015:  call       instance string [PLR]PLR.Runtime.ChannelSyncAction
                         ::get_Name()
    IL_001a:  ldstr      "a"
    IL_001f:  call       bool [mscorlib]System.String::op_Equality(string,
                                                                   string)
    IL_0024:  ret
  } // end of method Restrict::RestrictByName

  .method public virtual instance class [PLR]PLR.Runtime.RestrictAction 
          get_Restrict() cil managed
  {
    .override [PLR]PLR.Runtime.ProcessBase::get_Restrict
    // Code size       13 (0xd)
    .maxstack  3
    IL_0000:  ldnull
    IL_0001:  ldftn      bool Restrict::RestrictByName(class 
                         [PLR]PLR.Runtime.IAction)
    IL_0007:  newobj     instance void [PLR]PLR.Runtime.RestrictAction
                         ::.ctor(object, native int)
    IL_000c:  ret
  } // end of method Restrict::get_Restrict

  .method public virtual instance void  RunProcess() cil managed
  {
    .override [PLR]PLR.Runtime.ProcessBase::RunProcess
    // Code size       103 (0x67)
    .maxstack  10
    .locals init ([0] class [PLR]PLR.Runtime.ChannelSyncAction V_0,
             [1] class [PLR]PLR.Runtime.ProcessBase V_1)
    IL_0000:  ldarg.0
    IL_0001:  call       instance void [PLR]PLR.Runtime.ProcessBase
                         ::InitSetID()
    .try
    {
      IL_0006:  ldarg.0
      IL_0007:  ldstr      "Preparing to sync now..."
      IL_000c:  call       instance void [PLR]PLR.Runtime.ProcessBase
                           ::Debug(string)
      IL_0011:  ldarg.0
      IL_0012:  ldstr      "a"
      IL_0017:  ldarg.0
      IL_0018:  ldc.i4     0x0
      IL_001d:  ldc.i4.1
      IL_001e:  newobj     instance void [PLR]PLR.Runtime.ChannelSyncAction
                           ::.ctor(string, class [PLR]PLR.Runtime.ProcessBase
                           , int32, bool)
      IL_0023:  stloc.0
      IL_0024:  ldloc.0
      IL_0025:  call       instance void [PLR]PLR.Runtime.ProcessBase
                           ::Sync(class [PLR]PLR.Runtime.IAction)
      IL_002a:  nop
      IL_002b:  newobj     instance void Restrict/Inner::.ctor()
      IL_0030:  stloc.1
      IL_0031:  ldloc.1
      IL_0032:  ldarg.0
      IL_0033:  call       instance void [PLR]PLR.Runtime.ProcessBase
                           ::set_Parent(class [PLR]PLR.Runtime.ProcessBase)
      IL_0038:  ldloc.1
      IL_0039:  ldarg.0
      IL_003a:  call       instance valuetype [mscorlib]System.Guid 
                           [PLR]PLR.Runtime.ProcessBase::get_SetID()
      IL_003f:  call       instance void [PLR]PLR.Runtime.ProcessBase
                           ::set_SetID(valuetype [mscorlib]System.Guid)
      IL_0044:  ldloc.1
      IL_0045:  call       instance void [PLR]PLR.Runtime.ProcessBase::Run()
      IL_004a:  leave      IL_0060

    }  // end .try
    catch [PLR]PLR.Runtime.ProcessKilledException 
    {
      IL_004f:  pop
      IL_0050:  ldarg.0
      IL_0051:  ldstr      "Caught ProcessKilledException"
      IL_0056:  call       instance void [PLR]PLR.Runtime.ProcessBase
                           ::Debug(string)
      IL_005b:  leave      IL_0060

    }  // end handler
    IL_0060:  ldarg.0
    IL_0061:  call       instance void [PLR]PLR.Runtime.ProcessBase::Die()
    IL_0066:  ret
  } // end of method Restrict::RunProcess

} // end of class Restrict

\end{cil}

\section{Relabel}

	\begin{verbatim}
  Relabel = ( a . (d . 0)[dnew/d] )[anew/a]
	\end{verbatim}
	
	is compiled as follows:

\begin{cil}

.class public auto ansi beforefieldinit Relabel
       extends [PLR]PLR.Runtime.ProcessBase
{
  .class auto ansi nested public beforefieldinit Inner
         extends [PLR]PLR.Runtime.ProcessBase
  {
    .method public static class [PLR]PLR.Runtime.IAction 
            RelabelAction(class [PLR]PLR.Runtime.IAction A_0) cil managed
    {
      // Code size       59 (0x3b)
      .maxstack  4
      .locals init ([0] class [PLR]PLR.Runtime.ChannelSyncAction V_0)
      IL_0000:  ldarg.0
      IL_0001:  isinst     [PLR]PLR.Runtime.ChannelSyncAction
      IL_0006:  brtrue     IL_000d

      IL_000b:  ldarg.0
      IL_000c:  ret

      IL_000d:  ldarg.0
      IL_000e:  castclass  [PLR]PLR.Runtime.ChannelSyncAction
      IL_0013:  stloc.0
      IL_0014:  ldloc.0
      IL_0015:  call       instance string [PLR]PLR.Runtime.ChannelSyncAction
                           ::get_Name()
      IL_001a:  ldstr      "d"
      IL_001f:  call       bool [mscorlib]System.String::op_Equality(string,
                                                                     string)
      IL_0024:  brfalse    IL_0039

      IL_0029:  ldloc.0
      IL_002a:  ldstr      "dnew"
      IL_002f:  call       instance void [PLR]PLR.Runtime.ChannelSyncAction
                           ::set_Name(string)
      IL_0034:  br         IL_0039

      IL_0039:  ldloc.0
      IL_003a:  ret
    } // end of method Inner::RelabelAction

    .method public virtual instance class [PLR]PLR.Runtime.PreProcessAction 
            get_PreProcess() cil managed
    {
      .override [PLR]PLR.Runtime.ProcessBase::get_PreProcess
      // Code size       13 (0xd)
      .maxstack  3
      IL_0000:  ldnull
      IL_0001:  ldftn      class [PLR]PLR.Runtime.IAction Relabel/Inner
                           ::RelabelAction(class [PLR]PLR.Runtime.IAction)
      IL_0007:  newobj     instance void [PLR]PLR.Runtime.PreProcessAction
                           ::.ctor(object, native int)
      IL_000c:  ret
    } // end of method Inner::get_PreProcess

    .method public virtual instance void 
            RunProcess() cil managed
    {
      .override [PLR]PLR.Runtime.ProcessBase::RunProcess
      // Code size       84 (0x54)
      .maxstack  10
      .locals init ([0] class [PLR]PLR.Runtime.ChannelSyncAction V_0)
      IL_0000:  ldarg.0
      IL_0001:  call       instance void [PLR]PLR.Runtime.ProcessBase
                           ::InitSetID()
      .try
      {
        IL_0006:  ldarg.0
        IL_0007:  ldstr      "Preparing to sync now..."
        IL_000c:  call       instance void [PLR]PLR.Runtime.ProcessBase
                             ::Debug(string)
        IL_0011:  ldarg.0
        IL_0012:  ldstr      "d"
        IL_0017:  ldarg.0
        IL_0018:  ldc.i4     0x0
        IL_001d:  ldc.i4.1
        IL_001e:  newobj     instance void [PLR]PLR.Runtime.ChannelSyncAction
                             ::.ctor(string, class 
                             [PLR]PLR.Runtime.ProcessBase, int32, bool)
        IL_0023:  stloc.0
        IL_0024:  ldloc.0
        IL_0025:  call       instance void [PLR]PLR.Runtime.ProcessBase
                             ::Sync(class [PLR]PLR.Runtime.IAction)
        IL_002a:  nop
        IL_002b:  nop
        IL_002c:  ldarg.0
        IL_002d:  ldstr      "Turned into 0"
        IL_0032:  call       instance void [PLR]PLR.Runtime.ProcessBase
                             ::Debug(string)
        IL_0037:  leave      IL_004d

      }  // end .try
      catch [PLR]PLR.Runtime.ProcessKilledException 
      {
        IL_003c:  pop
        IL_003d:  ldarg.0
        IL_003e:  ldstr      "Caught ProcessKilledException"
        IL_0043:  call       instance void [PLR]PLR.Runtime.ProcessBase
                             ::Debug(string)
        IL_0048:  leave      IL_004d

      }  // end handler
      IL_004d:  ldarg.0
      IL_004e:  call       instance void [PLR]PLR.Runtime.ProcessBase::Die()
      IL_0053:  ret
    } // end of method Inner::RunProcess

    .method public specialname rtspecialname 
            instance void  .ctor() cil managed
    {
      // Code size       7 (0x7)
      .maxstack  2
      IL_0000:  ldarg.0
      IL_0001:  call       instance void [PLR]PLR.Runtime.ProcessBase
                           ::.ctor()
      IL_0006:  ret
    } // end of method Inner::.ctor

  } // end of class Inner

  .method public specialname rtspecialname 
          instance void  .ctor() cil managed
  {
    // Code size       7 (0x7)
    .maxstack  2
    IL_0000:  ldarg.0
    IL_0001:  call       instance void [PLR]PLR.Runtime.ProcessBase
                         ::.ctor()
    IL_0006:  ret
  } // end of method Relabel::.ctor

  .method public static class [PLR]PLR.Runtime.IAction 
          RelabelAction(class [PLR]PLR.Runtime.IAction A_0) cil managed
  {
    // Code size       59 (0x3b)
    .maxstack  4
    .locals init ([0] class [PLR]PLR.Runtime.ChannelSyncAction V_0)
    IL_0000:  ldarg.0
    IL_0001:  isinst     [PLR]PLR.Runtime.ChannelSyncAction
    IL_0006:  brtrue     IL_000d

    IL_000b:  ldarg.0
    IL_000c:  ret

    IL_000d:  ldarg.0
    IL_000e:  castclass  [PLR]PLR.Runtime.ChannelSyncAction
    IL_0013:  stloc.0
    IL_0014:  ldloc.0
    IL_0015:  call       instance string [PLR]PLR.Runtime.ChannelSyncAction
                         ::get_Name()
    IL_001a:  ldstr      "a"
    IL_001f:  call       bool [mscorlib]System.String::op_Equality(string,
                                                                   string)
    IL_0024:  brfalse    IL_0039

    IL_0029:  ldloc.0
    IL_002a:  ldstr      "anew"
    IL_002f:  call       instance void [PLR]PLR.Runtime.ChannelSyncAction
                         ::set_Name(string)
    IL_0034:  br         IL_0039

    IL_0039:  ldloc.0
    IL_003a:  ret
  } // end of method Relabel::RelabelAction

  .method public virtual instance class [PLR]PLR.Runtime.PreProcessAction 
          get_PreProcess() cil managed
  {
    .override [PLR]PLR.Runtime.ProcessBase::get_PreProcess
    // Code size       13 (0xd)
    .maxstack  3
    IL_0000:  ldnull
    IL_0001:  ldftn      class [PLR]PLR.Runtime.IAction Relabel
                         ::RelabelAction(class [PLR]PLR.Runtime.IAction)
    IL_0007:  newobj     instance void [PLR]PLR.Runtime.PreProcessAction
                         ::.ctor(object, native int)
    IL_000c:  ret
  } // end of method Relabel::get_PreProcess

  .method public virtual instance void  RunProcess() cil managed
  {
    .override [PLR]PLR.Runtime.ProcessBase::RunProcess
    // Code size       103 (0x67)
    .maxstack  10
    .locals init ([0] class [PLR]PLR.Runtime.ChannelSyncAction V_0,
             [1] class [PLR]PLR.Runtime.ProcessBase V_1)
    IL_0000:  ldarg.0
    IL_0001:  call       instance void [PLR]PLR.Runtime.ProcessBase
                         ::InitSetID()
    .try
    {
      IL_0006:  ldarg.0
      IL_0007:  ldstr      "Preparing to sync now..."
      IL_000c:  call       instance void [PLR]PLR.Runtime.ProcessBase
                           ::Debug(string)
      IL_0011:  ldarg.0
      IL_0012:  ldstr      "a"
      IL_0017:  ldarg.0
      IL_0018:  ldc.i4     0x0
      IL_001d:  ldc.i4.1
      IL_001e:  newobj     instance void [PLR]PLR.Runtime.ChannelSyncAction
                           ::.ctor(string, class 
                           [PLR]PLR.Runtime.ProcessBase, int32, bool)
      IL_0023:  stloc.0
      IL_0024:  ldloc.0
      IL_0025:  call       instance void [PLR]PLR.Runtime.ProcessBase
                           ::Sync(class [PLR]PLR.Runtime.IAction)
      IL_002a:  nop
      IL_002b:  newobj     instance void Relabel/Inner::.ctor()
      IL_0030:  stloc.1
      IL_0031:  ldloc.1
      IL_0032:  ldarg.0
      IL_0033:  call       instance void [PLR]PLR.Runtime.ProcessBase
                           ::set_Parent(class [PLR]PLR.Runtime.ProcessBase)
      IL_0038:  ldloc.1
      IL_0039:  ldarg.0
      IL_003a:  call       instance valuetype [mscorlib]System.Guid 
                           [PLR]PLR.Runtime.ProcessBase::get_SetID()
      IL_003f:  call       instance void [PLR]PLR.Runtime.ProcessBase
                           ::set_SetID(valuetype [mscorlib]System.Guid)
      IL_0044:  ldloc.1
      IL_0045:  call       instance void [PLR]PLR.Runtime.ProcessBase::Run()
      IL_004a:  leave      IL_0060

    }  // end .try
    catch [PLR]PLR.Runtime.ProcessKilledException 
    {
      IL_004f:  pop
      IL_0050:  ldarg.0
      IL_0051:  ldstr      "Caught ProcessKilledException"
      IL_0056:  call       instance void [PLR]PLR.Runtime.ProcessBase
                           ::Debug(string)
      IL_005b:  leave      IL_0060

    }  // end handler
    IL_0060:  ldarg.0
    IL_0061:  call       instance void [PLR]PLR.Runtime.ProcessBase::Die()
    IL_0066:  ret
  } // end of method Relabel::RunProcess

} // end of class Relabel

	\end{cil}



\backmatter

\chaptermark{Bibliography}
\renewcommand{\sectionmark}[1]{\markright{#1}}
\sectionmark{Bibliography}

%%%%%%%BIBLIOGRAPHY INCLUDE%%%%%%%%%%%%%%%%%%%%%%%%%%%%%%%%%%%%%%%%%%%%%%

\bibliography{Bibliography}    % Bibliography
\bibliographystyle{plain}


\end{document}
