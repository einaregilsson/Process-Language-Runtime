\chapter{KLAIM Implementation}\label{ch:klaim}

	In this chapter we look at the second language implemented using the PLR, 
	the Kernel Language for Agents Interaction and Mobility, or KLAIM. Unlike 
	the CCS implementation, the PLR does not contain all the constructs needed 
	for an implementation of KLAIM, so this chapter is a case study in how the 
	PLR can be extended for use with other languages than CCS.
	
\section{Overview}
	
	KLAIM is a language introduced in 1998 in \cite{klaim}, it's main purpose is 
	to model mobile agents in a distributed environment. It derives many of its 
	constructs from process algebra, but it is also heavily influenced by 
	Linda \cite{linda1, linda2}, a distributed computing solution from which it 
	borrows the concept of \textit{tuple spaces}. A tuple space is a collection 
	of tuples of data, these tuples can be read and removed from the tuple space 
	or new tuples can be added to it. Selecting tuples to read is done by means 
	of pattern matching. In KLAIM, processes run in different 
	\textit{localities}, each of which contains zero or more processes 
	and its own tuple space. The actions performed by the processes input, 
	output and read tuples of data from these tuple spaces. Unlike CCS, where 
	processes synchronized with each other through channels, in KLAIM the way 
	processes communicate is strictly by adding and retrieving data from tuple 
	spaces of different localities.
	
	The concept of data stores (or tuple spaces) and actions which operate on 
	them is something that is not built into the PLR. KLAIM does however contain 
	many constructs which are a part of the PLR, namely parallel composition, 
	non-deterministic choice and action prefixing (albeit with different type of 
	actions than those built into the PLR). It also contains a fairly common 
	process algebra construct which is not part of the PLR, 
	\textit{replication}. These additional features make KLAIM an ideal 
	candidate to use as a test of how easily the PLR can be extended to 
	accommodate other languages than CCS.
	
	\subsection{The implemented subset of KLAIM}
	
	The implementation of KLAIM described in this chapter is only for a subset 
	of the full language. In particular the \textbf{in}, \textbf{out} and 
	\textbf{read} actions are implemented while the \textbf{eval} and 
	\textbf{newloc} actions are not. The implementation also does not support 
	process constants, processes are defined directly in the \textit{net} and so 
	cannot turn into other processes. The subset of KLAIM implemented here is 
	the same as that described in \cite{aspectk, giordano}, both of which are
	projects developed at the Technical University of Denmark that focus on an 
	aspect-oriented version of KLAIM. Section~\ref{sec:klaim_syntax} only shows 
	syntax and semantic for the subset being used, for a description of the full 
	KLAIM language we refer to \cite{klaim}. 
	
	KLAIMS stated purpose is to be a programming language for programming mobile 
	agents in a distributed environment. However in this implementation all 
	processes in the system are running on the same computer, and in the same 
	(operating system) process. The distributed part of it is purely conceptual. 
	As such, this implementation can be seen more as a simulation of how KLAIM 
	works rather than an implementation ready to be used for actual 
	applications. Examples of other implementations of KLAIM are given in the 
	next section.
		
	\subsection{Other implementations of KLAIM}
		
	Noteworthy implementations of KLAIM include KLAVA\cite{klava}, which is a 
	Java library representing the KLAIM constructs as Java classes, and X-Klaim 
	\cite{xklaim}, a compiler for a superset of KLAIM whose output is Java 
	source code that uses the KLAVA library. A different approach is taken in 
	\cite{giordano} where a virtual machine is developed specifically for 
	AspectK, an aspect-oriented version of KLAIM. A further explanation of these 
	implementations and a comparison with the PLR is given in 
	Section~\ref{sec:related_work}. 

\section{Formal syntax and semantics}\label{sec:klaim_syntax}
	
	A KLAIM net is made up of located processes and located tuples. We use $N$ 
	for a net and the composition of nodes of the net is given with the 
	$\mid\mid$ operator. 
	
	Table~\ref{tab:klaim_syntax} shows an overview of the syntax.

\begin{table}[t]
\begin{ARRAY}{r c l l}
		  N & ::= & N_1 \mid\mid\ N_2 & \mathrm{(Net\ composition)}\vspace{3pt}\\
			    & \mid & l::\langle t \rangle & \mathrm{(Located\ tuple)}\vspace{3pt}\\
			    & \mid & l::P & \mathrm{(Located\ process)}\vspace{10pt}\\

			P & ::= & \mathbf{nil} & \mathrm{(Nil\ process)}\vspace{3pt}\\
			    & \mid & a . P & \mathrm{(Action\ prefixing)}\vspace{3pt} \\
			    & \mid & P \mid Q & \mathrm{(Parallel\ composition)}\vspace{3pt}\\
			    & \mid & P + Q & \mathrm{(Nondeterministic\ choice)}\vspace{3pt}\\
			    & \mid & *P  & \mathrm{(Replication)}\vspace{10pt}\\

			a & ::= & \mathbf{out}(t)@\ell\ & \mathrm{(Add\ tuple\ to\ the\ \ell\  tuple\ space)}\vspace{3pt}\\
			
			    & \mid\ & \mathbf{in}(t)@\ell\ & \mathrm{(Remove\ tuple\ from\ the\ \ell\ tuple\ space)}\vspace{3pt}\\

			    & \mid\ & \mathbf{read}(t)@\ell\ & \mathrm{(Read\ tuple\ from\ the\ \ell\ tuple\ space)}\vspace{10pt}\\
			    
			t & ::= & e & \mathrm{(Expression)} \vspace{3pt}\\
			  & \mid & \ell & \mathrm{(Locality\ constant\ or\ variable)} \vspace{3pt}\\
			  & \mid & x & \mathrm{(Variable)} \vspace{3pt}\\
			  & \mid & !x & \mathrm{(The\ binding\ of\ variable\ x)} \vspace{3pt}\\
			  & \mid & t_1,t_2 & \mathrm{(Sequence\ of\ tuple\ elements)} \vspace{3pt}\\
\end{ARRAY}
\caption{KLAIM Nets and processes syntax}
\label{tab:klaim_syntax}
\end{table}


%%%%%%%%%%%%%%%%%%%%%%%%%%%%%%%%%%
%%Structure equivalence
%%%%%%%%%%%%%%%%%%%%%%%%%%%%%%%%%%
\begin{table}[t]
\begin{ARRAY}{l@{\qquad\qquad}l}
l:: P_1 \ppar P_2 \equiv l::P_1 \netpar l::P_2 &
l ::\ * P  \equiv l ::\ P \ppar\ * P
\\[2ex]
\Inference{N_1 \equiv N_2}{N \netpar N_1 \equiv N \netpar N_2}
\end{ARRAY}
\caption{KLAIM Structural Congruence.}
\label{tab:klaim_structural_congruence}
\end{table}
%%%%%%%%%%%%%%%%%%%%%%%%%%%%%%%%%%

%%Semantics
%%%%%%%%%%%%%%%%%%%%%%%%%%%%%%%%%%
\begin{table}[t]
\begin{ARRAY}{ll}
%%Out
l_s::\mathbf{out}(\veck\YZl)@l_0.P+\cdots \rightarrow
l_s::P \netpar l_0::\langle \veck\YZl \rangle
%%In
\\[1ex]
%\begin{array}{r}
l_s::\mathbf{in}(\veck\Lat)@l_0.P+ \cdots \netpar
l_0::\langle \veck\YZl \rangle \rightarrow l_s::P
\theta & \hbox{if $match(\veck\Lat;\veck\YZl)= \theta$}
%\end{array}
%%Read
\\[1ex]
%\begin{array}{r}
l_s::\mathbf{read}(\veck\Lat)@l_0.P+ \cdots \netpar
l_0::\langle \veck\YZl \rangle \rightarrow  l_s::P
\theta \netpar l_0::\langle \veck\YZl \rangle & 
\hbox{if $match(\veck\Lat;\veck\YZl)= \theta$}
%\end{array}
\\[2ex]
\Inference{ N_1 \rightarrow N_1'}{ N_1 \netpar
N_2 \rightarrow  N_1' \netpar N_2}
\hfill
\Inference{N \equiv N'\quad  N' \rightarrow N'' \quad N'' \equiv
N'''}{ N \rightarrow N'''}
\end{ARRAY}
\caption{KLAIM Reaction Semantics (on closed nets).}
\label{tab:klaim_structural_congruence}
\end{table}
%%%%%%%%%%%%%%%%%%%%%%%%%%%%%%%%%%
\section{Implementation}

\subsection{Class overview}
	
\section{Summary}
