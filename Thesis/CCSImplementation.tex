\chapter{CCS Implementation}

In this chapter we look at the Calculus of Communicating System process language, its history, formal syntax and semantics, and its implementation with the PLR as the back end compiler. CCS was the first language implemented using the PLR, and it was implemented at the same time as the PLR was being designed. For that reason the PLR contains all the main constructs of CCS so that the CCS specific parts of the implementation are fairly small and mostly have to do with the front end, that is the lexer and parser. A description of a another language, one that extends the PLR further, is found in Chapter~\ref{ch:klaim}.

\section{History}

\section{Formal syntax and semantics}\label{sec:ccs_syntax}

CCS can be described by the following syntax.

\begin{itemize}
		\item Let $\mathcal{A}$ be the set of \textit{channel names}, we use 
		$\alpha$ as a typical member of this set in demonstrations.
	
		\item $\overline{\mathcal{A}} := \{\overline{a}\mid a \in \mathcal{A}\}$
		denotes the set of co-names. For each channel name there is a
		corresponding co-name. Channel name represents input, co-name represents 
		output.
	
		\item Let $\mathcal{L} := \mathcal{A} \cup \overline{\mathcal{A}}$ be the 
		set of \textit{labels}.
		
		\item \textsf{Act} $:= \mathcal{L} \cup {\tau}$ is the set of 
		\textit{actions}, where $\tau$ denotes the silent (or unobservable) action.
		
		\item Let $\mathcal{K}$ be a set of process identifiers.
		
		\item The set $\mathcal{P}$ of process expressions is defined by the 
		following syntax:

		\begin{tabular}{r c l l}
			$P$ & ::= & 0 & (Nil process)\\
			    & $\mid$ & $\alpha$ . $P$ & (Action prefixing) \\
			    & $\mid$ & $K$ & (Invoking a process)\\
			    & $\mid$ & $P \mid\ Q$ & (Parallel composition)\\
			    & $\mid$ & $P$ + $Q$ & (Nondeterministic choice)\\
			    & $\mid$ & $P$[$f$] & (Relabelling)\\
			    & $\mid$ & $P \backslash\ L$ & (Restriction)
		\end{tabular}
		
		where 
		
		\begin{itemize}
			\item $P,Q$ are processes in $\mathcal{P}$
			\item $\alpha$ is an action in \textsf{Act}
			\item $K$ is a process name from $\mathcal{K}$
			\item $f$ : \textsf{Act} $\rightarrow$ \textsf{Act} is a relabelling 
			function satisfying the constraints $f(\tau) = \tau$ and 
			$f(\overline{a}) = \overline{f(a)}$ for each label \textit{a}.
			\item $L$ is a set of labels from $\mathcal{L}$
		\end{itemize} 
		
		\item We assume that the behavior of each process name $K \in \mathcal{K}$ 
		is given by a defining equation
		
		\begin{center}$K \defeq P$\end{center}
		
		where $P \in \mathcal{P}$.

	\end{itemize}
	To avoid the use of too many parantheses in writing CCS expressions 
	the convention is used that operators have decreasing binding strength in 
	the following order: restriction and relabelling (the tightest binding), 
	action prefixing, parallel composition and summation. For example, the 
	expression $a.0 \mid b.P\ \backslash\ L + c.0$ stands for
		
	\begin{center}$((a.0) \mid (b.(P\ \backslash\ L))) + (c.0)$\end{center}
		
	For an informal description of each of the constructs(action-prefixing, 
	parallel composition, non-deterministic choice, restrictino and relabelling) 
	we refer to Section~\ref{sec:common_constructs} where such a description was 
	already given. For a formal description, Table~\ref{tab:ccs_sos} shows the 
	structural operational semantics of CCS.


\begin{table}[t]
\begin{ARRAY}{c}
%%Out
	\mathrm{ACT}\ \ \ \Inference{                              }
	                            {\alpha.P \infarrow{\alpha} P} \vspace{10pt}\\
	\mathrm{SUM}_j\ \ \ \Inference{P_j \infarrow{\alpha} P_j'}
	                              {\sum_{i\in I}P_i \infarrow{\alpha} P_j'}\ \ \mathrm{where}\ j \in I\vspace{10pt}\\
	\mathrm{COM1}\ \ \ \Inference{P \infarrow{\alpha} P'}
	                             {P \mid Q \infarrow{\alpha} P' \mid Q} \vspace{10pt}\\
	\mathrm{COM2}\ \ \ \Inference{Q \infarrow{\alpha} Q'}
	                             {P \mid Q \infarrow{\alpha} P \mid Q'} \vspace{10pt}\\
	\mathrm{COM3}\ \ \ \Inference{P \infarrow{\alpha} P'\ \ \ \ Q \infarrow{\alpha} Q'}
	                             {P \mid Q \infarrow{\tau} P' \mid Q'} \vspace{10pt}\\
	\mathrm{RES}\ \ \ \Inference{P \infarrow{\alpha} P'}
	                              {P\ \backslash\ L \infarrow{\alpha} P'\  \backslash\ L}\ \ \mathrm{where}\ \alpha, \overline{\alpha} \notin L\vspace{10pt}\\
	\mathrm{REL}\ \ \ \Inference{P \infarrow{\alpha} P'}
	                             {P[f] \infarrow{f(\alpha)} P'[f]} \vspace{10pt}\\
	\mathrm{CON}\ \ \ \Inference{P \infarrow{\alpha} P'}
	                             {K \infarrow{\alpha} P'}\ \ \mathrm{where}\ K \defeq P

\end{ARRAY}
\caption{CCS Structural Operational Semantics}
\label{tab:ccs_sos}
\end{table}
	

\section{Value passing syntax and semantics}\label{ccs_value_syntax}

\section{Implementation}

	The CCS implementation was written in C\# using Visual Studio 2008 as the 
	development environment. Since the PLR includes abstract syntax tree nodes 
	for all constructs of CCS, no additional nodes were created specifically for 
	the CCS implementation. There are only a handful of classes used, below is a 
	short summary of each one.

\subsection{Class overview}
	\begin{itemize}
	
	\item \textsf{Program} is the entry point into the compiler. It contains a 
	\textsf{Main} method that parses command line parameters, validates 
	parameters such as input and output file names and then creates a 
	\textsf{Parser} object. It then calls the parser's \textsf{Parse} method, 
	receives a PLR \textsf{ProcessSystem} node and calls \textsf{Compile} on it. 
	
	\item \textsf{Scanner} is the lexer class whose responsibility it is to 
	tokenize a CCS source code file into valid CCS terminals. 
	Listing~\ref{lst:ccsterminals} shows the more complicated terminals of CCS 
	defined by regular expressions. The simpler terminals, who are just string 
	constants, are given directly in quotes in the parser definition. The 
	scanner is implemented as a finite state automaton.
	
  \item \textsf{Parser} is a recursive-descent parser for CCS. It uses the 
  tokens generated by \textsf{Scanner} and applies a number of 
  \textsf{productions} to recognize the language. The Extended Backus-Naur 
  Form (EBNF) of the productions is given in Listing~\ref{lst:ccs_ebnf}. The 
  parser constructs a PLR abstract syntax tree as it parses, if the parsing is 
  without errors the syntax tree can then be compiled. Both the 
  \textsf{Parser} and \textsf{Scanner} are generated by the parser generator 
  Coco/R \cite{cocor}. It takes as input an EBNF specification of the language 
  interspersed with C\# source code and from that it generates the 
  \textsf{Scanner} and \textsf{Parser} classes. 
	
	\item \textsf{ParserTest} contains unit tests for the parser and scanner, to 
	be run with the NUnit unit testing framework.
	
	\end{itemize}
	\begin{figure}
\lstset{showtabs=false,showspaces=false,showstringspaces=false}
\begin{lstlisting}[caption=Terminals of CCS scanner,label=lst:ccsterminals,frame=trbl,basicstyle=\scriptsize\ttfamily,showtabs=false,showspaces=false]
  PROCNAME   = [A-Z][A-Za-z0-9]*
  LCASEIDENT = [a-z][A-Za-z0-9]*
  CLASSNAME  = [A-Z][A-Za-z0-9]*(\\.[A-Z][A-Za-z0-9]*)*
  OUTACTION  = _[a-z][A-Za-z0-9]*_
  METHOD     = :[a-zA-Z][A-Za-z0-9]*
  NUMBER     = [0-9]+
  STRING     = "[^"]*"'
  \end{lstlisting}
	\end{figure}
	
	\begin{figure}
\lstset{showtabs=false,showspaces=false,showstringspaces=false}
\begin{lstlisting}[caption=EBNF Productions of CCS parser,frame=trbl,label=lst:ccs_ebnf,basicstyle=\scriptsize\ttfamily,showtabs=false,showspaces=false]
CCS = { ClassImport } ProcessDefinition { ProcessDefinition } 

ClassImport = "use" CLASSNAME 

ProcessDefinition = 
  PROCNAME [ "(" LCASEIDENT {"," LCASEIDENT } ")" ] "=" Process 

Process = NonDeterministicChoice

NonDeterministicChoice = ParallelComposition { "+" ParallelComposition }

ParallelComposition = ActionPrefix { "|" ActionPrefix } .

ActionPrefix =
  { Action "." }
  (
    "(" Process ")"
    | "0"
    | ProcessConstantInvoke
    | BranchProcess
  )
  [ Relabelling ]
  [ Restriction ]

BranchProcess = "if" Expression "then" Process "else" Process .

ProcessConstantInvoke = 
  PROCNAME [ "(" ArithmeticExpression {"," ArithmeticExpression } ")" ]

Action =
  LCASEIDENT [ "(" LCASEIDENT { "," LCASEIDENT } ")" ]
  | OUTACTION [ "(" ArithmeticExpression { "," ArithmeticExpression } ")" ]
  | METHOD "(" [ CallParam { "," CallParam } ")"

CallParam = ArithmeticExpression | STRING

Relabelling =
  "[" METHOD "]"
  | "[" LCASEIDENT "/" LCASEIDENT { "," LCASEIDENT "/" LCASEIDENT } "]"

Restriction = 
  "\" 
  (
    LCASEIDENT 
    | "{" LCASEIDENT {"," LCASEIDENT } "}"
    | METHOD
  )        

Expression = OrTerm { "or" OrTerm }

OrTerm = AndTerm { "and" AndTerm }

AndTerm = RelationalTerm { "xor" RelationalTerm }

RelationalTerm = 
  ArithmeticExpression [ ("=="|"!="|">"|">="|"<"|"<=") ArithmeticExpression ]
		
ArithmeticExpression = PlusMinusTerm { ("+" | "-") PlusMinusTerm }

PlusMinusTerm = UnaryMinusTerm { ("*"|"/"|"\%") UnaryMinusTerm }

UnaryMinusTerm =
  [ "-" ]                                   
  (
    "(" ArithmeticExpression ")"
    | NUMBER 
    | "0"
    | "true"
    | "false"
    | LCASEIDENT
  )
	\end{lstlisting}
	\end{figure}
	
\subsection{Extensions to CCS syntax}

	Section~\ref{sec:ccs_syntax} shows the formal syntax for CCS, it however 
	does not account for integrating with the .NET environment to allow 
	arbitrary method calls to be made as actions and .NET methods to be used as 
	relabelling functions and/or restriction functions. The EBNF specification 
	for the parser which shows all allowed syntax can be seen in 
	Listing~\ref{lst:ccs_ebnf}, but to quickly summarize the changes from formal 
	CCS, they are as follows:
	
	\begin{itemize}
		
		\item A CCS source code file can start with one or more \texttt{use} 
		statements, which consist of the token \texttt{use} followed by the fully 
		qualified name of a .NET class. E.g.
		
		\begin{center}\texttt{use System.Console\ \ \ \ \ \ \ \ \ \ \ \ \\\texttt{use MyCustomNamespace.MyClass}}\end{center}
		
		These referenced classes can be in the .NET core library, 
		\texttt{mscorlib}, or in any arbitrary .NET assembly. During compilation 
		the filenames of assemblies containing classes used in the source code 
		must be passed to the compiler so that it knows where to look for 
		classnames found in \texttt{use} statements.
	
		\item Actions can be calls to .NET methods in addition to synchronization 
		and value passing on channels. A .NET method call consists of a colon 
		followed by the method name and parantheses around expressions passed as 
		parameters to the method, e.g. 
		
		\begin{center}\texttt{P = :Print(``Hello world'') . 0}\end{center}
		
		At compile time the PLR resolves which class the method belongs to by 
		looking at the classes imported with \texttt{use} statements and 
		inspecting their methods. If more than one imported class has a candidate 
		method an exception is thrown. The methods must be static, if instance 
		methods are to be used it is necessary to write a static wrapper method 
		around them that creates an instance of the object in question and calls 
		the instance method.
		
		\item Relabelling functions can be specified as .NET methods in addition 
		to be constant replacements. To use a .NET method for relabelling its name 
		prefixed with a colon is put inside the square brackets that usually 
		define relabelling functions in CCS, e.g. \texttt{[:MyRelabelMethod]}. The 
		method is resolved to an imported class at compile time and is required to 
		be a method that takes a single parameter, an instance of the 
		\textsf{IAction} interface from the PLR runtime library.
		
		\item Restrictions functions can be specified as .NET methods in much the 
		same way as relabelling functions and must resolve to a method that takes 
		a \textsf{IAction} instance as a parameter. An example of a process which 
		uses a .NET method for restriction could be \texttt{(a . 0) \ :MyRestrict}.

	\end{itemize}