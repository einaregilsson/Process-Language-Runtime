\chapter{Generated bytecode}

  For those that are interested in the actual code generation that takes place 
  in the PLR, we now look at a sample system and its generated code. The 
  system we look at is the following:
  
  \begin{verbatim}
  use PLR.Runtime.BuiltIns

  StartProc = ActionPrefix | NonDeterministicChoice 

  ActionPrefix = a .  0

  ValuePassSend(x) = _a_(x+3 /:Rand(2)) . 0
  ValuePassReceive = a(y) . 0

  NonDeterministicChoice = _a_ . 0 + NDC2
  NDC2 = b . 0
 
  ParallelComposition = a . 0 | PC2 | 0 
  PC2 = b . 0

  MethodCall = :Print("Hello") . 0

  Restrict = ( a . (d . 0)\ d ) \{a}

  Relabel = ( a . (d . 0)[dnew/d] )[anew/a]  
  \end{verbatim}

  This system obviously is not a model of anything special, it is simply 
  composed to show the available features of the PLR, and the processes are 
  named accordingly, e.g. ActionPrefix and Restrict. In this appendix each 
  activity type will have its own section, where the relevant process is first 
  shown in CCS and then the generated bytecode is shown. The text 
  representation of the bytecode was generated using the intermediate language 
  disassembler tool from Microsoft, \textit{ILDASM}.
  
  \section{Header}
  
  The header of the CIL file is as follows:

	\begin{cil}}

//  Microsoft (R) .NET Framework IL Disassembler.  Version 3.5.21022.8
//  Copyright (c) Microsoft Corporation.  All rights reserved.

// Metadata version: v2.0.50727
.assembly extern PLR
{
  .ver 1:0:0:0
}
.assembly extern mscorlib
{
  .publickeytoken = (B7 7A 5C 56 19 34 E0 89 )        // .z\V.4..
  .ver 2:0:0:0
}
.assembly disassemble.exe
{
  .hash algorithm 0x00008004
  .ver 0:0:0:0
}
.module disassemble.exe
// MVID: {A23EA4EB-468E-4E2B-A4CA-AA68D8C2320E}
.imagebase 0x00400000
.file alignment 0x00000200
.stackreserve 0x00100000
.subsystem 0x0003       // WINDOWS_CUI
.corflags 0x00000001    //  ILONLY
// Image base: 0x00970000

\end{cil}

  \section{Main method}
  
	The main method of the executable, which is the entry point, is as follows:
	
	\begin{cil}
// ================== GLOBAL METHODS =========================

.method public static int32  Main() cil managed
{
  .entrypoint
  // Code size       18 (0x12)
  .maxstack  1
  .locals init (class StartProc V_0)
  IL_0000:  newobj     instance void StartProc::.ctor()
  IL_0005:  stloc.0
  IL_0006:  call       class [PLR]PLR.Runtime.Scheduler 
                       [PLR]PLR.Runtime.Scheduler::get_Instance()
  IL_000b:  call       instance void [PLR]PLR.Runtime.Scheduler::Run()
  IL_0010:  ldc.i4.0
  IL_0011:  ret
} // end of global method Main


// =============================================================

	\end{cil}	  
  \section{StartProc}
	\begin{verbatim}
  StartProc = ActionPrefix | NonDeterministicChoice 
	\end{verbatim}
	
	is compiled as follows:
	
	\begin{cil}

.class public auto ansi beforefieldinit StartProc
       extends [PLR]PLR.Runtime.ProcessBase
{
  .method public specialname rtspecialname 
          instance void  .ctor() cil managed
  {
    // Code size       7 (0x7)
    .maxstack  2
    IL_0000:  ldarg.0
    IL_0001:  call       instance void [PLR]PLR.Runtime.ProcessBase::.ctor()
    IL_0006:  ret
  } // end of method StartProc::.ctor

  .method public virtual instance void  RunProcess() cil managed
  {
    .override [PLR]PLR.Runtime.ProcessBase::RunProcess
    // Code size       63 (0x3f)
    .maxstack  2
    .locals init ([0] class [PLR]PLR.Runtime.ProcessBase V_0,
             [1] class [PLR]PLR.Runtime.ProcessBase V_1)
    IL_0000:  ldarg.0
    IL_0001:  call       instance void [PLR]PLR.Runtime.ProcessBase
                         ::InitSetID()
    IL_0006:  nop
    IL_0007:  newobj     instance void ActionPrefix::.ctor()
    IL_000c:  stloc.0
    IL_000d:  ldloc.0
    IL_000e:  ldarg.0
    IL_000f:  call       instance class [PLR]PLR.Runtime.ProcessBase 
                         [PLR]PLR.Runtime.ProcessBase::get_Parent()
    IL_0014:  call       instance void [PLR]PLR.Runtime.ProcessBase
                         ::set_Parent(class [PLR]PLR.Runtime.ProcessBase)
    IL_0019:  ldloc.0
    IL_001a:  call       instance void [PLR]PLR.Runtime.ProcessBase::Run()
    IL_001f:  nop
    IL_0020:  newobj     instance void NonDeterministicChoice::.ctor()
    IL_0025:  stloc.1
    IL_0026:  ldloc.1
    IL_0027:  ldarg.0
    IL_0028:  call       instance class [PLR]PLR.Runtime.ProcessBase 
                         [PLR]PLR.Runtime.ProcessBase::get_Parent()
    IL_002d:  call       instance void [PLR]PLR.Runtime.ProcessBase
                         ::set_Parent(class [PLR]PLR.Runtime.ProcessBase)
    IL_0032:  ldloc.1
    IL_0033:  call       instance void [PLR]PLR.Runtime.ProcessBase::Run()
    IL_0038:  ldarg.0
    IL_0039:  call       instance void [PLR]PLR.Runtime.ProcessBase::Die()
    IL_003e:  ret
  } // end of method StartProc::RunProcess

} // end of class StartProc

\end{cil}

\section{ActionPrefix}
	
	\begin{verbatim}
  ActionPrefix = a .  0
	\end{verbatim}
	
	is compiled as follows:
	
	\begin{cil}

.class public auto ansi beforefieldinit ActionPrefix
       extends [PLR]PLR.Runtime.ProcessBase
{
  .method public specialname rtspecialname 
          instance void  .ctor() cil managed
  {
    // Code size       7 (0x7)
    .maxstack  2
    IL_0000:  ldarg.0
    IL_0001:  call       instance void [PLR]PLR.Runtime.ProcessBase::.ctor()
    IL_0006:  ret
  } // end of method ActionPrefix::.ctor

  .method public virtual instance void  RunProcess() cil managed
  {
    .override [PLR]PLR.Runtime.ProcessBase::RunProcess
    // Code size       84 (0x54)
    .maxstack  10
    .locals init ([0] class [PLR]PLR.Runtime.ChannelSyncAction V_0)
    IL_0000:  ldarg.0
    IL_0001:  call       instance void [PLR]PLR.Runtime.ProcessBase
                         ::InitSetID()
    .try
    {
      IL_0006:  ldarg.0
      IL_0007:  ldstr      "Preparing to sync now..."
      IL_000c:  call       instance void [PLR]PLR.Runtime.ProcessBase
                           ::Debug(string)
      IL_0011:  ldarg.0
      IL_0012:  ldstr      "a"
      IL_0017:  ldarg.0
      IL_0018:  ldc.i4     0x0
      IL_001d:  ldc.i4.1
      IL_001e:  newobj     instance void [PLR]PLR.Runtime.ChannelSyncAction
                           ::.ctor(string, class [PLR]PLR.Runtime.ProcessBase
                                   ,int32, bool)
      IL_0023:  stloc.0
      IL_0024:  ldloc.0
      IL_0025:  call       instance void [PLR]PLR.Runtime.ProcessBase
                           ::Sync(class [PLR]PLR.Runtime.IAction)
      IL_002a:  nop
      IL_002b:  nop
      IL_002c:  ldarg.0
      IL_002d:  ldstr      "Turned into 0"
      IL_0032:  call       instance void [PLR]PLR.Runtime.ProcessBase
                           ::Debug(string)
      IL_0037:  leave      IL_004d

    }  // end .try
    catch [PLR]PLR.Runtime.ProcessKilledException 
    {
      IL_003c:  pop
      IL_003d:  ldarg.0
      IL_003e:  ldstr      "Caught ProcessKilledException"
      IL_0043:  call       instance void [PLR]PLR.Runtime.ProcessBase
                           ::Debug(string)
      IL_0048:  leave      IL_004d

    }  // end handler
    IL_004d:  ldarg.0
    IL_004e:  call       instance void [PLR]PLR.Runtime.ProcessBase::Die()
    IL_0053:  ret
  } // end of method ActionPrefix::RunProcess

} // end of class ActionPrefix

\end{cil}

\section{ValuePassSend and ValuePassReceive}

	\begin{verbatim}
  ValuePassSend(x) = _a_(x+3 /:Rand(2)) . 0
  ValuePassReceive = a(y) . 0
	\end{verbatim}
	
	are compiled as follows:

\begin{cil}
.class public auto ansi beforefieldinit ValuePassSend_1
       extends [PLR]PLR.Runtime.ProcessBase
{
  .field assembly object x
  .method public specialname rtspecialname 
          instance void  .ctor(object x) cil managed
  {
    // Code size       19 (0x13)
    .maxstack  4
    IL_0000:  ldarg.0
    IL_0001:  call       instance void [PLR]PLR.Runtime.ProcessBase::.ctor()
    IL_0006:  ldarg.0
    IL_0007:  ldarg      x
    IL_000b:  nop
    IL_000c:  nop
    IL_000d:  stfld      object ValuePassSend_1::x
    IL_0012:  ret
  } // end of method ValuePassSend_1::.ctor

  .method public virtual instance void  RunProcess() cil managed
  {
    .override [PLR]PLR.Runtime.ProcessBase::RunProcess
    // Code size       125 (0x7d)
    .maxstack  10
    .locals init ([0] object x,
             [1] class [PLR]PLR.Runtime.ChannelSyncAction V_1)
    IL_0000:  ldarg.0
    IL_0001:  call       instance void [PLR]PLR.Runtime.ProcessBase
                         ::InitSetID()
    .try
    {
      IL_0006:  ldarg.0
      IL_0007:  ldfld      object ValuePassSend_1::x
      IL_000c:  stloc.0
      IL_000d:  ldarg.0
      IL_000e:  ldstr      "Preparing to sync now..."
      IL_0013:  call       instance void [PLR]PLR.Runtime.ProcessBase
                           ::Debug(string)
      IL_0018:  ldarg.0
      IL_0019:  ldstr      "a"
      IL_001e:  ldarg.0
      IL_001f:  ldc.i4     0x1
      IL_0024:  ldc.i4.0
      IL_0025:  newobj     instance void [PLR]PLR.Runtime.ChannelSyncAction
                           ::.ctor(string,class [PLR]PLR.Runtime.ProcessBase,
                           int32, bool)
      IL_002a:  stloc.1
      IL_002b:  ldloc.1
      IL_002c:  ldloc.0
      IL_002d:  unbox.any  [mscorlib]System.Int32
      IL_0032:  ldc.i4     0x3
      IL_0037:  ldc.i4     0x2
      IL_003c:  call       int32 [PLR]PLR.Runtime.BuiltIns::Rand(int32)
      IL_0041:  div
      IL_0042:  add.ovf
      IL_0043:  box        [mscorlib]System.Int32
      IL_0048:  call       instance void [PLR]PLR.Runtime.ChannelSyncAction
                           ::AddValue(object)
      IL_004d:  ldloc.1
      IL_004e:  call       instance void [PLR]PLR.Runtime.ProcessBase
                           ::Sync(class [PLR]PLR.Runtime.IAction)
      IL_0053:  nop
      IL_0054:  nop
      IL_0055:  ldarg.0
      IL_0056:  ldstr      "Turned into 0"
      IL_005b:  call       instance void [PLR]PLR.Runtime.ProcessBase
                           ::Debug(string)
      IL_0060:  leave      IL_0076

    }  // end .try
    catch [PLR]PLR.Runtime.ProcessKilledException 
    {
      IL_0065:  pop
      IL_0066:  ldarg.0
      IL_0067:  ldstr      "Caught ProcessKilledException"
      IL_006c:  call       instance void [PLR]PLR.Runtime.ProcessBase
                           ::Debug(string)
      IL_0071:  leave      IL_0076

    }  // end handler
    IL_0076:  ldarg.0
    IL_0077:  call       instance void [PLR]PLR.Runtime.ProcessBase::Die()
    IL_007c:  ret
  } // end of method ValuePassSend_1::RunProcess

} // end of class ValuePassSend_1

.class public auto ansi beforefieldinit ValuePassReceive
       extends [PLR]PLR.Runtime.ProcessBase
{
  .method public specialname rtspecialname 
          instance void  .ctor() cil managed
  {
    // Code size       7 (0x7)
    .maxstack  2
    IL_0000:  ldarg.0
    IL_0001:  call       instance void [PLR]PLR.Runtime.ProcessBase::.ctor()
    IL_0006:  ret
  } // end of method ValuePassReceive::.ctor

  .method public virtual instance void  RunProcess() cil managed
  {
    .override [PLR]PLR.Runtime.ProcessBase::RunProcess
    // Code size       96 (0x60)
    .maxstack  10
    .locals init ([0] class [PLR]PLR.Runtime.ChannelSyncAction V_0,
             [1] object y)
    IL_0000:  ldarg.0
    IL_0001:  call       instance void [PLR]PLR.Runtime.ProcessBase
                         ::InitSetID()
    .try
    {
      IL_0006:  ldarg.0
      IL_0007:  ldstr      "Preparing to sync now..."
      IL_000c:  call       instance void [PLR]PLR.Runtime.ProcessBase
                           ::Debug(string)
      IL_0011:  ldarg.0
      IL_0012:  ldstr      "a"
      IL_0017:  ldarg.0
      IL_0018:  ldc.i4     0x1
      IL_001d:  ldc.i4.1
      IL_001e:  newobj     instance void [PLR]PLR.Runtime.ChannelSyncAction
                           ::.ctor(string,class [PLR]PLR.Runtime.ProcessBase,
                           int32, bool)
      IL_0023:  stloc.0
      IL_0024:  ldloc.0
      IL_0025:  call       instance void [PLR]PLR.Runtime.ProcessBase
                           ::Sync(class [PLR]PLR.Runtime.IAction)
      IL_002a:  nop
      IL_002b:  ldloc.0
      IL_002c:  ldc.i4     0x0
      IL_0031:  call       instance object [PLR]PLR.Runtime.ChannelSyncAction
                           ::GetValue(int32)
      IL_0036:  stloc.1
      IL_0037:  nop
      IL_0038:  ldarg.0
      IL_0039:  ldstr      "Turned into 0"
      IL_003e:  call       instance void [PLR]PLR.Runtime.ProcessBase
                           ::Debug(string)
      IL_0043:  leave      IL_0059

    }  // end .try
    catch [PLR]PLR.Runtime.ProcessKilledException 
    {
      IL_0048:  pop
      IL_0049:  ldarg.0
      IL_004a:  ldstr      "Caught ProcessKilledException"
      IL_004f:  call       instance void [PLR]PLR.Runtime.ProcessBase
                           ::Debug(string)
      IL_0054:  leave      IL_0059

    }  // end handler
    IL_0059:  ldarg.0
    IL_005a:  call       instance void [PLR]PLR.Runtime.ProcessBase::Die()
    IL_005f:  ret
  } // end of method ValuePassReceive::RunProcess

} // end of class ValuePassReceive

\end{cil}

\section{NonDeterministicChoice}

	\begin{verbatim}
  NonDeterministicChoice = _a_ . 0 + NDC2
  NDC2 = b . 0
	\end{verbatim}
	
	are compiled as follows:

\begin{cil}
.class public auto ansi beforefieldinit NonDeterministicChoice
       extends [PLR]PLR.Runtime.ProcessBase
{
  .class auto ansi nested public beforefieldinit NonDeterministic1
         extends [PLR]PLR.Runtime.ProcessBase
  {
    .method public virtual instance void 
            RunProcess() cil managed
    {
      .override [PLR]PLR.Runtime.ProcessBase::RunProcess
      // Code size       84 (0x54)
      .maxstack  10
      .locals init ([0] class [PLR]PLR.Runtime.ChannelSyncAction V_0)
      IL_0000:  ldarg.0
      IL_0001:  call       instance void [PLR]PLR.Runtime.ProcessBase
                           ::InitSetID()
      .try
      {
        IL_0006:  ldarg.0
        IL_0007:  ldstr      "Preparing to sync now..."
        IL_000c:  call       instance void [PLR]PLR.Runtime.ProcessBase
                             ::Debug(string)
        IL_0011:  ldarg.0
        IL_0012:  ldstr      "a"
        IL_0017:  ldarg.0
        IL_0018:  ldc.i4     0x0
        IL_001d:  ldc.i4.0
        IL_001e:  newobj     instance void [PLR]PLR.Runtime.ChannelSyncAction
                             ::.ctor(string,
                             class [PLR]PLR.Runtime.ProcessBase, int32, bool)
        IL_0023:  stloc.0
        IL_0024:  ldloc.0
        IL_0025:  call       instance void [PLR]PLR.Runtime.ProcessBase
                             ::Sync(class [PLR]PLR.Runtime.IAction)
        IL_002a:  nop
        IL_002b:  nop
        IL_002c:  ldarg.0
        IL_002d:  ldstr      "Turned into 0"
        IL_0032:  call       instance void [PLR]PLR.Runtime.ProcessBase
                             ::Debug(string)
        IL_0037:  leave      IL_004d

      }  // end .try
      catch [PLR]PLR.Runtime.ProcessKilledException 
      {
        IL_003c:  pop
        IL_003d:  ldarg.0
        IL_003e:  ldstr      "Caught ProcessKilledException"
        IL_0043:  call       instance void [PLR]PLR.Runtime.ProcessBase
                             ::Debug(string)
        IL_0048:  leave      IL_004d

      }  // end handler
      IL_004d:  ldarg.0
      IL_004e:  call       instance void [PLR]PLR.Runtime.ProcessBase::Die()
      IL_0053:  ret
    } // end of method NonDeterministic1::RunProcess

    .method public specialname rtspecialname 
            instance void  .ctor() cil managed
    {
      // Code size       7 (0x7)
      .maxstack  2
      IL_0000:  ldarg.0
      IL_0001:  call       instance void [PLR]PLR.Runtime.ProcessBase
                           ::.ctor()
      IL_0006:  ret
    } // end of method NonDeterministic1::.ctor

  } // end of class NonDeterministic1

  .method public specialname rtspecialname 
          instance void  .ctor() cil managed
  {
    // Code size       7 (0x7)
    .maxstack  2
    IL_0000:  ldarg.0
    IL_0001:  call       instance void [PLR]PLR.Runtime.ProcessBase
                         ::.ctor()
    IL_0006:  ret
  } // end of method NonDeterministicChoice::.ctor

  .method public virtual instance void  RunProcess() cil managed
  {
    .override [PLR]PLR.Runtime.ProcessBase::RunProcess
    // Code size       85 (0x55)
    .maxstack  2
    .locals init (class [PLR]PLR.Runtime.ProcessBase V_0,
             class [PLR]PLR.Runtime.ProcessBase V_1)
    IL_0000:  ldarg.0
    IL_0001:  call       instance void [PLR]PLR.Runtime.ProcessBase
                         ::InitSetID()
    IL_0006:  newobj     instance void 
                         NonDeterministicChoice/NonDeterministic1::.ctor()
    IL_000b:  stloc.0
    IL_000c:  ldloc.0
    IL_000d:  ldarg.0
    IL_000e:  call       instance class [PLR]PLR.Runtime.ProcessBase 
                         [PLR]PLR.Runtime.ProcessBase::get_Parent()
    IL_0013:  call       instance void [PLR]PLR.Runtime.ProcessBase
                         ::set_Parent(class [PLR]PLR.Runtime.ProcessBase)
    IL_0018:  ldloc.0
    IL_0019:  ldarg.0
    IL_001a:  call       instance valuetype [mscorlib]System.Guid 
                         [PLR]PLR.Runtime.ProcessBase::get_SetID()
    IL_001f:  call       instance void [PLR]PLR.Runtime.ProcessBase
                         ::set_SetID(valuetype [mscorlib]System.Guid)
    IL_0024:  ldloc.0
    IL_0025:  call       instance void [PLR]PLR.Runtime.ProcessBase::Run()
    IL_002a:  newobj     instance void NDC2::.ctor()
    IL_002f:  stloc.1
    IL_0030:  ldloc.1
    IL_0031:  ldarg.0
    IL_0032:  call       instance class [PLR]PLR.Runtime.ProcessBase 
                         [PLR]PLR.Runtime.ProcessBase::get_Parent()
    IL_0037:  call       instance void [PLR]PLR.Runtime.ProcessBase
                         ::set_Parent(class [PLR]PLR.Runtime.ProcessBase)
    IL_003c:  ldloc.1
    IL_003d:  ldarg.0
    IL_003e:  call       instance valuetype [mscorlib]System.Guid 
                         [PLR]PLR.Runtime.ProcessBase::get_SetID()
    IL_0043:  call       instance void [PLR]PLR.Runtime.ProcessBase
                         ::set_SetID(valuetype [mscorlib]System.Guid)
    IL_0048:  ldloc.1
    IL_0049:  call       instance void [PLR]PLR.Runtime.ProcessBase::Run()
    IL_004e:  ldarg.0
    IL_004f:  call       instance void [PLR]PLR.Runtime.ProcessBase::Die()
    IL_0054:  ret
  } // end of method NonDeterministicChoice::RunProcess

} // end of class NonDeterministicChoice

.class public auto ansi beforefieldinit NDC2
       extends [PLR]PLR.Runtime.ProcessBase
{
  .method public specialname rtspecialname 
          instance void  .ctor() cil managed
  {
    // Code size       7 (0x7)
    .maxstack  2
    IL_0000:  ldarg.0
    IL_0001:  call       instance void [PLR]PLR.Runtime.ProcessBase::.ctor()
    IL_0006:  ret
  } // end of method NDC2::.ctor

  .method public virtual instance void  RunProcess() cil managed
  {
    .override [PLR]PLR.Runtime.ProcessBase::RunProcess
    // Code size       84 (0x54)
    .maxstack  10
    .locals init ([0] class [PLR]PLR.Runtime.ChannelSyncAction V_0)
    IL_0000:  ldarg.0
    IL_0001:  call       instance void [PLR]PLR.Runtime.ProcessBase
                         ::InitSetID()
    .try
    {
      IL_0006:  ldarg.0
      IL_0007:  ldstr      "Preparing to sync now..."
      IL_000c:  call       instance void [PLR]PLR.Runtime.ProcessBase
                           ::Debug(string)
      IL_0011:  ldarg.0
      IL_0012:  ldstr      "b"
      IL_0017:  ldarg.0
      IL_0018:  ldc.i4     0x0
      IL_001d:  ldc.i4.1
      IL_001e:  newobj     instance void [PLR]PLR.Runtime.ChannelSyncAction
                           ::.ctor(string,class [PLR]PLR.Runtime.ProcessBase,
                           int32, bool)
      IL_0023:  stloc.0
      IL_0024:  ldloc.0
      IL_0025:  call       instance void [PLR]PLR.Runtime.ProcessBase
                           ::Sync(class [PLR]PLR.Runtime.IAction)
      IL_002a:  nop
      IL_002b:  nop
      IL_002c:  ldarg.0
      IL_002d:  ldstr      "Turned into 0"
      IL_0032:  call       instance void [PLR]PLR.Runtime.ProcessBase
                           ::Debug(string)
      IL_0037:  leave      IL_004d

    }  // end .try
    catch [PLR]PLR.Runtime.ProcessKilledException 
    {
      IL_003c:  pop
      IL_003d:  ldarg.0
      IL_003e:  ldstr      "Caught ProcessKilledException"
      IL_0043:  call       instance void [PLR]PLR.Runtime.ProcessBase
                           ::Debug(string)
      IL_0048:  leave      IL_004d

    }  // end handler
    IL_004d:  ldarg.0
    IL_004e:  call       instance void [PLR]PLR.Runtime.ProcessBase::Die()
    IL_0053:  ret
  } // end of method NDC2::RunProcess

} // end of class NDC2

\end{cil}

\section{ParallelComposition}

	\begin{verbatim}
  ParallelComposition = a . 0 | PC2 | 0 
  PC2 = b . 0
	\end{verbatim}
	
	are compiled as follows:

\begin{cil}

.class public auto ansi beforefieldinit ParallelComposition
       extends [PLR]PLR.Runtime.ProcessBase
{
  .class auto ansi nested public beforefieldinit Parallel1
         extends [PLR]PLR.Runtime.ProcessBase
  {
    .method public virtual instance void 
            RunProcess() cil managed
    {
      .override [PLR]PLR.Runtime.ProcessBase::RunProcess
      // Code size       84 (0x54)
      .maxstack  10
      .locals init ([0] class [PLR]PLR.Runtime.ChannelSyncAction V_0)
      IL_0000:  ldarg.0
      IL_0001:  call       instance void [PLR]PLR.Runtime.ProcessBase
                           ::InitSetID()
      .try
      {
        IL_0006:  ldarg.0
        IL_0007:  ldstr      "Preparing to sync now..."
        IL_000c:  call       instance void [PLR]PLR.Runtime.ProcessBase
                             ::Debug(string)
        IL_0011:  ldarg.0
        IL_0012:  ldstr      "a"
        IL_0017:  ldarg.0
        IL_0018:  ldc.i4     0x0
        IL_001d:  ldc.i4.1
        IL_001e:  newobj     instance void [PLR]PLR.Runtime.ChannelSyncAction
                             ::.ctor(string,class 
                             [PLR]PLR.Runtime.ProcessBase, int32, bool)
        IL_0023:  stloc.0
        IL_0024:  ldloc.0
        IL_0025:  call       instance void [PLR]PLR.Runtime.ProcessBase
                             ::Sync(class [PLR]PLR.Runtime.IAction)
        IL_002a:  nop
        IL_002b:  nop
        IL_002c:  ldarg.0
        IL_002d:  ldstr      "Turned into 0"
        IL_0032:  call       instance void [PLR]PLR.Runtime.ProcessBase
                             ::Debug(string)
        IL_0037:  leave      IL_004d

      }  // end .try
      catch [PLR]PLR.Runtime.ProcessKilledException 
      {
        IL_003c:  pop
        IL_003d:  ldarg.0
        IL_003e:  ldstr      "Caught ProcessKilledException"
        IL_0043:  call       instance void [PLR]PLR.Runtime.ProcessBase
                             ::Debug(string)
        IL_0048:  leave      IL_004d

      }  // end handler
      IL_004d:  ldarg.0
      IL_004e:  call       instance void [PLR]PLR.Runtime.ProcessBase::Die()
      IL_0053:  ret
    } // end of method Parallel1::RunProcess

    .method public specialname rtspecialname 
            instance void  .ctor() cil managed
    {
      // Code size       7 (0x7)
      .maxstack  2
      IL_0000:  ldarg.0
      IL_0001:  call       instance void [PLR]PLR.Runtime.ProcessBase
                           ::.ctor()
      IL_0006:  ret
    } // end of method Parallel1::.ctor

  } // end of class Parallel1

  .class auto ansi nested public beforefieldinit Parallel3
         extends [PLR]PLR.Runtime.ProcessBase
  {
    .method public virtual instance void 
            RunProcess() cil managed
    {
      .override [PLR]PLR.Runtime.ProcessBase::RunProcess
      // Code size       25 (0x19)
      .maxstack  2
      IL_0000:  ldarg.0
      IL_0001:  call       instance void [PLR]PLR.Runtime.ProcessBase
                           ::InitSetID()
      IL_0006:  nop
      IL_0007:  ldarg.0
      IL_0008:  ldstr      "Turned into 0"
      IL_000d:  call       instance void [PLR]PLR.Runtime.ProcessBase
                           ::Debug(string)
      IL_0012:  ldarg.0
      IL_0013:  call       instance void [PLR]PLR.Runtime.ProcessBase
                           ::Die()
      IL_0018:  ret
    } // end of method Parallel3::RunProcess

    .method public specialname rtspecialname 
            instance void  .ctor() cil managed
    {
      // Code size       7 (0x7)
      .maxstack  2
      IL_0000:  ldarg.0
      IL_0001:  call       instance void [PLR]PLR.Runtime.ProcessBase
                           ::.ctor()
      IL_0006:  ret
    } // end of method Parallel3::.ctor

  } // end of class Parallel3

  .method public specialname rtspecialname 
          instance void  .ctor() cil managed
  {
    // Code size       7 (0x7)
    .maxstack  2
    IL_0000:  ldarg.0
    IL_0001:  call       instance void [PLR]PLR.Runtime.ProcessBase
                         ::.ctor()
    IL_0006:  ret
  } // end of method ParallelComposition::.ctor

  .method public virtual instance void  RunProcess() cil managed
  {
    .override [PLR]PLR.Runtime.ProcessBase::RunProcess
    // Code size       86 (0x56)
    .maxstack  2
    .locals init ([0] class [PLR]PLR.Runtime.ProcessBase V_0,
             [1] class [PLR]PLR.Runtime.ProcessBase V_1,
             [2] class [PLR]PLR.Runtime.ProcessBase V_2)
    IL_0000:  ldarg.0
    IL_0001:  call       instance void [PLR]PLR.Runtime.ProcessBase+
                         ::InitSetID()
    IL_0006:  newobj     instance void ParallelComposition/Parallel1
                         ::.ctor()
    IL_000b:  stloc.0
    IL_000c:  ldloc.0
    IL_000d:  ldarg.0
    IL_000e:  call       instance class [PLR]PLR.Runtime.ProcessBase 
                         [PLR]PLR.Runtime.ProcessBase::get_Parent()
    IL_0013:  call       instance void [PLR]PLR.Runtime.ProcessBase
                         ::set_Parent(class [PLR]PLR.Runtime.ProcessBase)
    IL_0018:  ldloc.0
    IL_0019:  call       instance void [PLR]PLR.Runtime.ProcessBase::Run()
    IL_001e:  nop
    IL_001f:  newobj     instance void PC2::.ctor()
    IL_0024:  stloc.1
    IL_0025:  ldloc.1
    IL_0026:  ldarg.0
    IL_0027:  call       instance class [PLR]PLR.Runtime.ProcessBase 
                         [PLR]PLR.Runtime.ProcessBase::get_Parent()
    IL_002c:  call       instance void [PLR]PLR.Runtime.ProcessBase
                         ::set_Parent(class [PLR]PLR.Runtime.ProcessBase)
    IL_0031:  ldloc.1
    IL_0032:  call       instance void [PLR]PLR.Runtime.ProcessBase::Run()
    IL_0037:  newobj     instance void ParallelComposition/Parallel3::.ctor()
    IL_003c:  stloc.2
    IL_003d:  ldloc.2
    IL_003e:  ldarg.0
    IL_003f:  call       instance class [PLR]PLR.Runtime.ProcessBase 
                         [PLR]PLR.Runtime.ProcessBase::get_Parent()
    IL_0044:  call       instance void [PLR]PLR.Runtime.ProcessBase
                         ::set_Parent(class [PLR]PLR.Runtime.ProcessBase)
    IL_0049:  ldloc.2
    IL_004a:  call       instance void [PLR]PLR.Runtime.ProcessBase::Run()
    IL_004f:  ldarg.0
    IL_0050:  call       instance void [PLR]PLR.Runtime.ProcessBase::Die()
    IL_0055:  ret
  } // end of method ParallelComposition::RunProcess

} // end of class ParallelComposition

.class public auto ansi beforefieldinit PC2
       extends [PLR]PLR.Runtime.ProcessBase
{
  .method public specialname rtspecialname 
          instance void  .ctor() cil managed
  {
    // Code size       7 (0x7)
    .maxstack  2
    IL_0000:  ldarg.0
    IL_0001:  call       instance void [PLR]PLR.Runtime.ProcessBase::.ctor()
    IL_0006:  ret
  } // end of method PC2::.ctor

  .method public virtual instance void  RunProcess() cil managed
  {
    .override [PLR]PLR.Runtime.ProcessBase::RunProcess
    // Code size       84 (0x54)
    .maxstack  10
    .locals init ([0] class [PLR]PLR.Runtime.ChannelSyncAction V_0)
    IL_0000:  ldarg.0
    IL_0001:  call       instance void [PLR]PLR.Runtime.ProcessBase
                         ::InitSetID()
    .try
    {
      IL_0006:  ldarg.0
      IL_0007:  ldstr      "Preparing to sync now..."
      IL_000c:  call       instance void [PLR]PLR.Runtime.ProcessBase
                           ::Debug(string)
      IL_0011:  ldarg.0
      IL_0012:  ldstr      "b"
      IL_0017:  ldarg.0
      IL_0018:  ldc.i4     0x0
      IL_001d:  ldc.i4.1
      IL_001e:  newobj     instance void [PLR]PLR.Runtime.ChannelSyncAction
                           ::.ctor(string, 
                           class [PLR]PLR.Runtime.ProcessBase, int32, bool)
      IL_0023:  stloc.0
      IL_0024:  ldloc.0
      IL_0025:  call       instance void [PLR]PLR.Runtime.ProcessBase
                           ::Sync(class [PLR]PLR.Runtime.IAction)
      IL_002a:  nop
      IL_002b:  nop
      IL_002c:  ldarg.0
      IL_002d:  ldstr      "Turned into 0"
      IL_0032:  call       instance void [PLR]PLR.Runtime.ProcessBase
                           ::Debug(string)
      IL_0037:  leave      IL_004d

    }  // end .try
    catch [PLR]PLR.Runtime.ProcessKilledException 
    {
      IL_003c:  pop
      IL_003d:  ldarg.0
      IL_003e:  ldstr      "Caught ProcessKilledException"
      IL_0043:  call       instance void [PLR]PLR.Runtime.ProcessBase
                           ::Debug(string)
      IL_0048:  leave      IL_004d

    }  // end handler
    IL_004d:  ldarg.0
    IL_004e:  call       instance void [PLR]PLR.Runtime.ProcessBase::Die()
    IL_0053:  ret
  } // end of method PC2::RunProcess

} // end of class PC2
\end{cil}

\section{MethodCall}

	\begin{verbatim}
	MethodCall = :Print("Hello") . 0
	\end{verbatim}
	
	is compiled as follows:

\begin{cil}

.class public auto ansi beforefieldinit MethodCall
       extends [PLR]PLR.Runtime.ProcessBase
{
  .method public specialname rtspecialname 
          instance void  .ctor() cil managed
  {
    // Code size       7 (0x7)
    .maxstack  2
    IL_0000:  ldarg.0
    IL_0001:  call       instance void [PLR]PLR.Runtime.ProcessBase::.ctor()
    IL_0006:  ret
  } // end of method MethodCall::.ctor

  .method public virtual instance void  RunProcess() cil managed
  {
    .override [PLR]PLR.Runtime.ProcessBase::RunProcess
    // Code size       86 (0x56)
    .maxstack  8
    IL_0000:  ldarg.0
    IL_0001:  call       instance void [PLR]PLR.Runtime.ProcessBase
                         ::InitSetID()
    .try
    {
      IL_0006:  ldarg.0
      IL_0007:  ldstr      "Preparing to sync now..."
      IL_000c:  call       instance void [PLR]PLR.Runtime.ProcessBase
                           ::Debug(string)
      IL_0011:  ldarg.0
      IL_0012:  ldstr      "Print(\"Hello\")"
      IL_0017:  ldarg.0
      IL_0018:  newobj     instance void [PLR]PLR.Runtime.MethodCallAction
                           ::.ctor(string, 
                           class [PLR]PLR.Runtime.ProcessBase)
      IL_001d:  call       instance void [PLR]PLR.Runtime.ProcessBase
                           ::Sync(class [PLR]PLR.Runtime.IAction)
      IL_0022:  nop
      IL_0023:  ldstr      "Hello"
      IL_0028:  call       void [PLR]PLR.Runtime.BuiltIns::Print(object)
      IL_002d:  nop
      IL_002e:  ldarg.0
      IL_002f:  ldstr      "Turned into 0"
      IL_0034:  call       instance void [PLR]PLR.Runtime.ProcessBase
                           ::Debug(string)
      IL_0039:  leave      IL_004f

    }  // end .try
    catch [PLR]PLR.Runtime.ProcessKilledException 
    {
      IL_003e:  pop
      IL_003f:  ldarg.0
      IL_0040:  ldstr      "Caught ProcessKilledException"
      IL_0045:  call       instance void [PLR]PLR.Runtime.ProcessBase
                           ::Debug(string)
      IL_004a:  leave      IL_004f

    }  // end handler
    IL_004f:  ldarg.0
    IL_0050:  call       instance void [PLR]PLR.Runtime.ProcessBase
                         ::Die()
    IL_0055:  ret
  } // end of method MethodCall::RunProcess

} // end of class MethodCall

\end{cil}

\section{Restrict}

	\begin{verbatim}
  Restrict = ( a . (d . 0)\ d ) \{a}
	\end{verbatim}
	
	is compiled as follows:

\begin{cil}

.class public auto ansi beforefieldinit Restrict
       extends [PLR]PLR.Runtime.ProcessBase
{
  .class auto ansi nested public beforefieldinit Inner
         extends [PLR]PLR.Runtime.ProcessBase
  {
    .method public static bool  RestrictByName(
        class [PLR]PLR.Runtime.IAction A_0) cil managed
    {
      // Code size       37 (0x25)
      .maxstack  3
      .locals init ([0] class [PLR]PLR.Runtime.ChannelSyncAction V_0)
      IL_0000:  ldarg.0
      IL_0001:  isinst     [PLR]PLR.Runtime.ChannelSyncAction
      IL_0006:  brtrue     IL_000d

      IL_000b:  ldc.i4.0
      IL_000c:  ret

      IL_000d:  ldarg.0
      IL_000e:  castclass  [PLR]PLR.Runtime.ChannelSyncAction
      IL_0013:  stloc.0
      IL_0014:  ldloc.0
      IL_0015:  call       instance string [PLR]PLR.Runtime.ChannelSyncAction
                           ::get_Name()
      IL_001a:  ldstr      "d"
      IL_001f:  call       bool [mscorlib]System.String::op_Equality(string,
                                                                     string)
      IL_0024:  ret
    } // end of method Inner::RestrictByName

    .method public virtual instance class [PLR]PLR.Runtime.RestrictAction 
            get_Restrict() cil managed
    {
      .override [PLR]PLR.Runtime.ProcessBase::get_Restrict
      // Code size       13 (0xd)
      .maxstack  3
      IL_0000:  ldnull
      IL_0001:  ldftn      bool Restrict/Inner::RestrictByName(class 
                           [PLR]PLR.Runtime.IAction)
      IL_0007:  newobj     instance void [PLR]PLR.Runtime.RestrictAction
                           ::.ctor(object, native int)
      IL_000c:  ret
    } // end of method Inner::get_Restrict

    .method public virtual instance void 
            RunProcess() cil managed
    {
      .override [PLR]PLR.Runtime.ProcessBase::RunProcess
      // Code size       84 (0x54)
      .maxstack  10
      .locals init ([0] class [PLR]PLR.Runtime.ChannelSyncAction V_0)
      IL_0000:  ldarg.0
      IL_0001:  call       instance void [PLR]PLR.Runtime.ProcessBase
                           ::InitSetID()
      .try
      {
        IL_0006:  ldarg.0
        IL_0007:  ldstr      "Preparing to sync now..."
        IL_000c:  call       instance void [PLR]PLR.Runtime.ProcessBase
                             ::Debug(string)
        IL_0011:  ldarg.0
        IL_0012:  ldstr      "d"
        IL_0017:  ldarg.0
        IL_0018:  ldc.i4     0x0
        IL_001d:  ldc.i4.1
        IL_001e:  newobj     instance void [PLR]PLR.Runtime.ChannelSyncAction
                             ::.ctor(string, class 
                             [PLR]PLR.Runtime.ProcessBase, int32, bool)
        IL_0023:  stloc.0
        IL_0024:  ldloc.0
        IL_0025:  call       instance void [PLR]PLR.Runtime.ProcessBase
                             ::Sync(class [PLR]PLR.Runtime.IAction)
        IL_002a:  nop
        IL_002b:  nop
        IL_002c:  ldarg.0
        IL_002d:  ldstr      "Turned into 0"
        IL_0032:  call       instance void [PLR]PLR.Runtime.ProcessBase
                             ::Debug(string)
        IL_0037:  leave      IL_004d

      }  // end .try
      catch [PLR]PLR.Runtime.ProcessKilledException 
      {
        IL_003c:  pop
        IL_003d:  ldarg.0
        IL_003e:  ldstr      "Caught ProcessKilledException"
        IL_0043:  call       instance void [PLR]PLR.Runtime.ProcessBase
                             ::Debug(string)
        IL_0048:  leave      IL_004d

      }  // end handler
      IL_004d:  ldarg.0
      IL_004e:  call       instance void [PLR]PLR.Runtime.ProcessBase::Die()
      IL_0053:  ret
    } // end of method Inner::RunProcess

    .method public specialname rtspecialname 
            instance void  .ctor() cil managed
    {
      // Code size       7 (0x7)
      .maxstack  2
      IL_0000:  ldarg.0
      IL_0001:  call       instance void [PLR]PLR.Runtime.ProcessBase
                           ::.ctor()
      IL_0006:  ret
    } // end of method Inner::.ctor

  } // end of class Inner

  .method public specialname rtspecialname 
          instance void  .ctor() cil managed
  {
    // Code size       7 (0x7)
    .maxstack  2
    IL_0000:  ldarg.0
    IL_0001:  call       instance void [PLR]PLR.Runtime.ProcessBase
                         ::.ctor()
    IL_0006:  ret
  } // end of method Restrict::.ctor

  .method public static bool  RestrictByName(class [PLR]PLR.Runtime.IAction 
      A_0) cil managed
  {
    // Code size       37 (0x25)
    .maxstack  3
    .locals init ([0] class [PLR]PLR.Runtime.ChannelSyncAction V_0)
    IL_0000:  ldarg.0
    IL_0001:  isinst     [PLR]PLR.Runtime.ChannelSyncAction
    IL_0006:  brtrue     IL_000d

    IL_000b:  ldc.i4.0
    IL_000c:  ret

    IL_000d:  ldarg.0
    IL_000e:  castclass  [PLR]PLR.Runtime.ChannelSyncAction
    IL_0013:  stloc.0
    IL_0014:  ldloc.0
    IL_0015:  call       instance string [PLR]PLR.Runtime.ChannelSyncAction
                         ::get_Name()
    IL_001a:  ldstr      "a"
    IL_001f:  call       bool [mscorlib]System.String::op_Equality(string,
                                                                   string)
    IL_0024:  ret
  } // end of method Restrict::RestrictByName

  .method public virtual instance class [PLR]PLR.Runtime.RestrictAction 
          get_Restrict() cil managed
  {
    .override [PLR]PLR.Runtime.ProcessBase::get_Restrict
    // Code size       13 (0xd)
    .maxstack  3
    IL_0000:  ldnull
    IL_0001:  ldftn      bool Restrict::RestrictByName(class 
                         [PLR]PLR.Runtime.IAction)
    IL_0007:  newobj     instance void [PLR]PLR.Runtime.RestrictAction
                         ::.ctor(object, native int)
    IL_000c:  ret
  } // end of method Restrict::get_Restrict

  .method public virtual instance void  RunProcess() cil managed
  {
    .override [PLR]PLR.Runtime.ProcessBase::RunProcess
    // Code size       103 (0x67)
    .maxstack  10
    .locals init ([0] class [PLR]PLR.Runtime.ChannelSyncAction V_0,
             [1] class [PLR]PLR.Runtime.ProcessBase V_1)
    IL_0000:  ldarg.0
    IL_0001:  call       instance void [PLR]PLR.Runtime.ProcessBase
                         ::InitSetID()
    .try
    {
      IL_0006:  ldarg.0
      IL_0007:  ldstr      "Preparing to sync now..."
      IL_000c:  call       instance void [PLR]PLR.Runtime.ProcessBase
                           ::Debug(string)
      IL_0011:  ldarg.0
      IL_0012:  ldstr      "a"
      IL_0017:  ldarg.0
      IL_0018:  ldc.i4     0x0
      IL_001d:  ldc.i4.1
      IL_001e:  newobj     instance void [PLR]PLR.Runtime.ChannelSyncAction
                           ::.ctor(string, class [PLR]PLR.Runtime.ProcessBase
                           , int32, bool)
      IL_0023:  stloc.0
      IL_0024:  ldloc.0
      IL_0025:  call       instance void [PLR]PLR.Runtime.ProcessBase
                           ::Sync(class [PLR]PLR.Runtime.IAction)
      IL_002a:  nop
      IL_002b:  newobj     instance void Restrict/Inner::.ctor()
      IL_0030:  stloc.1
      IL_0031:  ldloc.1
      IL_0032:  ldarg.0
      IL_0033:  call       instance void [PLR]PLR.Runtime.ProcessBase
                           ::set_Parent(class [PLR]PLR.Runtime.ProcessBase)
      IL_0038:  ldloc.1
      IL_0039:  ldarg.0
      IL_003a:  call       instance valuetype [mscorlib]System.Guid 
                           [PLR]PLR.Runtime.ProcessBase::get_SetID()
      IL_003f:  call       instance void [PLR]PLR.Runtime.ProcessBase
                           ::set_SetID(valuetype [mscorlib]System.Guid)
      IL_0044:  ldloc.1
      IL_0045:  call       instance void [PLR]PLR.Runtime.ProcessBase::Run()
      IL_004a:  leave      IL_0060

    }  // end .try
    catch [PLR]PLR.Runtime.ProcessKilledException 
    {
      IL_004f:  pop
      IL_0050:  ldarg.0
      IL_0051:  ldstr      "Caught ProcessKilledException"
      IL_0056:  call       instance void [PLR]PLR.Runtime.ProcessBase
                           ::Debug(string)
      IL_005b:  leave      IL_0060

    }  // end handler
    IL_0060:  ldarg.0
    IL_0061:  call       instance void [PLR]PLR.Runtime.ProcessBase::Die()
    IL_0066:  ret
  } // end of method Restrict::RunProcess

} // end of class Restrict

\end{cil}

\section{Relabel}

	\begin{verbatim}
  Relabel = ( a . (d . 0)[dnew/d] )[anew/a]
	\end{verbatim}
	
	is compiled as follows:

\begin{cil}

.class public auto ansi beforefieldinit Relabel
       extends [PLR]PLR.Runtime.ProcessBase
{
  .class auto ansi nested public beforefieldinit Inner
         extends [PLR]PLR.Runtime.ProcessBase
  {
    .method public static class [PLR]PLR.Runtime.IAction 
            RelabelAction(class [PLR]PLR.Runtime.IAction A_0) cil managed
    {
      // Code size       59 (0x3b)
      .maxstack  4
      .locals init ([0] class [PLR]PLR.Runtime.ChannelSyncAction V_0)
      IL_0000:  ldarg.0
      IL_0001:  isinst     [PLR]PLR.Runtime.ChannelSyncAction
      IL_0006:  brtrue     IL_000d

      IL_000b:  ldarg.0
      IL_000c:  ret

      IL_000d:  ldarg.0
      IL_000e:  castclass  [PLR]PLR.Runtime.ChannelSyncAction
      IL_0013:  stloc.0
      IL_0014:  ldloc.0
      IL_0015:  call       instance string [PLR]PLR.Runtime.ChannelSyncAction
                           ::get_Name()
      IL_001a:  ldstr      "d"
      IL_001f:  call       bool [mscorlib]System.String::op_Equality(string,
                                                                     string)
      IL_0024:  brfalse    IL_0039

      IL_0029:  ldloc.0
      IL_002a:  ldstr      "dnew"
      IL_002f:  call       instance void [PLR]PLR.Runtime.ChannelSyncAction
                           ::set_Name(string)
      IL_0034:  br         IL_0039

      IL_0039:  ldloc.0
      IL_003a:  ret
    } // end of method Inner::RelabelAction

    .method public virtual instance class [PLR]PLR.Runtime.PreProcessAction 
            get_PreProcess() cil managed
    {
      .override [PLR]PLR.Runtime.ProcessBase::get_PreProcess
      // Code size       13 (0xd)
      .maxstack  3
      IL_0000:  ldnull
      IL_0001:  ldftn      class [PLR]PLR.Runtime.IAction Relabel/Inner
                           ::RelabelAction(class [PLR]PLR.Runtime.IAction)
      IL_0007:  newobj     instance void [PLR]PLR.Runtime.PreProcessAction
                           ::.ctor(object, native int)
      IL_000c:  ret
    } // end of method Inner::get_PreProcess

    .method public virtual instance void 
            RunProcess() cil managed
    {
      .override [PLR]PLR.Runtime.ProcessBase::RunProcess
      // Code size       84 (0x54)
      .maxstack  10
      .locals init ([0] class [PLR]PLR.Runtime.ChannelSyncAction V_0)
      IL_0000:  ldarg.0
      IL_0001:  call       instance void [PLR]PLR.Runtime.ProcessBase
                           ::InitSetID()
      .try
      {
        IL_0006:  ldarg.0
        IL_0007:  ldstr      "Preparing to sync now..."
        IL_000c:  call       instance void [PLR]PLR.Runtime.ProcessBase
                             ::Debug(string)
        IL_0011:  ldarg.0
        IL_0012:  ldstr      "d"
        IL_0017:  ldarg.0
        IL_0018:  ldc.i4     0x0
        IL_001d:  ldc.i4.1
        IL_001e:  newobj     instance void [PLR]PLR.Runtime.ChannelSyncAction
                             ::.ctor(string, class 
                             [PLR]PLR.Runtime.ProcessBase, int32, bool)
        IL_0023:  stloc.0
        IL_0024:  ldloc.0
        IL_0025:  call       instance void [PLR]PLR.Runtime.ProcessBase
                             ::Sync(class [PLR]PLR.Runtime.IAction)
        IL_002a:  nop
        IL_002b:  nop
        IL_002c:  ldarg.0
        IL_002d:  ldstr      "Turned into 0"
        IL_0032:  call       instance void [PLR]PLR.Runtime.ProcessBase
                             ::Debug(string)
        IL_0037:  leave      IL_004d

      }  // end .try
      catch [PLR]PLR.Runtime.ProcessKilledException 
      {
        IL_003c:  pop
        IL_003d:  ldarg.0
        IL_003e:  ldstr      "Caught ProcessKilledException"
        IL_0043:  call       instance void [PLR]PLR.Runtime.ProcessBase
                             ::Debug(string)
        IL_0048:  leave      IL_004d

      }  // end handler
      IL_004d:  ldarg.0
      IL_004e:  call       instance void [PLR]PLR.Runtime.ProcessBase::Die()
      IL_0053:  ret
    } // end of method Inner::RunProcess

    .method public specialname rtspecialname 
            instance void  .ctor() cil managed
    {
      // Code size       7 (0x7)
      .maxstack  2
      IL_0000:  ldarg.0
      IL_0001:  call       instance void [PLR]PLR.Runtime.ProcessBase
                           ::.ctor()
      IL_0006:  ret
    } // end of method Inner::.ctor

  } // end of class Inner

  .method public specialname rtspecialname 
          instance void  .ctor() cil managed
  {
    // Code size       7 (0x7)
    .maxstack  2
    IL_0000:  ldarg.0
    IL_0001:  call       instance void [PLR]PLR.Runtime.ProcessBase
                         ::.ctor()
    IL_0006:  ret
  } // end of method Relabel::.ctor

  .method public static class [PLR]PLR.Runtime.IAction 
          RelabelAction(class [PLR]PLR.Runtime.IAction A_0) cil managed
  {
    // Code size       59 (0x3b)
    .maxstack  4
    .locals init ([0] class [PLR]PLR.Runtime.ChannelSyncAction V_0)
    IL_0000:  ldarg.0
    IL_0001:  isinst     [PLR]PLR.Runtime.ChannelSyncAction
    IL_0006:  brtrue     IL_000d

    IL_000b:  ldarg.0
    IL_000c:  ret

    IL_000d:  ldarg.0
    IL_000e:  castclass  [PLR]PLR.Runtime.ChannelSyncAction
    IL_0013:  stloc.0
    IL_0014:  ldloc.0
    IL_0015:  call       instance string [PLR]PLR.Runtime.ChannelSyncAction
                         ::get_Name()
    IL_001a:  ldstr      "a"
    IL_001f:  call       bool [mscorlib]System.String::op_Equality(string,
                                                                   string)
    IL_0024:  brfalse    IL_0039

    IL_0029:  ldloc.0
    IL_002a:  ldstr      "anew"
    IL_002f:  call       instance void [PLR]PLR.Runtime.ChannelSyncAction
                         ::set_Name(string)
    IL_0034:  br         IL_0039

    IL_0039:  ldloc.0
    IL_003a:  ret
  } // end of method Relabel::RelabelAction

  .method public virtual instance class [PLR]PLR.Runtime.PreProcessAction 
          get_PreProcess() cil managed
  {
    .override [PLR]PLR.Runtime.ProcessBase::get_PreProcess
    // Code size       13 (0xd)
    .maxstack  3
    IL_0000:  ldnull
    IL_0001:  ldftn      class [PLR]PLR.Runtime.IAction Relabel
                         ::RelabelAction(class [PLR]PLR.Runtime.IAction)
    IL_0007:  newobj     instance void [PLR]PLR.Runtime.PreProcessAction
                         ::.ctor(object, native int)
    IL_000c:  ret
  } // end of method Relabel::get_PreProcess

  .method public virtual instance void  RunProcess() cil managed
  {
    .override [PLR]PLR.Runtime.ProcessBase::RunProcess
    // Code size       103 (0x67)
    .maxstack  10
    .locals init ([0] class [PLR]PLR.Runtime.ChannelSyncAction V_0,
             [1] class [PLR]PLR.Runtime.ProcessBase V_1)
    IL_0000:  ldarg.0
    IL_0001:  call       instance void [PLR]PLR.Runtime.ProcessBase
                         ::InitSetID()
    .try
    {
      IL_0006:  ldarg.0
      IL_0007:  ldstr      "Preparing to sync now..."
      IL_000c:  call       instance void [PLR]PLR.Runtime.ProcessBase
                           ::Debug(string)
      IL_0011:  ldarg.0
      IL_0012:  ldstr      "a"
      IL_0017:  ldarg.0
      IL_0018:  ldc.i4     0x0
      IL_001d:  ldc.i4.1
      IL_001e:  newobj     instance void [PLR]PLR.Runtime.ChannelSyncAction
                           ::.ctor(string, class 
                           [PLR]PLR.Runtime.ProcessBase, int32, bool)
      IL_0023:  stloc.0
      IL_0024:  ldloc.0
      IL_0025:  call       instance void [PLR]PLR.Runtime.ProcessBase
                           ::Sync(class [PLR]PLR.Runtime.IAction)
      IL_002a:  nop
      IL_002b:  newobj     instance void Relabel/Inner::.ctor()
      IL_0030:  stloc.1
      IL_0031:  ldloc.1
      IL_0032:  ldarg.0
      IL_0033:  call       instance void [PLR]PLR.Runtime.ProcessBase
                           ::set_Parent(class [PLR]PLR.Runtime.ProcessBase)
      IL_0038:  ldloc.1
      IL_0039:  ldarg.0
      IL_003a:  call       instance valuetype [mscorlib]System.Guid 
                           [PLR]PLR.Runtime.ProcessBase::get_SetID()
      IL_003f:  call       instance void [PLR]PLR.Runtime.ProcessBase
                           ::set_SetID(valuetype [mscorlib]System.Guid)
      IL_0044:  ldloc.1
      IL_0045:  call       instance void [PLR]PLR.Runtime.ProcessBase::Run()
      IL_004a:  leave      IL_0060

    }  // end .try
    catch [PLR]PLR.Runtime.ProcessKilledException 
    {
      IL_004f:  pop
      IL_0050:  ldarg.0
      IL_0051:  ldstr      "Caught ProcessKilledException"
      IL_0056:  call       instance void [PLR]PLR.Runtime.ProcessBase
                           ::Debug(string)
      IL_005b:  leave      IL_0060

    }  // end handler
    IL_0060:  ldarg.0
    IL_0061:  call       instance void [PLR]PLR.Runtime.ProcessBase::Die()
    IL_0066:  ret
  } // end of method Relabel::RunProcess

} // end of class Relabel

	\end{cil}

